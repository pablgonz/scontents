% \iffalse meta-comment
%
% Copyright (C) 2019-2025 by Pablo González L <pablgonz@educarchile.cl>
%
% This work may be distributed and/or modified under the conditions of the
% LaTeX Project Public License, either version 1.3c of this license or (at
% your option) any later version. The latest version of this license is in
%
%  https://www.latex-project.org/lppl.txt
%
% and version 1.3c or later is part of all distributions of LaTeX version
% 2005/12/01 or later.
%
% This work is "maintained" (as per the LPPL maintenance status)
% by Pablo González Luengo.
%
% This work consists of the files scontents.dtx and
%                                 scontents.ins,
% and the derived files           scontents.sty,
%                                 scontents.tex,
%                                 t-scontents.mkiv and
%                                 scontents-code.tex.
%
%<*driver>
\documentclass[full]{l3doc}
% imakeidx also opens a write stream for the .idx file, and that conflicts
% with the one opened by l3doc. Here we close that write stream:
\makeatletter
\immediate\closeout\@indexfile
\makeatother
% and later we will copy the write stream opened by imakeidx into \@indexfile
% so that entries written to both streams end up in the same file.
\usepackage[english]{babel}
\usepackage[top=0.5in,bottom=0.3in,left=2in,right=0.7in,footskip=0.2in,headheight=1cm,headsep=0.27cm]{geometry}
\usepackage[llscaled=0.95]{libertine} % need by setup-texlive-action@v3
\usepackage[osf,nomath,mono=false,ScaleSF=0.95,ScaleRM=0.95]{libertinus-otf}
% Set Monospace font, fix 'quotes' in verbatim
\usepackage{sourcecodepro}
\defaultfontfeatures[\ttfamily]
  {
    Numbers   = OldStyle,
    Scale     = 0.80,
    Extension = .otf,
  }
\setmonofont[
    UprightFont    = *-Regular,
    ItalicFont     = *-RegularIt,
    BoldFont       = *-Semibold,
    BoldItalicFont = *-SemiboldIt,
    RawFeature     = {+zero,+ss06}]{SourceCodePro}
\newfontfamily\mysourcecodeprolight{SourceCodePro-Regular.otf}[
    Scale = 0.80,FakeStretch =0.75,
    FontFace = {m}{it}{Font =SourceCodePro-RegularIt.otf,Color=FF0000}]
\RenewDocumentCommand{\MacroLongFont}{}
  {
    \mysourcecodeprolight\small
  }
% The character of visible space is now taken from Latin Modern Mono
% to prevent fonts in T1. The original definition for xetex/luatex is
% \def\verbvisiblespace{\usefont{OT1}{cmtt}{m}{n}\asciispace}
\def\verbvisiblespace{{\fontfamily{lmtt}\selectfont\char"2423}}

\usepackage{unicode-math}
% Unicode-math changes the position of the glyphs, so the position 26A is now
% a letter `j'. The doc package ensures that mathcode and the | actually prints
% a j inside docstrip guards. This changes the mathcode of | to 07C:
\makeatletter
\def\mod@math@codes{\mathcode`\|="207C \mathcode`\&="2026 %
                    \mathcode`\-="702D \mathcode`\+="702B
                    \mathcode`\:="703A \mathcode`\=="703D }
\makeatother
\setmathfont[Scale = 0.95]{latinmodern-math.otf}
\usepackage{fontawesome5}
\newfontfamily\fetamontotf{ffmw10.otf}[
   Scale             = 0.95,%
   RawFeature        = {+latn,+rand,+kern,+size},%
   FontFace          = {b}{n}{ffmw10.otf},% fix replace font
   ]
\usepackage[svgnames]{xcolor}
\usepackage[sf,bf,compact,medium,pagestyles]{titlesec}
\usepackage{lastpage,microtype,attachfile2}
\usepackage{adjustbox,multicol,listings,accsupp,titletoc}
\usepackage{imakeidx}
\usepackage{scontents} % main
\usepackage[contents]{colordoc}
% \show\docCodelineNo
\def\textnew#1{\color{color-def}#1\/}
\def\AltMacroFont{\fontencoding\encodingdefault
                  \fontfamily\ttdefault
                  \fontseries\mddefault
                  \fontshape\updefault
                  \small
                  }%

% Patching colordoc.sty to work with l3doc.cls
\ExplSyntaxOn
\cs_new_eq:cN { liii@xmacro@code } \__codedoc_xmacro_code:n
\ExplSyntaxOff
\begingroup
\makeatletter
\catcode`\[\@ne\catcode`\]\tw@
\@makebracesactive
\gdef{[\@openingbrace[\char'173]]
\gdef}[\@closingbrace[\char'175]]
\catcode`\|\z@\catcode`\%12
\catcode`\ \active\catcode`\\\active
|gdef|xmacro@code#1%    \end{macrocode}[|liii@xmacro@code[#1]|end[macrocode]]
|catcode`| 12|gdef|sxmacro@code#1%    \end{macrocode*}[|liii@xmacro@code[#1]|end[macrocode*]]
|endgroup

% Patching colordoc.sty to work with doc v3 (TeX Live 2022 & lualatex-dev)
\usepackage{etoolbox}
\makeatletter
\@ifpackagelater{doc}{2022-06-01}%
  {\patchcmd\macro@finish
    {\ifnot@excluded % \if
      \edef\@tempa{\noexpand\SpecialIndex{\bslash\macro@namepart}}%
      \@tempa \fi}%
    {\maybe@index@macro \macro@namepart}
    {}{\FAILED}%
   \let\default@color\current@color
  }%
  {}
\makeatother

\EnableCrossrefs
% \PageIndex % \CodelineIndex undoes what \PageIndex does
% \CodelineIndex tries to open another write stream for the index file. We don't
% want that, so we temporarily make \makeindex a no-op:
\let\ORGmakeindex\makeindex
\def\makeindex{}
\CodelineIndex
\let\makeindex\ORGmakeindex
\DoNotIndex{\ , \1,\^}
\expandafter\DoNotIndex\expandafter{\string\{}
\expandafter\DoNotIndex\expandafter{\string\}}
\expandafter\DoNotIndex\expandafter{\string\begin}
\newcommand{\HP}[1]{\emph{\hyperpage{#1}}\normalsize}
\ExplSyntaxOn
\cs_new_protected:Npn \StartImplementation
  { \bool_set_true:N \l__codedoc_in_implementation_bool }
\bool_set_false:N \l__codedoc_in_implementation_bool
\cs_new_protected:Npn \MYSortIndex #1 #2
  {
    \bool_if:NTF \l__codedoc_in_implementation_bool
      { \index{#1\actualchar#2|HP} }
      { \index[userdoc]{#1\actualchar#2|HP} }
  }
\ExplSyntaxOff

\indexsetup{level=\section,firstpagestyle=myheader,othercode=\pagestyle{myheader}}
\makeindex[name=userdoc,options={-q -s gind.ist},columnsep=15pt,title={Index of Documentation}]
\makeindex[options={-q -s gind.ist},columnsep=15pt,title={Index of Implementation}]
% Now, after imakeidx opens the write stream for the index file, we copy
% the reference to \@indexfile:
\makeatletter
\expandafter\let\expandafter\@indexfile\csname \jobname @idxfile\endcsname
\makeatother

\setlength{\parindent}{0pt}

% Colors
\definecolor{optcolor}{rgb}{0.281,0.275,0.485}
\definecolor{pkgcolor}{rgb}{0,0.5,0.5}
\definecolor{araracolor}{rgb}{0, 0.72, 0.28}
\definecolor{linkcolor}{rgb}{0.04,0.38,0.04}
\definecolor{rulecolor}{rgb}{0.96,0.96,0.96}

% Only for arara... I like arara :)
\usepackage[scale=0.85]{comfortaa}
\newcommand{\araratext}[1]{{\small\normalfont\comfortaa\color{araracolor}\bfseries#1}}
\newcommand*\arara{\araratext{ar\kern-.03emar\kern-.03ema}}

% User level commands
\ExplSyntaxOn
% Logo with fetamont font
\NewDocumentCommand \pkglogo {}
  {
    \group_begin:
    \fetamontotf{\textcolor{pkgcolor}{s}\textcolor{OrangeRed}{content}\textcolor{pkgcolor}{s}}
    \group_end:
  }

% Custom \meta[...]{...}, \marg[...]{...} and \oarg[...]{...} with color
\NewDocumentCommand \mymeta {O{}m}
  {
    \userdoc_meta_generic:Nnn \userdoc_meta:n { #1 } { #2 }
  }
\NewDocumentCommand \mymarg {O{}m}
  {
    \userdoc_meta_generic:Nnn \userdoc_marg:n { #1 } { #2 }
  }
\NewDocumentCommand \myoarg {O{}m}
  {
    \userdoc_meta_generic:Nnn \userdoc_oarg:n { #1 } { #2 }
  }
% Variables and keys
\tl_new:N \l_userdoc_meta_font_tl

\keys_define:nn { userdoc / mymeta }
  {
    type .choice:,
    type / tt .code:n = \tl_set:Nn \l_userdoc_meta_font_tl { \ttfamily },
    type / rm .code:n = \tl_set:Nn \l_userdoc_meta_font_tl { \rmfamily },
    type .initial:n   = rm,
    cf .tl_set:N      = \l_userdoc_meta_color_tl,
    cf .initial:n     = optcolor,
    ac .tl_set:N      = \l_userdoc_meta_anglecolor_tl,
    ac .initial:n     = lightgray,
    sbc .tl_set:N     = \l_userdoc_meta_brackcolor_tl,
    sbc .initial:n    = gray,
    cbc .tl_set:N     = \l_userdoc_meta_bracecolor_tl,
    cbc .initial:n    = gray,
  }
% Internal commands
\cs_new_protected:Npn \userdoc_meta_generic:Nnn #1 #2 #3
  {
    \group_begin:
    \keys_set:nn { userdoc / mymeta } { #2 }
    \color{ \l_userdoc_meta_color_tl }
    \l_userdoc_meta_font_tl
    #1 { #3 } % #1 is \userdoc_meta:n, \userdoc_marg:n or \userdoc_oarg:n
    \group_end:
  }
\cs_new_protected:Npn \userdoc_meta:n #1
  {
    \userdoc_meta_angle:n { \textlangle }
    \userdoc_meta_meta:n { #1 }
    \userdoc_meta_angle:n { \textrangle }
  }
\cs_new_protected:Npn \userdoc_marg:n #1
  {
    \userdoc_meta_brace:n { \textbraceleft }
    \userdoc_meta:n { #1 }
    \userdoc_meta_brace:n { \textbraceright }
  }
\cs_new_protected:Npn \userdoc_oarg:n #1
  {
    \userdoc_meta_brack:n { [ }
    \userdoc_meta:n { #1 }
    \userdoc_meta_brack:n { ] }
  }
\cs_new_protected:Npn \userdoc_meta_meta:n #1
  {
    \textnormal{\textit{#1}}
  }
\cs_new_protected:Npn \userdoc_meta_angle:n #1
  {
    \group_begin:
    \fontfamily{lmr}\selectfont
    \textcolor{\l_userdoc_meta_anglecolor_tl}{#1}
    \group_end:
  }
\cs_new_protected:Npn \userdoc_meta_brace:n #1
  {
    \group_begin:
    \color{\l_userdoc_meta_bracecolor_tl}
    #1
    \group_end:
  }
\cs_new_protected:Npn \userdoc_meta_brack:n #1
  {
    \textcolor{\l_userdoc_meta_brackcolor_tl}{#1}
  }
% \envexamp{m}
\newsavebox{\boxexaenv}
\NewDocumentCommand \envexamp {m}
  {
   \begin{lrbox}{\boxexaenv}%
    \begin{minipage}[t]{\marginparwidth}%
     \raggedright\ttfamily\small
     \textcolor{gray}{\textbackslash begin\{\textcolor{pkgcolor}{{#1}}\}\myoarg
     [type=tt,sbc=gray,ac=lightgray,cf=optcolor]{keyval ~ list}}\par%
     \hspace{0.5cm}\mymeta[type=tt,ac=lightgray,cf=gray]{env ~ contents}\par%
     \textcolor{gray}{\textbackslash end\{{\textcolor{pkgcolor}{{#1}}}\}}\par
    \end{minipage}%
   \end{lrbox}%
   \usebox{\boxexaenv}
  }
\NewDocumentCommand \envexaplain {O{} m O{end}}
  {
   \begin{lrbox}{\boxexaenv}%
    \begin{minipage}[t]{\marginparwidth}%
     \raggedright\ttfamily\small
     \textcolor{pkgcolor}{{\textbackslash#1#2}}\myoarg
     [type=tt,sbc=gray,ac=lightgray,cf=optcolor]{keyval ~ list}\par%
     \hspace{0.5cm}\mymeta[type=tt,ac=lightgray,cf=gray]{env ~ contents}\par%
     \textcolor{pkgcolor}{{\textbackslash #3#2}}\par
    \end{minipage}%
   \end{lrbox}%
   \usebox{\boxexaenv}
  }

% \cmdexamp{s m o m o}
\DeclareDocumentCommand \cmdexamp {o m o m o}
  {
    \group_begin:
    \small\ttfamily
    \textcolor{pkgcolor}{\textbackslash#2}
    \IfValueT{#1}{ \textcolor{MediumOrchid}{#1} }
    \IfValueT{#3}{ \myoarg[type=tt,sbc=gray,ac=lightgray,cf=optcolor]{#3} }
    \IfValueTF{#5}
      {
        \mymeta[ac=gray,type=tt,cf=MediumOrchid]{#5}%
        \mymeta[type=tt,ac=gray,cf=optcolor]{#4}
        \mymeta[ac=gray,type=tt,cf=MediumOrchid]{#5}%
      }
      { \mymarg[type=tt,cbc=gray,ac=lightgray,cf=optcolor]{#4} }
    \par
    \group_end:
  }

% \keyexamp*{m m m}
\NewDocumentCommand \keyexamp{s m m m }
  {
    \adjustbox{outer=-\marginparsep}{\textcolor{optcolor}{\small\ttfamily{#2}}}
    \IfBooleanTF{#1}
      {
        \hphantom{\textcolor{white}{\,\bfseries\texttt{=}}\,{}}
        \mymeta[type=tt,cbc=gray,ac=lightgray,cf=lightgray]{\small{#3}}
      }
      {
        \textcolor{gray}{\,\bfseries\texttt{=}}\,{}
        \mymarg[type=tt,cbc=gray,ac=lightgray,cf=optcolor]{\small{#3}}
      }
    \hfill\textcolor{gray}{\small\textsf{default}:~\emph{#4}}\par%
    \MYSortIndex{Keys}{Keys ~ provide ~ by ~ \pkglogo :>\texttt{#2}}
  }

% \mykey{m}
\NewDocumentCommand \mykey { m }
  {
    \textcolor{optcolor}{\texttt{#1}}
    \MYSortIndex{Keys}{Keys ~ provide ~ by ~\pkglogo :>\texttt{#1}}%
  }

% \mypkg{sm}
\NewDocumentCommand \mypkg {sm}
  {
    \group_begin:
    \IfBooleanTF{#1}
      {
        \pkglogo
        \MYSortIndex{packages}{Packages:>\texttt{#2}}
      }
      {
        \textcolor{darkgray}{\textsf{#2}}
        \MYSortIndex{packages}{Packages:>\texttt{#2}}%
      }
     \group_end:
  }

% \myenv{sm}
\NewDocumentCommand \myenv {sm}
  {
    \IfBooleanTF{#1}
      {
        \textcolor{pkgcolor}{\ttfamily{#2}}%
        \MYSortIndex{environment}{Environments ~ provide ~ by ~\pkglogo :>\texttt{#2}}
      }
      {
        \textcolor{darkgray}{\ttfamily{#2}}%
        \MYSortIndex{environment}{Environments:>\texttt{#2}}
      }
  }

% \ics{sm}
\NewDocumentCommand \ics {sm}
  {
    \IfBooleanTF{#1}
      {
        \tl_set:Nn \l_tmpa_tl { #2 }
        \regex_replace_once:nnN { (\*) } { \c{textcolor}\cB\{MediumOrchid\cE\}\cB\{\1\cE\} } \l_tmpa_tl
        \textcolor{pkgcolor}{\ttfamily\textbackslash{\tl_use:N \l_tmpa_tl}}
        \MYSortIndex{Commands}{ Commands ~ provide  ~ by ~ \pkglogo :>\texttt{\textbackslash#2}}
      }
      {
        \textcolor{darkgray}{\ttfamily\textbackslash{#2}}
        \MYSortIndex{#2}{\texttt{\textbackslash#2}}
      }
  }
% \bomba
\NewDocumentCommand \bomba { }
  {
    \par\hspace*{-0.5\marginparsep}%
    \makebox[0pt][r]
      {
        \makebox[\marginparsep][r]
          {
            \footnotesize\textcolor{pkgcolor}{\faIcon{bomb}}
          }
      }
    \hskip 0.5\marginparsep
  }
\NewDocumentEnvironment{important}{ s }
  {
    \sffamily
    \bomba%
    \IfBooleanTF{#1}
      { \small }
      { \color{OrangeRed}\Large }
  }{}
\ExplSyntaxOff

% email https://tex.stackexchange.com/a/663
\catcode`\_=11\relax%
\newcommand\email[1]{\_email #1\q_nil}%
\def\_email#1@#2\q_nil{%
  \href{mailto:#1@#2}{{\emailfont #1\emailampersat #2}}%
}%
\newcommand\emailfont{\sffamily}%
\newcommand\emailampersat{{\color{optcolor}\footnotesize @}}%
\catcode`\_=8\relax%

% Custom vertamimsc
\makeatletter
\let\verbatimsc\@undefined
\let\endverbatimsc\@undefined
\makeatother
\usepackage{listings}
\lstnewenvironment{verbatimsc}
  {
    \lstset{
       language   =,%
       basicstyle = \ttfamily\small,
       gobble     = 2,%
           }
   }{}
\makeatother

% Don't copy numbers in code example
\newcommand*{\noaccsupp}[1]{\BeginAccSupp{ActualText={}}#1\EndAccSupp{}}

% Create a language for documentation
\lstdefinelanguage{scontents-doc}{
    texcsstyle    = *,%
    escapechar    = Ç,%
    extendedchars = true,%
    keepspaces    = true,%
    stringstyle   = {\color{red}},%
    basicstyle    = \ttfamily\small,%
    alsoletter    = {-,@,.},%
% comments
    morecomment   = [l]{\%},%
    commentstyle  = \ttfamily\small\color{lightgray},%
% Important words 1
    keywordstyle  = [1]{\color{gray}},%
    keywords      = [1]{begin,end,Verbatim,env,myseq,false},%
% Other words 2
    keywordstyle  = [2]{\color{blue!75}},%
    keywords      = [2]{usepackage,section,documentclass,input,usemodule,
                    subsection,ExplSyntaxOff,ExplSyntaxOn,RequirePackage,
                    ProvidesExplPackage,obeyedline},%
% Other words 3
    keywordstyle  = [3]{\color{optcolor!85}},%
    keywords      = [3]{document,article,setlength,pagestyle,definecolor},%
% Reserved words 4 (scontents pkg)
    keywordstyle  = [4]{\color{pkgcolor}},%
    keywords      = [4]{scontents,Scontents,getstored,typestored,verbatimsc, mergesc, %
                    endverbatimsc,countsc,meaningsc,foreachsc,macros,newenvsc,
                    endscontents,startscontents,stopscontents,scontents.tex},%
% Reserved words 5 (in red)
    keywordstyle  = [5]{\color{red}},%
    keywords      = [5]{makeatletter,makeatother,let,verb,@undefined,makebox,def,%
                    par,item,parindent,lstinline,@ifpackagelater,ver@xparse.sty,%
                    ver@scontents.sty,dospecials,do@noligs,char,outer,jobname,inner},%
% Reserved words 6 (optcolor)
    keywordstyle  = [6]{\color{optcolor}},%
    keywords      = [6]{store-cmd,store-env,print-env,print-cmd,write-env,%
                    start,stopt,wrapper,sep,step,before,after,write-out,write-cmd,%
                    overwrite,width-tab,force-eol,inner,outer,I,J,M,L,Ascheol},%
% Reserved words 7 (OrangeRed)
    keywordstyle  = [7]{\color{OrangeRed}},%
    keywords      = [7]{DefineVerbatimEnvironment,lstnewenvironment,newminted,%
                    newtheorem,newenvironment,VerbatimEnvironment,myenvstore},%
% literate
    literate      =*{\{}{{{\color{gray}{\char`\{}}}}{1}
                    {\}}{{{\color{gray}{\char`\}}}}}{1}
                    {\|}{{{\color{gray}{\char`\|}}}}{1}
                    {\$}{{{\color{gray}{\char`\$}}}}{1}
                    {\#}{{{\color{optcolor}{\char`\#}}}}{1}
                    {[}{{\textcolor{gray}{[}}}{1}
                    {]}{{\textcolor{gray}{]}}}{1}
                    {,}{{\textcolor{gray}{,}}}{1}
                    {;}{{\textcolor{gray}{;}}}{1}
                    {`}{{\textcolor{gray}{\textasciigrave}}}{1}
                    {\&}{{\textcolor{gray}{\&}}}{1}
                    {>}{{\textcolor{OrangeRed}{\guillemotright}}}{1}
                    {<}{{\textcolor{OrangeRed}{\guillemotleft}}}{1}
                    {*}{{\textcolor{MediumOrchid}{*}}}{1}
                    {\^}{{\textcolor{optcolor}{\textasciicircum}}}{1}
                    {0}{{\textcolor{MediumOrchid}{0}}}{1}
                    {1}{{\textcolor{MediumOrchid}{1}}}{1}
                    {2}{{\textcolor{MediumOrchid}{2}}}{1}
                    {3}{{\textcolor{MediumOrchid}{3}}}{1}
                    {4}{{\textcolor{MediumOrchid}{4}}}{1}
                    {5}{{\textcolor{MediumOrchid}{5}}}{1}
                    {6}{{\textcolor{MediumOrchid}{6}}}{1}
                    {7}{{\textcolor{MediumOrchid}{7}}}{1}
                    {8}{{\textcolor{MediumOrchid}{8}}}{1}
                    {9}{{\textcolor{MediumOrchid}{9}}}{1}
                    {.0}{{\textcolor{MediumOrchid}{.0}}}{2}% Following is to ensure that only periods
                    {.1}{{\textcolor{MediumOrchid}{.1}}}{2}% followed by a digit are changed.
                    {.2}{{\textcolor{MediumOrchid}{.2}}}{2}
                    {.3}{{\textcolor{MediumOrchid}{.3}}}{2}
                    {.4}{{\textcolor{MediumOrchid}{.4}}}{2}
                    {.5}{{\textcolor{MediumOrchid}{.5}}}{2}
                    {.6}{{\textcolor{MediumOrchid}{.6}}}{2}
                    {.7}{{\textcolor{MediumOrchid}{.7}}}{2}
                    {.8}{{\textcolor{MediumOrchid}{.8}}}{2}
                    {.9}{{\textcolor{MediumOrchid}{.9}}}{2}
                    {=}{{\textcolor{gray}{=}}}{1},%
}[keywords,tex,comments,strings]% end language

% \begin{examplecode}[key=val]...\end{examplecode}
\lstnewenvironment{examplecode}[1][]{%
\lstset{
    language    = scontents-doc,%
    stringstyle = {\color{red}},%
    basicstyle  = \ttfamily\small,%
    columns     = fullflexible,%
    fontadjust  = true,%
    aboveskip   = 8pt,%
    belowskip   = 5pt,%
    rulecolor   = \color{gray!50},%
    framesep    = \fboxsep,%
    framerule   = \fboxrule,%
    xleftmargin = \dimexpr\fboxsep+\fboxrule\relax,%
    xrightmargin= \dimexpr\fboxsep+\fboxrule\relax,%
           #1,%
    }% close lstset
}{}% close examplecode

% \lstinline[style=inline]|...|
\lstdefinestyle{inline}
  {
    language    = scontents-doc,%
    upquote     = true,%
    numbersep   = 1em,%
    numberstyle = \tiny\color{lightgray}\noaccsupp,%
    literate    =*{\%}{{\textcolor{gray}{\%}}}{1}
  }

% Set default \lstinline style
\lstset{style=inline}

% Set short §...§
\lstMakeShortInline[language=scontents-doc,basicstyle=\ttfamily]§

% Get file info
\GetFileInfo{scontents.sty}

% Config hyperref
\hypersetup{
   allcolors          = linkcolor,
   colorlinks         = true,
   linktoc            = all,
   pdftitle           = {.:: The scontents package --- \fileinfo{} ::.},
   pdfauthor          = {Pablo González L},
   pdfsubject         = {Documentation for \fileversion{} [\filedate] },
   pdfkeywords        = {sequences, contents, external files, expl3, l3seq, store contents},
   bookmarksopenlevel = 1
  }

% Configuration titleps and titlesec
\settitlemarks{section}
\renewpagestyle{plain}[\color{optcolor}\small\sffamily]{%
\setfoot{\rlap{\hskip\dimexpr-\oddsidemargin-1in\relax%
         \parbox{1.93\paperwidth}{\hfil\thepage\,/\,\pageref{LastPage}}}}%
        {}%
        {}%
}
\newpagestyle{myheader}[\color{optcolor}\small\sffamily]{%
\renewcommand\makeheadrule{%
\rlap{\hskip\dimexpr-\oddsidemargin-1in\relax
      \color{rulecolor}\rule[0.3cm]{\paperwidth}{0.7cm}}\hss
}
\setfoot{\rlap{\hskip\dimexpr-\oddsidemargin-1in\relax%
         \parbox{1.93\paperwidth}{\hfil\thepage\,/\,\pageref{LastPage}}}}%
        {\parbox{\textwidth}{\raggedright \textcolor{gray}{\raisebox{-1pt}{\textcopyright}{}2019--2025 by Pablo González}}}%
        {}%
\sethead{\llap{\raisebox{0.55cm}{\parbox{\dimexpr\oddsidemargin+1in\relax}{\makebox[0pt][l]{\hspace{15pt}\pkglogo\space\fileversion}}}}}
        {\raisebox{0.55cm}{\parbox{\textwidth}{\hspace*{-\oddsidemargin}\centering\small\S.\thesection\space\sectiontitle}}}%
        {}%
}

\titlecontents{section}[0mm]{}%
    {\bfseries\contentspush{\makebox[5mm][l]{\thecontentslabel\hfill}}}%
    {\hspace*{-5mm}}% numberless
    {\hspace{0.25em}\titlerule*[6pt]{.}\contentspage}%
\titlecontents{subsection}[5mm]{}%
    {\contentspush{\makebox[6mm][l]{\thecontentslabel\hfill}}}
    {\hspace*{-11mm}}% numberless
    {\hspace{0.25em}\titlerule*[6pt]{.}\contentspage}%

% Table of contents
\makeatletter
\renewcommand\tableofcontents{%
\begingroup%
\section*{\contentsname\quad{\color{gray}\leaders\hrule height 5pt depth -4.4pt\hfill}%
  \@mkboth{%
    \MakeUppercase\contentsname}{\MakeUppercase\contentsname}}%
\vspace*{-14pt}
\setlength{\columnsep}{10pt}%
 \begin{multicols}{2}%
    \@starttoc{toc}%
\end{multicols}%
\vspace*{-3pt}{\color{gray}\hrule height 0.6pt}%
\vspace*{5pt}
\endgroup
}
\makeatother
% thanks
\makeatletter
\renewcommand*{\@fnsymbol}[1]{\ensuremath{\ifcase#1\or \textcolor{pkgcolor}{*}
\or \textcolor{pkgcolor}{\dagger} \or \ddagger\or \mathsection
\or \mathparagraph\or \|\or **\or \dagger\dagger
\or \ddagger\ddagger \else\@ctrerr\fi}}
\makeatother
\begin{document}
  \DocInput{\jobname.dtx}
\end{document}
%</driver>
% \fi
%
% \title{
%    \scalebox{1.57}{\pkglogo}\\[2pt]
%    \Large
%    \textsf{Stores \hologo{LaTeX}}{ \fetamontotf{contents} }\\[3pt]
%    \fileversion \quad \filedate\thanks{
%    This file describes a documentation for \fileversion, last revised
%    \filedate.}\\[25pt]
%    \author{
%    \large
%    \raisebox{-1pt}{\textcopyright}{}2019--2025 by Pablo González\thanks{
%    E-mail: \textcolor{OrangeRed}{\textsf{\guillemotleft}}\email{pablgonz@educarchile.cl}\textcolor{OrangeRed}{\textsf{\guillemotright}}.
%       }%
%    }
% \small
% \textsc{ctan}: \url{https://www.ctan.org/pkg/scontents}\\
% \textcolor{gray}{\scriptsize\faIcon[regular]{github}}\,\,\,\url{https://github.com/pablgonz/scontents}
% \vspace*{-2cm}
% }
% \date{}
% \maketitle
%
% \begin{scontents}[store-env=abstract, print-env=true]
% \begin{abstract}
% This package allows to store \hologo{LaTeX} code, including \enquote{\emph{verbatim}},
% in \mymeta{sequences} using the \mypkg{l3seq} module of \mypkg{expl3}. The \mymeta{stored
% content} can be used as many times as desired in the document, additionally you can write
% to \mymeta{external files} or show it in \mymeta{verbatim style}.
% \end{abstract}
% \end{scontents}
%
% \tableofcontents
%
% \begin{important}
% The next update removes compatibility with versions prior to 2024.
% \end{important}
%
% \setlength{\parskip}{3pt}
%
% \section{Description of the package}
%
% \begin{scontents}[store-env=description, print-env=true]
% The \mypkg*{scontents} package allows to \mymeta{store contents} in \mymeta{sequences}
% or \mymeta{external files}. In some ways it is similar to the \mypkg{filecontentsdef}
% package, with the difference in which the \mymeta{content} is stored. The idea behind
% this package is to get an approach to \hologo{ConTeXt} \enquote{\emph{buffers}} by
% making use \mymeta{sequences}.
% \end{scontents}
%
% \section{Motivation and Acknowledgments}
%
% In \hologo{LaTeX} there is no direct way to record content for later use, although
% you can do this using §\macros§, recording \mymeta[type=rm]{verbatim content}  is a
% problem, usually you can avoid this by creating external files or boxes.
%
% The general idea of this package is to try to imitate this implementation
% \enquote{\emph{buffers}} that has \hologo{ConTeXt} which allows you to save content in
% memory, including \emph{verbatim}, to be used later. The package
% \mypkg{filecontentsdef} solves this problem and since \mypkg{expl3} has an
% excellent way to manage data, ideas were combined giving rise to this package.
%
% This package would not be possible without the great work of \textsc{Jean
% Fran\c{c}ois Burnol} who was kind enough to take my requirements into account
% and add the \myenv{filecontentsdefmacro} environment. Also a special thanks to
% Phelype Oleinik who has collaborated and adapted a large part of the code and
% all \hologo{LaTeX} team for their great work and to the different members
% of the \href{https://tex.stackexchange.com}{TeX-SX} community who have provided
% great answers and ideas. Here a note of the main ones:
%
% \begin{enumerate}[left=0pt, labelsep=5pt, noitemsep, nosep, label=\small\arabic*.]
% \item \href{https://tex.stackexchange.com/q/45946/7832}{Stack datastructure using LaTeX}
%
% \item \href{https://tex.stackexchange.com/q/5338/7832}{LaTeX equivalent of ConTeXt buffers}
%
% \item \href{https://tex.stackexchange.com/q/215563/7832}{Storing an array of strings in a command}
%
% \item \href{https://tex.stackexchange.com/q/184503/7832}{Collecting contents of environment and store them for later retrieval}
%
% \item \href{https://tex.stackexchange.com/q/373647/7832}{Collect contents of an environment (that contains verbatim content)}
% \end{enumerate}
%
% \thispagestyle{plain}
% \newpage
% \pagestyle{myheader}
%
% \section{License and Requirements}
% \label{sec:licence}
%
% Permission is granted to copy, distribute and/or modify this software under
% the terms of the LaTeX Project Public License (lppl), version 1.3 or later
% (\url{https://www.latex-project.org/lppl.txt}). The software has the status
% \enquote{maintained}.
%
% The The \mypkg*{scontents} package is written (mostly) using
% \mypkg{expl3}, it requires an updated version of \LaTeX{} to work
% (minimum version 2022-06-01). This package can be used with "plain",
% "context", "xelatex", "lualatex", "pdflatex" and the classical workflow
% §latex>dvips>ps2pdf§.
%
% \section{The \texttt{scontents} package}
% \label{sec:pkgscontents}
%
% \subsection{Installation}
%
% The package \mypkg*{scontents} is present in \hologo{TeX} Live and \hologo{MiKTeX},
% use the package manager to install. For manual installation, download
% \href{https://mirrors.ctan.org/macros/latex/contrib/scontents.zip}{scontents.zip}
% and unzip it, run \lstinline[language=scontents-doc,basicstyle=\ttfamily]+luatex scontents.ins+
% and move all files to appropriate locations, then run "mktexlsr". To produce the documentation
% with source code run \lstinline[language=scontents-doc,basicstyle=\ttfamily]+luatex scontents.ins+ and
% "lualatex scontents.dtx" three times.
%
% \iffalse
%<*example>
% \fi
\begin{examplecode}[frame=single,language={}]
  scontents.tex       >  TDS:tex/generic/scontents/
  scontents-code.tex  >  TDS:tex/generic/scontents/
  scontents.sty       >  TDS:tex/latex/scontents/
  t-scontents.mkiv    >  TDS:tex/context/third/scontents/
  scontents.pdf       >  TDS:doc/latex/scontents/
  README.md           >  TDS:doc/latex/scontents/
  scontents.dtx       >  TDS:source/latex/scontents/
  scontents.ins       >  TDS:source/latex/scontents/
\end{examplecode}
% \iffalse
%</example>
% \fi
%
% \subsection{Loading and options}
%
% The package is loaded in the usual way:
%
% \subsubsection*{For \hologo{LaTeX} users}
% \iffalse
%<*example>
% \fi
\begin{examplecode}[frame=single]
\usepackage{scontents}
\end{examplecode}
% \iffalse
%</example>
% \fi
%
% or
%
% \iffalse
%<*example>
% \fi
\begin{examplecode}[frame=single]
\usepackageÇ\small\myoarg[type=tt]{key \textnormal{\textcolor{gray}{=}} val}Ç{scontents}
\end{examplecode}
% \iffalse
%</example>
% \fi
%
% The package options are not available for \hologo{plainTeX} and
% \hologo{ConTeXt}, see \ref{sec:confopt}.
%
% \subsubsection*{For \hologo{plainTeX} users}
%
% \iffalse
%<*example>
% \fi
\begin{examplecode}[frame=single]
\input scontents.tex
\end{examplecode}
% \iffalse
%</example>
% \fi
%
% \subsubsection*{For \hologo{ConTeXt} users}
%
% \iffalse
%<*example>
% \fi
\begin{examplecode}[frame=single]
\usemodule{scontents}
\end{examplecode}
% \iffalse
%</example>
% \fi
%
% \begin{important}*
% \hologo{ConTeXt} users should use |-luatex|, the implementation does not support LuaMetaTeX.
% \end{important}
%
% \subsection{The TAB character}
% \label{sec:tabcharacter}
%
% Some users use horizontal "TAB"s \enquote{\LKeyTab} from keyboard to indented the source
% code of the document and depending on the text editor used, some will use real "TAB"s
% (\enquote{\emph{hard tabs}}), others with \enquote{\emph{soft tabs}}(\verb*|  | or \verb*|    |)
% or both.
%
% At first glance it may seem the same, but the way in which "TAB"s (\enquote{\emph{hard tabs}})
% are processed according to the context in which they are found within
% a file, both in \mymeta{reading}\footnote{Check the answer given by Ulrich Diez
% in \href{https://tex.stackexchange.com/a/508103/7832}{Keyboard TAB
% character in argument v (xparse)}.} and \mymeta{writing}\footnote{Check the answer
% given by Enrico Gregorio in \href{https://tex.stackexchange.com/a/58740/7832}{How to output a
% tabulation into a file}.} are different and may have adverse consequences.
%
% In a standard \hologo{LaTeX} document, the character "TAB" \enquote{\LKeyTab}
% are treated as explicit spaces (in most contexts) and is the behavior when
% \mymeta{stored contents}, but when \mymeta{writing files} these are preserved.
%
% With a \hologo{TeX} Live distribution, the "TAB" character is \enquote{\emph{printable}}
% for "latex", "pdflatex" and "lualatex", but if you use "xelatex" you
% must add the \texttt{\textcolor{optcolor}{-8bit}} option on the command
% line, otherwise you will get \hologo{TeX}-"TAB" (§^^I§) in the \mymeta{output file}.
%
% As a general recommendation \enquote{Do not use "TAB" character unless strictly
% necessary}, for example within a \emph{verbatim} environment that supports
% this character such as \myenv{Verbatim} of the package \mypkg{fancyvrb} or \myenv{lstlisting}
% of the package \mypkg{listings} or when you want to generate a §MakeFile§ file.
%
% \subsection{Configuration of the options}
% \label{sec:confopt}
%
% Most of the options can be passed directly to the package or using the
% command \ics*{setupsc}. All boolean keys can be passed using the
% equal sign \enquote{\textnormal{\textcolor{gray}{=}}} or just the
% name of the key, all unknown keys will return an error. In this section
% are described some of the options, a summary of all options is shown in
% section \ref{sec:optover}.
%
% \vspace*{-10pt}
%
% \begin{function}{\setupsc}
%   \begin{syntax}
%       \cmdexamp{setupsc}{keyval list}
%   \end{syntax}
%
% The command \ics*{setupsc} sets the \mymeta{keys} in a global way,
% it can be used both in the preamble and in the body of the document
% as many times as desired. However, options set in the declaration of
% an environment (with \ics*{newenvsc}) have precedence over options
% set with \ics*{setupsc}.
%
% Options in the optional arguments of environments and
% commands have the highest precedence, overriding both
% options in \ics*{newenvsc}, and in \ics*{setupsc}.
% \end{function}
%
% \keyexamp{verb-font}{font family}{\textnormal{\ttfamily\textbackslash{}ttfamily}}
% Sets the \mymeta{font family} used to display the \mymeta{stored content} for the
% \ics*{typestored} and \ics*{meaningsc} commands. This key is only available
% as a package option or using \ics*{setupsc}.
%
% \medskip
%
% \keyexamp{store-all}{seq name}{not used}
% It is a \mymeta{meta-key} that sets the \mykey{store-env} key of the \myenv*{scontents}
% environment and the \mykey{store-cmd} key of the \ics*{Scontents} command. This key is
% only available as a package option or using \ics*{setupsc}.
%
% \medskip
%
% \keyexamp{overwrite}{true \textnormal{\textcolor{lightgray}{\textbar}} false}{false}
% Sets whether the \mymeta{files} generated by \mykey{write-out} and \mykey{write-env}
% from the \myenv*{scontents} environment will be rewritten. This key is available as
% a package option, for \ics*{setupsc}, for \ics*{Scontents*} and for the environment
% \myenv*{scontents}.
%
% \medskip
%
% \keyexamp{print-all}{true \textnormal{\textcolor{lightgray}{\textbar}} false}{false}
% It is a \mymeta{meta-key} that sets the \mykey{print-env} key of the \myenv*{scontents}
% environment and the \mykey{print-cmd} key of the \ics*{Scontents} command. This key is
% only available as a package option or using \ics*{setupsc}.
%
% \keyexamp{force-eol}{true \textnormal{\textcolor{lightgray}{\textbar}} false}{false}
% Sets if the end of line for the \mymeta{stored content} is hidden or not.
% This key is necessary only if the last line is the closing of some
% environment defined by the \mypkg{fancyvrb} package as §\end{Verbatim}§
% or another environment that does not support a comments \enquote{\ttfamily\textcolor{gray}{\%}}
% after closing §\end{<env>}%§. This key is available for the \myenv*{scontents}
% environment and the \ics*{Scontents} command.
%
% \keyexamp{width-tab}{integer}{1}
% Sets the equivalence in \mymeta{spaces} for the character "TAB"
% used when displaying stored content in \emph{verbatim style}. The value
% must be a \mymeta{positive integer}. This key is available for the \ics*{typestored}
% and the \ics*{meaningsc} commands.
%
% \subsection{Options Overview}
% \label{sec:optover}
%
% \newcommand*{\xmark}{\textcolor{OrangeRed}{✘}}%
% \newcommand*{\cmark}{\textcolor{linkcolor}{✔}}%
%
% Summary of available options:
%
% \setlength{\tabcolsep}{0.25em}
% \begin{center}
% \small
% \begin{tabular}{cccccccc}
% \toprule
% \texttt{key} &package &\ics*{setupsc} &\myenv*{scontents} &\ics*{Scontents} &\ics*{Scontents*} &\ics*{typestored} &\ics*{meaningsc}\\
% \midrule
%  \mykey{store-env} & \cmark  & \cmark  &  \cmark  & \xmark  & \xmark  & \xmark  & \xmark \\
%  \mykey{store-cmd} & \cmark  & \cmark  &  \xmark  & \cmark  & \cmark  & \xmark  & \xmark \\
%  \mykey{print-env} & \cmark  & \cmark  &  \cmark  & \xmark  & \xmark  & \xmark  & \xmark \\
%  \mykey{print-cmd} & \cmark  & \cmark  &  \xmark  & \cmark  & \cmark  & \xmark  & \xmark \\
%  \mykey{print-all} & \cmark  & \cmark  &  \xmark  & \xmark  & \xmark  & \xmark  & \xmark \\
%  \mykey{store-all} & \cmark  & \cmark  &  \xmark  & \xmark  & \xmark  & \xmark  & \xmark \\
%  \mykey{write-env} & \xmark  & \xmark  &  \cmark  & \xmark  & \xmark  & \xmark  & \xmark \\
%  \mykey{write-cmd} & \xmark  & \xmark  &  \xmark  & \xmark  & \cmark  & \xmark  & \xmark \\
%  \mykey{write-out} & \xmark  & \xmark  &  \cmark  & \xmark  & \cmark  & \xmark  & \xmark \\
%  \mykey{overwrite} & \cmark  & \cmark  &  \cmark  & \xmark  & \cmark  & \xmark  & \xmark \\
%  \mykey{width-tab} & \cmark  & \cmark  &  \xmark  & \xmark  & \xmark  & \cmark  & \cmark \\
%  \mykey{force-eol} & \cmark  & \cmark  &  \cmark  & \cmark  & \cmark  & \xmark  & \xmark \\
%  \mykey{verb-font} & \cmark  & \cmark  &  \xmark  & \xmark  & \xmark  & \xmark  & \xmark \\
% \bottomrule
% \end{tabular}
% \end{center}
%
% \section{User interface}
% \label{sec:interface}
%
% The user interface consists in \myenv*{scontents} environment, \ics*{Scontents}
% and \ics*{Scontents*} commands to \mymeta{stored content} and \ics*{getstored}
% command to get the \mymeta{stored content} along with other utilities described
% in this documentation.
%
% \subsection{The environment \texttt{scontents}}
% \label{sec:envscontents}
%
% \vspace*{-10pt}
%
% \begin{function}[label=SCONTENTS]{scontents}
%   \begin{syntax}
%   \envexamp{scontents}
%   \end{syntax}
%
% The \myenv*{scontents} environment allows you to \mymeta{store} and \mymeta{write}
% content, including \emph{verbatim} material. After the package has been
% loaded, the environment can be used both in the preamble and in the body
% of the document.
%
% For the correct operation §\begin{scontents}§ and §\end{scontents}§ must
% be in different lines, all \mymeta{keys} must be passed separated by
% commas and \enquote{without separation} of the start of the environment.
% \end{function}
%
% Comments \enquote{\textcolor{gray}{\%}} or \enquote{any character} after
% §\begin{scontents}§ or \myoarg{keyval list} on the same line are not
% supported, the package will return an \enquote{error} message if this happens.
% In a similar way comments \enquote{\textcolor{gray}{\%}} or \enquote{any character} after
% §\end{scontents}§ on the same line the package will return a \enquote{warning} message.
%
% The environment can be \mymeta{nested} if it is properly balanced and
% does not appear \enquote{literally} in commented lines or in some \emph{verbatim}
% environment or command. As an example:
%
% \iffalse
%<*example>
% \fi
\begin{examplecode}[frame=single]
\begin{scontents}[store-env=outer]
This text is in the outer environment (before nested).
\begin{scontents}[store-env=inner]
This text is found in the inner environment (inside of nested).
\end{scontents}
This text is in the outer environment (after nested).
\end{scontents}
\end{examplecode}
% \iffalse
%</example>
% \fi
%
% Of course, content stored in the \mymeta{inner} sequence is only available
% after content stored in the \mymeta{outer} sequence one has been retrieved,
% either by using the key \mykey{print-env} or \ics*{getstored} command.
%
% It is advisable to store content within sequences with different names,
% so as not to get lost in the order in which content is stored.
%
% \subsection*{Notes for \hologo{plainTeX} and \hologo{ConTeXt} users}
%
% In \hologo{plainTeX} there is not environments as in \hologo{LaTeX}.
% Instead of using the environment \myenv*{scontents}, one should use a
% \emph{pseudo environment} delimited by \ics*{scontents} and \ics*{endscontents}.
%
% \vspace*{-10pt}
% \begin{function}{\scontents,\endscontents}
%   \begin{syntax}
%   \envexaplain{scontents}
%   \end{syntax}
% \end{function}
%
% \hologo{ConTeXt} users should use \ics*{startscontents} and \ics*{stopscontents}.
% \vspace*{-10pt}
% \begin{function}{\startscontents,\stopscontents}
%   \begin{syntax}
%   \envexaplain[start]{scontents}[stop]
%   \end{syntax}
% \end{function}
%
% \subsection*{Options for environment}
%
% The environment options can be configured globally using option
% in package or the \ics*{setupsc} command and locally using
% \myoarg{key \textnormal{\textcolor{gray}{=}} val} in the environment.
% The key \mykey{force-eol} is available for this environment.
%
% \keyexamp{store-env}{seq name}{contents}
% Sets the name of the \mymeta{sequence} in which the contents will be
% stored. If the sequence does not exist, it will be created globally.
%
% \medskip
%
% \keyexamp{print-env}{true \textnormal{\textcolor{lightgray}{\textbar}} false}{false}
% Sets if the \mymeta{stored content} is displayed or not at the time of
% running the environment. The content is extracted from the \mymeta{sequence}
% in which it is stored.
%
% \medskip
%
% \keyexamp{write-env}{file.ext}{not used}
% Sets the name of the \mymeta{external file} in which the \mymeta{contents} of
% the environment will be written.  The \mymeta{file.ext} will be created
% in the working directory, relative or absolute paths are not supported. If \mymeta{file.ext}
% does not exist, it will be created or overwritten if the \mykey{overwrite} key is used.
%
% The characters "TAB"s will be written in \mymeta{file.ext} and the \mymeta{contents}
% will be stored in the \mymeta{sequence} established at that time. \hologo{XeLaTeX} users
% using the "TAB" character must add \texttt{\textcolor{optcolor}{-8bit}}
% at the command line, otherwise you will get \hologo{TeX}-"TAB" (§^^I§)
% in \mymeta{file.ext}.
%
% \medskip
%
% \keyexamp{write-out}{file.ext}{not used}
% Sets the name of the \mymeta{external file} in which the \mymeta{contents} of
% the environment will be written. The \mymeta{file.ext} will be created
% in the working directory, relative or absolute paths are not supported. If \mymeta{file.ext}
% does not exist, it will be created or overwritten if the \mykey{overwrite} key is used.
%
% The characters "TAB"s will be written in \mymeta{file.ext}, the rest of the
% \mymeta{keys} will not be available and the \mymeta{contents} will NOT
% be stored in any \mymeta{sequence}. \hologo{XeLaTeX} users using the "TAB" character
% must add \texttt{\textcolor{optcolor}{-8bit}} at the command line,
% otherwise you will get \hologo{TeX}-"TAB" (§^^I§) in \mymeta{file.ext}.
%
%
% \subsection{The command \cs{newenvsc}}
% \label{sec:newenvsc}
%
% \newsavebox{\exaboxtmp}
% \begin{lrbox}{\exaboxtmp}
%    \small\ttfamily
%    \textcolor{pkgcolor}{\textbackslash newenvsc}\mymarg[type=tt]{env name}\myoarg[type=tt]{initial keys}
% \end{lrbox}
%
% \vspace*{-10pt}
%
% \begin{function}{\newenvsc}
%   \begin{syntax}
%    \usebox{\exaboxtmp}
%   \end{syntax}
% \end{function}
%
% The command \ics*{newenvsc} allows you to create \mymeta{new environments}
% based on the same characteristics of the \myenv*{scontents} environment.
% The values entered in \myoarg{initial keys} will be considered as the
% default values for this new environment and the valid \mymeta{keys} are
% \mykey{store-env} and \mykey{print-env}. For example:
%
% \iffalse
%<*example>
% \fi
\begin{examplecode}[frame=single]
\newenvsc{myenvstore}[store-env=myseq,print-env=false]
\end{examplecode}
% \iffalse
%</example>
% \fi
%
% created the §myenvstore§ environment that stored the content in the §myseq§
% sequence and will not display the content when it is executed.
%
% \subsection{The command \cs{Scontents}}
% \label{sec:Scontents}
%
% \vspace*{-10pt}
%
% \begin{function}{\Scontents}
%   \begin{syntax}
%     \cmdexamp{Scontents}[key \textnormal{\textcolor{gray}{=}} val]{argument}
%     \cmdexamp[*]{Scontents}[key \textnormal{\textcolor{gray}{=}} val]{argument}
%     \cmdexamp[*]{Scontents}[key \textnormal{\textcolor{gray}{=}} val]{argument}[del]
%   \end{syntax}
% \end{function}
%
% The \ics*{Scontents} command reads the \mymarg{argument} in standard
% mode. It is not possible to pass environments such as \emph{verbatim},
% but it is possible to use the implementation of \ics{Verb} provided by
% the \mypkg{fvextra} package for contents on one line and \ics{lstinline}
% from \mypkg{listings} package, but it is preferable to use the starred (§*§)
% version.
%
% The \ics*{Scontents*} command reads the \mymarg{argument} under \emph{verbatim}
% category code regimen. If its first delimiter is a brace, it will be
% assumed that the \mymarg{argument} is nested into braces. Otherwise it
% will be assumed that the ending of that \mymeta{argument} is delimited by that
% first delimiter \mymeta[cf=MediumOrchid]{del} like command \ics{verb}.
%
% Blank lines are preserved, escaped braces \enquote{\texttt{\textcolor{gray}{\textbackslash\hspace{-1pt}\{}}} and
% \enquote{\texttt{\textcolor{gray}{\textbackslash\}}}} must also be
% balanced if the argument is used with braces and "TAB"s characters typed
% from the keyboard are converted into spaces.
% The starred argument (§*§) and \myoarg{key \textnormal{\textcolor{gray}{=}} val}
% must not be separated by horizontal spaces between them and the command.
%
% Both versions can be used anywhere in the document and cannot be used
% as an \mymeta{argument} for other command.
%
% \subsection*{Options for command}
% \label{sec:optcmdsc}
%
% The command options can be configured globally using option in package
% or the \ics*{setupsc} command and locally using \myoarg{key \textnormal{\textcolor{gray}{=}} val}.
% The key \mykey{force-eol} is available for this command.
%
% \keyexamp{store-cmd}{seq name}{contents}
% Sets the name of the \mymeta{sequence} in which the contents will be stored.
% If the sequence does not exist, it will be created globally.
%
% \medskip
%
% \keyexamp{print-cmd}{true \textnormal{\textcolor{lightgray}{\textbar}} false}{false}
% Sets if the \mymeta{stored content} is displayed or not at the time of
% running the command. The content is extracted from the \mymeta{sequence}
% in which it is stored.
%
% \subsection*{Options only for the starred version}
% \label{sec:onlystarred}
%
% \keyexamp{write-cmd}{file.ext}{not used}
% Sets the name of the \mymeta{external file} in which the \mymeta{contents} of
% the \mymarg{argument} will be written. The \mymeta{file.ext} will be created
% in the working directory, relative or absolute paths are not supported. If \mymeta{file.ext}
% does not exist, it will be created or overwritten if the \mykey{overwrite} key is used.
%
% The characters "TAB"s will be written in \mymeta{file.ext} and the \mymeta{contents}
% will be stored in the \mymeta{sequence} established at that time. \hologo{XeLaTeX} users
% using the "TAB" character must add \texttt{\textcolor{optcolor}{-8bit}}
% at the command line, otherwise you will get \hologo{TeX}-"TAB" (§^^I§)
% in \mymeta{file.ext}.
%
% \medskip
%
% \keyexamp{write-out}{file.ext}{not used}
% Sets the name of the \mymeta{external file} in which the \mymeta{contents} of
% the \mymarg{argument} will be written. The \mymeta{file.ext} will be created
% in the working directory, relative or absolute paths are not supported. If \mymeta{file.ext}
% does not exist, it will be created or overwritten if the \mykey{overwrite} key is used.
%
% The characters "TAB"s will be written in \mymeta{file.ext}, the rest of the
% \mymeta{keys} will not be available and the \mymeta{contents} will NOT
% be stored in any \mymeta{sequence}. \hologo{XeLaTeX} users using the "TAB" character
% must add \texttt{\textcolor{optcolor}{-8bit}} at the command line,
% otherwise you will get \hologo{TeX}-"TAB" (§^^I§) in \mymeta{file.ext}.
%
% The key \mykey{overwrite} is available for this command.
%
% \subsection{The command \cs{getstored}}
% \label{sec:getstored}
%
% \vspace*{-10pt}
%
% \begin{function}{\getstored}
%   \begin{syntax}
%      \cmdexamp{getstored}[index]{seq name}
%   \end{syntax}
% \end{function}
%
% The command \ics*{getstored} gets the content stored in \mymarg{seq name}
% according to the \mymeta{index} in which it was stored. The command is
% robust and can be used as an \mymeta{argument} for another command. If
% the optional argument is not passed, the default value is the \enquote{last
% element} stored in \mymarg{seq name}.
%
% \subsection{The command \cs{foreachsc}}
% \label{sec:foreachsc}
%
% \vspace*{-10pt}
%
% \begin{function}{\foreachsc}
%   \begin{syntax}
%     \cmdexamp{foreachsc}[key \textnormal{\textcolor{gray}{=}} val]{seq name}
%   \end{syntax}
% \end{function}
%
% The command \ics*{foreachsc} goes through and executes the command §\getstored§
% on the contents stored in \mymarg{seq name}. If you pass without options run
% §\getstored§ on all contents stored in \mymarg{seq name}.
%
% \subsubsection*{Options for command}
% \label{sec:optcmdfor}
%
% \keyexamp{sep}{code}{empty}
% Establishes the separation between each content stored in \mymarg[type=tt]{seq name}.
% For example, you can use §sep={\\[10pt]}§ for vertical separation of stored
% contents.
%
% \keyexamp{step}{integer}{1}
% Sets the increment (\mymeta{step}) applied to the value set by key §start§
% for each element stored in the \mymarg[type=tt]{seq name}. The value must be a
% \mymeta{positive integer}.
%
% \keyexamp{start}{integer}{1}
% Sets the \mymeta{index} number of the \mymarg[type=tt]{seq name} from which execution
% will start. The value must be a \mymeta{positive integer}.
%
% \keyexamp{stop}{integer}{total}
% Sets the \mymeta{index} number of the \mymarg[type=tt]{seq name} from which execution
% it will finish executing. The value must be a \mymeta{positive integer}.
%
% \keyexamp{before}{code}{empty}
% Sets the \mymarg[type=tt]{code} that will be executed \mymeta{before} each content stored
% in \mymarg[type=tt]{seq name}. The \mymarg[type=tt]{code} must be passed between braces.
%
% \keyexamp{after}{code}{empty}
% Sets the \mymarg[type=tt]{code} that will be executed \mymeta{after} each content stored
% in \mymarg[type=tt]{seq name}. The \mymarg[type=tt]{code} must be passed between braces.
%
% \newsavebox{\hashbox}
% \begin{lrbox}{\hashbox}
% \lstinline[language=scontents-doc,basicstyle=\small\ttfamily]+{#1}+
% \end{lrbox}
%
% \keyexamp{wrapper}{code{} \space\usebox{\hashbox} more code}{empty}
% Wraps the content stored in \mymarg[type=tt]{seq name} referenced by §{#1}§.
% The \mymarg[type=tt]{code} must be passed between braces. For example
% §\foreachsc[wrapper={\makebox[1em][l]{#1}}]{contents}§.
%
%
% \subsection{The command \cs{typestored}}
% \label{sec:typestored}
%
% \vspace*{-10pt}
%
% \begin{function}{\typestored}
%   \begin{syntax}
%      \cmdexamp{typestored}[index\textnormal{\textcolor{gray}{,}} 1-end\textnormal{\textcolor{gray}{,}} width-tab \textnormal{\textcolor{gray}{=}} number]{seq name}
%   \end{syntax}
% The command \ics*{typestored} internally places the content stored in
% the \mymarg[type=tt]{seq name} into the \myenv*{verbatimsc} environment. The \mymeta{index}
% corresponds to the position in which the content is stored in the \mymarg[type=tt]{seq name},
% if \mymeta[type=tt]{1-end} is used \emph{\enquote{all}} content stored in \mymarg[type=tt]{seq name} will be printed.
% \end{function}
%
% If the optional argument is not passed it defaults to the first element
% stored in the \mymarg[type=tt]{seq name}. The key \mykey{width-tab} is available for
% this command.
%
% \subsection{The command \cs{mergesc}}
% \label{sec:mergesc}
%
% \vspace*{-10pt}
%
% \newsavebox{\argmergesc}
% \begin{lrbox}{\argmergesc}
% \small\mymarg[type=tt]{seq A}\myoarg[type=tt]{index}\textcolor{gray}{, }\mymarg[type=tt]{seq B}\myoarg[type=tt]{start \textcolor{gray}{-} stop}\textcolor{gray}{, }\mymarg[type=tt]{seq C}\myoarg[type=tt]{1-end}
% \end{lrbox}
%
% \begin{function}{\mergesc}
%   \begin{syntax}
%      \cmdexamp{mergesc}[typestored \textnormal{\textcolor{lightgray}{\textbar}} meaningsc\textcolor{lightgray}{, }keys]{\usebox{\argmergesc}}
%   \end{syntax}
% The command \ics*{mergesc} internally assembles the content stored in
% the \mymarg[type=tt]{seq A}§[1]§, \mymarg[type=tt]{seq B}§[2-5]§ and \mymarg[type=tt]{seq C}§[1-end]§ into a temporary internal
% \mymeta[type=tt]{seq temp}.
% \end{function}
%
% The use of the keys \mykey{typestored} or \mykey{meaningsc} are
% \emph{\enquote{mandatory}} and disjoint from each other, the rest of the
% accepted \mymeta{keys} are \mykey{print-cmd}, \mykey{write-out},
% \mykey{width-tab} and \mykey{overwrite}.
%
% The use of the \mykey{write-out} key with this command follows the same
% rules already described, the main advantage is that it allows to join
% stored content \emph{without rewriting} the file over and over again,
% by design \hologo{TeX} does not have an append mode for writing
% files, this effectively allows you to write chunks of code and then
% merge them into a single file.
%
% \newpage
%
% \subsection{The environment \env{verbatimsc}}
% \label{sec:verbatimsc}
%
% \vspace*{-10pt}
%
% \begin{function}{verbatimsc}
% Internal environment used by \ics*{typestored} to display \mymeta{verbatim style}
% contents.
% \end{function}
%
% One consideration to keep in mind is that this is a \enquote{\emph{representation}}
% of the \mymeta{stored content} in a \enquote{\emph{verbatim}} environment.
%
% The \myenv*{verbatimsc} environment can be customized in the
% following ways after loading the \mypkg*{scontents} package:
%
% Using the package \mypkg{fancyvrb}:
% \iffalse
%<*example>
% \fi
\begin{examplecode}[frame=single]
\makeatletter
\let\verbatimsc\@undefined
\let\endverbatimsc\@undefined
\makeatother
\usepackage{fancyvrb}
\DefineVerbatimEnvironment{verbatimsc}{Verbatim}{numbers=left}
\end{examplecode}
% \iffalse
%</example>
% \fi
% Using the package \mypkg{minted}:
% \iffalse
%<*example>
% \fi
\begin{examplecode}[frame=single]
\makeatletter
\let\verbatimsc\@undefined
\let\endverbatimsc\@undefined
\makeatother
\usepackage{minted}
\newminted{tex}{linenos}
\newenvironment{verbatimsc}{\VerbatimEnvironment\begin{texcode}}{\end{texcode}}
\end{examplecode}
% \iffalse
%</example>
% \fi
% Using the package \mypkg{listings}:
% \iffalse
%<*example>
% \fi
\begin{examplecode}[frame=single]
\makeatletter
\let\verbatimsc\@undefined
\let\endverbatimsc\@undefined
\makeatother
\usepackage{listings}
\lstnewenvironment{verbatimsc}
  {
   \lstset{
           basicstyle=\small\ttfamily,
           columns=fullflexible,
           language=[LaTeX]TeX,
           numbers=left,
           numberstyle=\tiny\color{gray},
           keywordstyle=\color{red}
          }
  }{}
\end{examplecode}
% \iffalse
%</example>
% \fi
%
% \section{Other commands provided}
%
% \subsection{The command \cs{meaningsc}}
% \label{sec:meaningsc}
%
% \vspace*{-10pt}
%
% \begin{function}{\meaningsc}
%   \begin{syntax}
%      \cmdexamp{meaningsc}[index\textnormal{\textcolor{gray}{,}} width-tab \textnormal{\textcolor{gray}{=}} number]{seq name}
%   \end{syntax}
% The command \ics*{meaningsc} executes \ics{meaning} on the content stored
% in \mymarg{seq name}. The \mymeta{index} corresponds to the position
% in which the content is stored in the \mymarg{seq name}.
% \end{function}
%
% If the optional argument is not passed it defaults to the first element
% stored in the \mymarg{seq name}. The key \mykey{width-tab} is available for
% this command.
%
% \subsection{The command \cs{countsc}}
% \label{sec:countsc}
%
% \vspace*{-10pt}
%
% \begin{function}{\countsc}
%   \begin{syntax}
%      \cmdexamp{countsc}{seq name}
%   \end{syntax}
% The command \ics*{countsc} count a number of contents stored in \mymarg{seq name}.
% \end{function}
%
% \subsection{The command \cs{cleanseqsc}}
% \label{sec:cleansc}
%
% \vspace*{-10pt}
%
% \begin{function}{\cleanseqsc}
%   \begin{syntax}
%      \cmdexamp{cleanseqsc}{seq name}
%   \end{syntax}
% The command \ics*{cleanseqsc} remove all contents stored in \mymarg{seq name}.
% \end{function}
%
% \section[The \textnormal{\texttt{scontents}} package in action]{The \mypkg*{scontents} package in action}
%
% Remember the abstract on the first page?, this is it:
%
% \getstored{abstract}
%
% And the description of the package?
%
% \getstored{description}
%
% \medskip
%
% I've only written:
%
% \typestored{abstract}
%
% and
%
% \typestored{description}
%
% Of course, I didn't copy and paste. The real code they were written with is:
%
% \iffalse
%<*example>
% \fi
\begin{examplecode}[numbers=left]
\begin{scontents}[store-env=abstract,print-env=true]
\begin{abstract}
This package allows to store \hologo{LaTeX} code, including \enquote{\emph{verbatim}},
in \mymeta{sequences} using the \mypkg{l3seq} module of \mypkg{expl3}. The \mymeta{stored
content} can be used as many times as desired in the document, additionally you can write
to \mymeta{external files} or show it in \mymeta{verbatim style}.
\end{abstract}
\end{scontents}
\end{examplecode}
% \iffalse
%</example>
% \fi
%
% and
%
% \iffalse
%<*example>
% \fi
\begin{examplecode}[numbers=left]
\begin{scontents}[store-env=description, print-env=true]
The \mypkg*{scontents} package allows to \mymeta{store contents} in \mymeta{sequences}
or \mymeta{external files}. In some ways it is similar to the \mypkg{filecontentsdef}
package, with the difference in which the \mymeta{content} is stored. The idea behind
this package is to get an approach to \hologo{ConTeXt} \enquote{\emph{buffers}} by
making use \mymeta{sequences}.
\end{scontents}
\end{examplecode}
% \iffalse
%</example>
% \fi
%
% I stored the content in memory and then ran §\getstored§ and
% §\typestored§. This is one of the ways you can use \mypkg*{scontents}.
%
% \section{Examples}
%
% These are some adapted examples that have served as inspiration for
% the creation of this package. The examples are attached to this documentation
% and can be extracted from your PDF viewer or from the command line by running:
% \iffalse
%<*example>
% \fi
\begin{examplecode}[frame=single]
$ pdfdetach -saveall scontents.pdf
\end{examplecode}
% \iffalse
%</example>
% \fi
% and then you can use the excellent \arara\footnote{The cool \TeX\ automation tool:
% \url{https://www.ctan.org/pkg/arara}} tool to compile them.
%
% \subsection{From \texttt{answers} package}
%
% \subsubsection*{Example 1}
%
% \iffalse
%<*example>
% \fi
\begin{scontents}[write-out=scexamp1.ltx]
% arara: pdflatex
% arara: clean: { extensions: [ aux, log] }
\documentclass{article}
\usepackage[store-cmd=solutions]{scontents}
\newtheorem{ex}{Exercise}
\setlength{\parindent}{0pt}
\pagestyle{empty}
\begin{document}
\section{Problems}
\begin{ex}
First exercise
\Scontents{First solution.}
\end{ex}

\begin{ex}
Second exercise
\Scontents{Second solution.}
\end{ex}

\section{Solutions}
\foreachsc[sep={\\[10pt]}]{solutions}
\end{document}
\end{scontents}
% \iffalse
%</example>
% \fi
%
% Adaptation of example 1 of the package \mypkg{answers}
% \textattachfile[color=linkcolor,print=false]{scexamp1.ltx}{\faFile*[regular]}.
% \lstinputlisting[language=scontents-doc,numbers=left]{scexamp1.ltx}
%
% \subsection{From \texttt{filecontentsdef} package}
%
% \subsubsection*{Example 2}
%
% \iffalse
%<*example>
% \fi
\begin{scontents}[write-out=scexamp2.ltx]
% arara: pdflatex
% arara: clean: { extensions: [ aux, log] }
\documentclass{article}
\usepackage[store-env=defexercise,store-cmd=defexercise]{scontents}
\setlength{\parindent}{0pt}
\pagestyle{empty}
\begin{document}
% not starred
\Scontents{
Prove that \[x^n+y^n=z^n\] is not solvable in positive integers if $n$ is at
most $-3$.\par
}
% starred
\Scontents*|Refute the existence of black holes in less than $140$ characters.|
% write environment to \jobname.txt
\begin{scontents}[write-env=\jobname.txt]
\def\NSA{NSA}%
Prove that factorization is easily done via probabilistic algorithms and
advance evidence from knowledge of the names of its employees in the
seventies that the \NSA\ has known that for 40 years.\par
\end{scontents}
% see all stored
\begin{itemize}
\foreachsc[before={\item }]{defexercise}
\end{itemize}
% \getstored are robust :)
\section{\getstored[2]{defexercise}}
\end{document}
\end{scontents}
% \iffalse
%</example>
% \fi
%
% Adaptation of example from package \mypkg{filecontentsdef}
% \textattachfile[color=linkcolor,print=false]{scexamp2.ltx}{\faFile*[regular]}.
% \lstinputlisting[language=scontents-doc,numbers=left]{scexamp2.ltx}
%
% \subsection{From TeX-SX}
%
% \subsubsection*{Example 3}
%
% \iffalse
%<*example>
% \fi
\begin{scontents}[write-out=scexamp3.ltx]
% arara: pdflatex
% arara: clean: { extensions: [ aux, log] }
\documentclass{article}
\usepackage[store-cmd=tikz]{scontents}
\usepackage{tikz}
\setlength{\parindent}{0pt}
\pagestyle{empty}
\Scontents{\matrix{ \node (a) {$a$} ; & \node (b) {$b$} ; \\ } ;}
\Scontents{\matrix[ampersand replacement=\&]
{ \node (a) {$a$} ; \& \node (b) {$b$} ; \\ } ;}
\Scontents{\matrix{\node (a) {$a$} ; & \node (b) {$b$} ; \\ } ; }
\begin{document}
\section{tikzpicture}
\begin{tikzpicture}
\getstored[1]{tikz}
\end{tikzpicture}

\begin{tikzpicture}
\getstored[2]{tikz}
\end{tikzpicture}

\begin{tikzpicture}
\getstored{tikz}
\end{tikzpicture}

\begin{scontents}[store-env=buffer]
Hello World!

This is a \verb*|fake poor man's buffer :)|.
\end{scontents}

\section{source tikz}
\typestored[1]{tikz}
\typestored[2]{tikz}
\typestored[3]{tikz}

\section{fake buffer}
\subsection{real content}
\getstored[1]{buffer}
\subsection{verbatim style}
\typestored[1]{buffer}
\subsection{meaning}
\meaningsc[1]{buffer}

\section{tikz again}
\foreachsc[before={\begin{tikzpicture}},after={\end{tikzpicture}},sep={\\[10pt]}]{tikz}
\end{document}
\end{scontents}
% \iffalse
%</example>
% \fi
%
% Adapted from \href{https://tex.stackexchange.com/q/5338/7832}{LaTeX equivalent of ConTeXt buffers}
% \textattachfile[color=linkcolor,print=false]{scexamp3.ltx}{\faFile*[regular]}.
% \lstinputlisting[language=scontents-doc,numbers=left]{scexamp3.ltx}
%
% \subsubsection*{Example 4}
%
% \iffalse
%<*example>
% \fi
\begin{scontents}[write-out=scexamp4.ltx]
% arara: pdflatex
% arara: clean: { extensions: [ aux, log] }
\documentclass{article}
\usepackage{scontents}
\setlength{\parindent}{0pt}
\pagestyle{empty}
\begin{document}
\begin{scontents}[store-env=main]
Something for main A.
\end{scontents}

\begin{scontents}[store-env=main]
Something for \verb|main B|.
\end{scontents}

\begin{scontents}[store-env=other]
Something for \verb|other|.
\end{scontents}

\textbf{Let's print them}

This is first stored in main: \getstored[1]{main}\par
This is second stored in main: \getstored{main}\par
This is stored in other: \getstored{other}

\textbf{Print all of stored in main}\par
\foreachsc[sep={\\[10pt]}]{main}
\end{document}
\end{scontents}
% \iffalse
%</example>
% \fi
%
% Adapted from \href{https://tex.stackexchange.com/q/184503/7832}{Collecting contents of environment and store them for later retrieval}
% \textattachfile[color=linkcolor,print=false]{scexamp4.ltx}{\faFile*[regular]}.
% \lstinputlisting[language=scontents-doc,numbers=left]{scexamp4.ltx}
%
% \subsubsection*{Example 5}
%
% \iffalse
%<*example>
% \fi
\begin{scontents}[write-out=scexamp5.ltx]
% arara: pdflatex
% arara: clean: { extensions: [ aux, log] }
\documentclass{article}
\usepackage{scontents}
\setlength{\parindent}{0pt}
\pagestyle{empty}
\begin{document}
\section{Problem stated the first time}
\begin{scontents}[print-env=true,store-env=problem]
This is normal text.
\verb|This is from the verb command.|
\verb*|This is from the verb* command.|
This is normal text.
\begin{verbatim}
This is from the verbatim environment:
&%{}~
\end{verbatim}
\end{scontents}
\section{Problem restated}
\getstored[1]{problem}
\section{Problem restated once more}
\getstored[1]{problem}
\end{document}
\end{scontents}
% \iffalse
%</example>
% \fi
%
% Adapted from \href{https://tex.stackexchange.com/q/373647/7832}{Collect contents of an environment (that contains verbatim content)}
% \textattachfile[color=linkcolor,print=false]{scexamp5.ltx}{\faFile*[regular]}.
% \lstinputlisting[language=scontents-doc,numbers=left]{scexamp5.ltx}
%
% \subsubsection*{Example 6}
%
% \iffalse
%<*example>
% \fi
\begin{scontents}[write-out=scexamp6.ltx]
% arara: pdflatex
% arara: clean: { extensions: [ aux, log] }
\documentclass[10pt]{article}
\usepackage{scontents}
\newenvsc{forshort}[store-env=forshort,print-env=false]
\setlength{\parindent}{0pt}
\pagestyle{empty}
\begin{document}

Something in the whole course.

\begin{forshort}
    Just a summary...
\end{forshort}

\end{document}
\end{scontents}
% \iffalse
%</example>
% \fi
%
% Adapted from \href{https://tex.stackexchange.com/q/509595/7832}{Environment hiding its content}
% \textattachfile[color=linkcolor,print=false]{scexamp6.ltx}{\faFile*[regular]}.
% \lstinputlisting[language=scontents-doc,numbers=left]{scexamp6.ltx}
%
% \subsection{Customization of \texttt{verbatimsc}}
%
% \subsubsection*{Example 7}
%
% \iffalse
%<*example>
% \fi
\begin{scontents}[write-out=scexamp7.ltx]
% arara: pdflatex
% arara: clean: { extensions: [ aux, log] }
\documentclass{article}
\usepackage{scontents}
\makeatletter
\let\verbatimsc\@undefined
\let\endverbatimsc\@undefined
\makeatother
\usepackage{fvextra}
\usepackage{xcolor}
\definecolor{mygray}{gray}{0.9}
\usepackage{tcolorbox}
\newenvironment{verbatimsc}%
{\VerbatimEnvironment
\begin{tcolorbox}[colback=mygray, boxsep=0pt, arc=0pt, boxrule=0pt]
\begin{Verbatim}[fontsize=\scriptsize, breaklines, breakafter=*, breaksymbolsep=0.5em,
breakaftersymbolpre={\,\tiny\ensuremath{\rfloor}}]}%
{\end{Verbatim}%
\end{tcolorbox}}
\setlength{\parindent}{0pt}
\pagestyle{empty}
\begin{document}
\section{Test \texttt{\textbackslash begin\{scontents\}} with \texttt{fancyvrb}}
Test \verb+{scontents}+ \par

\begin{scontents}
Using \verb+scontents+ env no \verb+[key=val]+, save in seq \verb+contents+
with index 1.

Prove new \Verb*{ fancyvrb with braces } and environment \verb+Verbatim*+
\begin{verbatim}
 verbatim  environment
\end{verbatim}
\end{scontents}

\section{Test \texttt{\textbackslash Scontents} with \texttt{fancyvrb}}
\Scontents{ We have coded this in \LaTeX: $E=mc^2$.}

\section{Test \texttt{\textbackslash getstored}}
\getstored[1]{contents}\par
\getstored{contents}

\section{Test \texttt{\textbackslash meaningsc}}
\meaningsc[1]{contents}\par
\meaningsc[2]{contents}

\section{Test \texttt{\textbackslash typestored}}
\typestored[1]{contents}
\typestored[2]{contents}
\end{document}
\end{scontents}
% \iffalse
%</example>
% \fi
%
% Customization of \myenv*{verbatimsc} using the \mypkg{fancyvrb} and
% \mypkg{tcolorbox} package
% \textattachfile[color=linkcolor,print=false]{scexamp7.ltx}{\faFile*[regular]}.
% \lstinputlisting[language=scontents-doc,numbers=left]{scexamp7.ltx}
%
% \subsubsection*{Example 8}
%
% \iffalse
%<*example>
% \fi
\begin{scontents}[write-out=scexamp8.ltx]
% arara: pdflatex
% arara: clean: { extensions: [ aux, log] }
\documentclass{article}
\usepackage{scontents}
\makeatletter
\let\verbatimsc\@undefined
\let\endverbatimsc\@undefined
\makeatother
\usepackage{xcolor}
\usepackage{listings}
\lstnewenvironment{verbatimsc}
  {
   \lstset{
           basicstyle=\small\ttfamily,
           breaklines=true,
           columns=fullflexible,
           language=[LaTeX]TeX,
           numbers=left,
           numbersep=1em,
           numberstyle=\tiny\color{gray},
           keywordstyle=\color{red}
          }
  }{}
\setlength{\parindent}{0pt}
\pagestyle{empty}
\begin{document}
\section{Test \texttt{\textbackslash begin\{scontents\}} with \texttt{listings}}
Test \verb+{scontents}+ \par

\begin{scontents}
Using \verb+scontents+ env no \verb+[key=val]+, save in seq \verb+contents+ with index 1.\par

Prove \lstinline[basicstyle=\ttfamily]| lstinline | and environment \verb+Verbatim*+
\begin{verbatim}
  verbatim  environment
\end{verbatim}
\end{scontents}

\section{Test \texttt{\textbackslash Scontents*} with \texttt{listings}}

\Scontents*+ We have coded this in \lstinline[basicstyle=\ttfamily]|\LaTeX: $E=mc^2$|
and more.+

\section{Test \texttt{\textbackslash getstored}}
\getstored{contents}\par
\getstored[1]{contents}

\section{Test \texttt{\textbackslash typestored}}
\typestored[1]{contents}
\typestored[2]{contents}
\end{document}
\end{scontents}
% \iffalse
%</example>
% \fi
%
% Customization of \myenv*{verbatimsc} using the \mypkg{listings} package
% \textattachfile[color=linkcolor,print=false]{scexamp8.ltx}{\faFile*[regular]}.
% \lstinputlisting[language=scontents-doc,numbers=left]{scexamp8.ltx}
%
% \subsubsection*{Example 9}
%
% \iffalse
%<*example>
% \fi
\begin{scontents}[write-out=scexamp9.ltx]
% arara: xelatex: {shell: true, options: [-8bit]}
% arara: clean: { extensions: [ aux, log] }
\documentclass{article}
\usepackage{scontents}
\makeatletter
\let\verbatimsc\@undefined
\let\endverbatimsc\@undefined
\makeatother
\usepackage{minted}
\newminted{tex}{linenos}
\newenvironment{verbatimsc}{\VerbatimEnvironment\begin{texcode}}{\end{texcode}}
\pagestyle{empty}
\setlength{\parindent}{0pt}
\begin{document}
\section{Test \texttt{\textbackslash begin\{scontents\}} with \texttt{minted}}
Test \verb+{scontents}+ \par

\begin{scontents}[overwrite,write-env=\jobname.tsc,force-eol=true]
Using \verb+scontents+ env no \verb+[key=val]+, save in seq \verb+contents+
with index 1.\par

Prove new \Verb*{ new fvextra with braces } and environment \verb+Verbatim*+
\begin{Verbatim}[obeytabs, showtabs, tab=\rightarrowfill, tabcolor=red]
No tab
	One real tab
		Two real Tab plus 	one tab
\end{Verbatim}
\end{scontents}

\section{See \Verb{\jobname.tsc}}
Read \Verb{\jobname.tsc} (shows TABs as red arrows):
\VerbatimInput[obeytabs, showtabs, tab=\rightarrowfill, tabcolor=red]{\jobname.tsc}

\section{Test \texttt{\textbackslash Scontents} with \texttt{minted}}

\Scontents{ We have coded \par this in \LaTeX: $E=mc^2$.}

\section{Test \texttt{\textbackslash getstored}}
\getstored[1]{contents}\par
\getstored{contents}

\section{Test \texttt{\textbackslash typestored}}
\typestored[1]{contents}
\typestored[2]{contents}
\end{document}
\end{scontents}
% \iffalse
%</example>
% \fi
%
% Customization of \myenv*{verbatimsc} using the \mypkg{minted} package
% \textattachfile[color=linkcolor,print=false]{scexamp9.ltx}{\faFile*[regular]}.
% \lstinputlisting[language=scontents-doc,numbers=left]{scexamp9.ltx}
%
% \subsection{The command \texttt{\textbackslash{}mergesc} in action}
%
% \iffalse
%<*example>
% \fi
\begin{scontents}[write-out=scexamp10.ltx]
% arara: pdflatex
% arara: clean: { extensions: [ aux, log] }
\documentclass{article}
\usepackage{scontents}
% Fix part of a MCE that should go before babel's loading
\begin{scontents}[store-env=mce]
\documentclass[french]{article}
\usepackage[T1]{fontenc}
\usepackage[utf8]{inputenc}
\usepackage{lmodern}
\usepackage[a4paper]{geometry}
\end{scontents}
% Fix part of a MCE that should go after (>=) babel's loading
\begin{scontents}[store-env=mce]
\usepackage{babel}
\begin{document}
\end{scontents}
% Fix part of a MCE that should go after its body
\begin{scontents}[store-env=mce]
\end{document}
\end{scontents}
\begin{document}
\section{First annswer}
% Variable part of a MCE that should added to the fixed preamble, before babel's loading
\begin{scontents}[store-env=mce-1]
\usepackage{amsmath}
\end{scontents}
% Variable part of a MCE being the code snippet
\begin{scontents}[store-env=mce-1]
\begin{align}
  0 & \neq 1 \\
  1 & \neq 0
\end{align}
\end{scontents}
\begin{description}
\item[Preamble's addition]\leavevmode
  \typestored[1]{mce-1}
\item[Code snippet]\leavevmode
  \typestored[2]{mce-1}
\item[MCE]\leavevmode
  \mergesc[typestored, print-cmd=true]
    {
      {mce}[1], {mce-1}[1], {mce}[2], {mce-1}[2], {mce}[3]
    }
\end{description}
\section{Second annswer}
% Variable part of a MCE that should added to the fixed preamble, before babel's loading
\begin{scontents}[store-env=mce-2]
\usepackage{amsmath}
\end{scontents}
% Variable part of a MCE being the code snippet
\begin{scontents}[store-env=mce-2]
\begin{flalign}
  0 & \neq 1 \\
  1 & \neq 0
\end{flalign}
\end{scontents}

\begin{description}
\item[Preamble's addition]\leavevmode
  \typestored[1]{mce-2}
\item[Code snippet]\leavevmode
  \typestored[2]{mce-2}
\item[MCE]\leavevmode
  \mergesc[typestored, print-cmd=true, write-out=mce.txt, overwrite=true]
    {
      {mce}[1], {mce-2}[1], {mce}[2], {mce-2}[2], {mce}[3]
    }
\end{description}
\end{document}
\end{scontents}
% \iffalse
%</example>
% \fi
% The command §\mergesc§ in action, adapted from Denis Bitouzé request at \url{https://github.com/pablgonz/scontents/issues/2}
% \textattachfile[color=linkcolor,print=false]{scexamp9.ltx}{\faFile*[regular]}.
% \lstinputlisting[language=scontents-doc,numbers=left]{scexamp10.ltx}
%
% \newpage
%
% \section{Change history}
% \label{sec:changes}
%
% In this section you will find some (not all) of the changes in \mypkg*{scontents}
% development, from the first public implementation using the \mypkg{filecontentsdef}
% package to the current version with only \mypkg{expl3}.
%
% \medskip
%
% \setlist[itemize,1]{label=\textendash,wide=0.5em,nosep,noitemsep,leftmargin=10pt}
% \newlength\descrwidth
% \settowidth{\descrwidth}{\textsf{v1.0a, (ctan), 2019-07-30} }
%
% \begin{description}[font=\small\sffamily,wide=0pt,style=multiline,leftmargin=\descrwidth,nosep,noitemsep]
% \item [\fileversion{} (ctan), \filedate]
%    \begin{itemize}
%        \item Fix internal definition for some functions.
%        \item Replace |\peek_charcode_ignore_spaces:NTF| by |\peek_charcode:NTF|.
%        \item Set correct code for |\obeyedline| implement in \LaTeX{} release 2024-06-01.
%    \end{itemize}
% \item [v2.1 (ctan), 2024-06-14]
%    \begin{itemize}
%        \item Fix |\cleansc| command.
%        \item Add |\mergesc| command.
%        \item Fix internal definition for seq var.
%        \item Fix internal code for |\typestored|.
%        \item Replace |\cs_argument_spec:N| by |\cs_parameter_spec:N|.
%        \item Detect |l3keys2e| package (obsolete in june 2022 \LaTeX{} release).
%        \item Minor adjustments in the documentation.
%    \end{itemize}
% \item [v2.0 (ctan), 2022-04-04]
%    \begin{itemize}
%        \item Adapting the |verbatimsc| environment (compatibility |verbatim| package).
%        \item Removed compatibility layer for older \hologo{LaTeX} releases.
%        \item Fix loader in \hologo{plainTeX} and \hologo{ConTeXt}.
%        \item Minor adjustments in the documentation.
%    \end{itemize}
% \item [v1.9 (ctan), 2020-01-21]
%    \begin{itemize}
%        \item Update and improvements in the internal code.
%        \item Updating the generic code for |I/O| verification.
%        \item Add |write-cmd| and |write-out| keys for |\Scontents*|.
%        \item Fix |sep| key in |\foreachsc|.
%    \end{itemize}
% \item [v1.8 (ctan), 2019-11-18]
%    \begin{itemize}
%        \item Add |\newenvsc| command.
%        \item Fix nested environment in \hologo{plainTeX} and \hologo{ConTeXt}.
%        \item Modified default value in |\getstored|.
%        \item Add |overwrite| key to reduce |I/O| operations.
%        \item Deleted an unnecessary group in the code.
%    \end{itemize}
% \item [v1.7 (ctan), 2019-10-29]
%    \begin{itemize}
%        \item The |verbatimsc| environment was rewritten.
%        \item Minor adjustments in documentation.
%    \end{itemize}
% \item [v1.6 (ctan), 2019-10-26]
%    \begin{itemize}
%        \item The internal behavior of |\getstored| has been modified.
%        \item The internal behavior of |\foreachsc| has been modified.
%        \item Corrected file extension for \hologo{ConTeXt}.
%        \item Remove spurious warning.
%    \end{itemize}
% \item [v1.5 (ctan), 2019-10-24]
%    \begin{itemize}
%        \item Add support for \hologo{plainTeX} and \hologo{ConTeXt}.
%        \item Split internal code for optimization.
%        \item Add support for vertical spaces in |key=val|.
%        \item Add |\foreachsc| command.
%        \item Check if |verbatim| package is loaded.
%    \end{itemize}
% \item [v1.4 (ctan), 2019-10-03]
%    \begin{itemize}
%        \item Add |store-all| key.
%        \item Messages and keys were separated.
%        \item Restructuring of documentation.
%        \item Now the version of |expl3| is checked instead of |xparse|.
%        \item The internal behavior of |force-eol| has been modified.
%    \end{itemize}
% \item [v1.3 (ctan), 2019-09-24]
%    \begin{itemize}
%        \item The environment can now nest.
%        \item Added |force-eol|, |verb-font| and |width-tab| keys.
%        \item The extra space has been removed when you run |\getstored|.
%        \item Internal code has been rewritten more efficiently.
%        \item Remove starred argument for |\typestored|.
%        \item Remove |filecontentsdef| dependency.
%        \item Changing |\regex_replace_all:| for |\tl_replace_all:|.
%    \end{itemize}
% \item [v1.2 (ctan), 2019-08-28]
%    \begin{itemize}
%        \item Restructuring of documentation.
%        \item Added copy of |\tex_scantokens:|.
%    \end{itemize}
% \item [v1.1 (ctan), 2019-08-12]
%    \begin{itemize}
%        \item Extension of documentation.
%        \item Replace |\tex_endinput:D| by |\file_input_stop:|.
%    \end{itemize}
% \item [v1.0 (ctan), 2019-07-30]
%    \begin{itemize}
%        \item First public release.
%    \end{itemize}
% \end{description}
%
% \newpage
%
% \indexprologue{
% The italic numbers denote the pages where the corresponding entry is
% described.}
%
% \printindex[userdoc]
%
% \let\stdsection\section
% \def\section*#1{\stdsection{#1}}
%
% \begin{thebibliography}{9}
% \bibitem{expl3} The \hologo{LaTeX} Project. \enquote{The \textsf{expl3}
% package}. Available from
% \textsc{ctan}, \url{https://www.ctan.org/pkg/expl3}, 2023.
%
% \bibitem{xparse} The \hologo{LaTeX} Project. \enquote{The \textsf{xparse}
% package}. Available from
% \textsc{ctan}, \url{https://www.ctan.org/pkg/xparse}, 2023.
%
% \bibitem{l3keys} The \hologo{LaTeX} Project. \enquote{The \textsf{l3keys2e}
% package}. Available from
% \textsc{ctan}, \url{https://www.ctan.org/pkg/l3keys2e}, 2022.
%
% \bibitem{keyval} \textsc{Wright, Joseph}. \enquote{Programming
% \textsf{key–value} in \textsf{expl3}}. Available from
% \textsc{TUGboat},
% \url{https://www.tug.org/TUGboat/tb31-1/tb97wright-l3keys.pdf}, 2010.
% \end{thebibliography}
%
% \let\section\stdsection
%
% \newpage
%
% \StartImplementation
%
% \StopEventually{^^A
% \newgeometry{top=0.5in,bottom=0.3in,left=1.0in,right=0.5in,footskip=0.2in,headheight=1cm,headsep=0.27cm}
% \addtocontents{toc}{\protect\setcounter{tocdepth}{2}}
% \cleardoublepage
% \phantomsection
% \indexprologue{
% The italic numbers denote the pages where the corresponding entry is
% described, the numbers underlined and all others indicate the line on
% which they are implemented in the package code.
% }
% \printindex
% }
%
% \section{Implementation}
% \label{sec:Implementation}
% \addtocontents{toc}{\protect\setcounter{tocdepth}{0}}
%
% The most recent publicly released version of \mypkg*{scontents} is available at
% \textsc{ctan}: \url{https://www.ctan.org/pkg/scontents}. Historical and
% developmental versions are available at \textcolor{gray}{\scriptsize\faIcon[regular]{github}}
% \url{https://github.com/pablgonz/scontents}. While general feedback via email is
% welcomed, specific bugs or feature requests should be reported through the issue
% tracker: \url{https://github.com/pablgonz/scontents/issues}.
%
% \subsection{Declaration of the package}
%
% First we set up the module name for \pkg{l3doc}:
%    \begin{macrocode}
%<@@=scontents>
%    \end{macrocode}
%
% Now we define some common macros to hold the package date and version:
%    \begin{macrocode}
%<loader>\def\ScontentsFileDate{2025-03-26}%
%<core>\def\ScontentsCoreFileDate{2025-03-26}%
%<*loader>
\def\ScontentsFileVersion{2.2}%
\def\ScontentsFileDescription{Stores LaTeX contents in memory or files}%
%    \end{macrocode}
%
% The \LaTeX{} loader is fairly simple: just load the dependencies, load the
% core code, and then set interfaces up.
%
%    \begin{macrocode}
%<*latex>
\IfFormatAtLeastTF { 2022-06-01 }
  { }
  {
    \RequirePackage{l3keys2e}[2020/02/08]
    \PackageWarning { scontents }
      {
        The next update removes compatibility with versions prior to 2024.
      }
  }
\ProvidesExplPackage
  {scontents} {\ScontentsFileDate} {\ScontentsFileVersion} {\ScontentsFileDescription}
%</latex>
%    \end{macrocode}
%
% The \hologo{plainTeX} and \hologo{ConTeXt} loaders are similar
% (probably because I don't know how to make a proper \hologo{ConTeXt}
% module :-). We define a \LaTeX{}-style §\ver@scontents.sty§ macro with
% version info (just in case) and add §\ExplSyntaxOn§ to be able to load
% \pkg{xparse} later.
%
%    \begin{macrocode}
%<*!latex>
%<context>\writestatus{loading}{User Module scontents v\ScontentsFileVersion}
%<context>\unprotect
\input expl3-generic.tex
\ExplSyntaxOn
\tl_gset:cx { ver @ scontents . sty } { \ScontentsFileDate\space
  v\ScontentsFileVersion\space \ScontentsFileDescription }
\iow_log:x { Package: ~ scontents ~ \use:c { ver @ scontents . sty } }
%</!latex>
%    \end{macrocode}
%
% In \hologo{plainTeX}, check that the package isn't being loaded twice
% (\LaTeX{} and \hologo{ConTeXt} already defend against that):
%    \begin{macrocode}
%<*plain>
\msg_set:nnn { scontents } { already-loaded }
  { The~'scontents'~package~is~already~loaded.~Aborting~input~\msg_line_context:. }
\cs_if_exist:NT \@@_rescan_tokens:n
  {
    \msg_warning:nn { scontents } { already-loaded }
    \ExplSyntaxOff
    \file_input_stop:
  }
%</plain>
%    \end{macrocode}
%
% \subsection{Definition of variables by format}
%
% We define and set variables that must be handled separately in order
% to work properly with \hologo{plainTeX}, \hologo{ConTeXt} and \LaTeX.
%
% \begin{macro}{\g_@@_end_verbatimsc_tl}
%   A global token list \cs{g_@@_end_verbatimsc_tl} match when ending
%   \env{verbatimsc} environment.
%
%    \begin{macrocode}
\tl_new:N \g_@@_end_verbatimsc_tl
\tl_gset_rescan:Nnn \g_@@_end_verbatimsc_tl
  {
    \char_set_catcode_other:N \\
%<*latex>
    \char_set_catcode_other:N \{
    \char_set_catcode_other:N \}
%</latex>
  }
%<latex>  { \end{verbatimsc} }
%<plain>  { \endverbatimsc }
%<context>  { \stopverbatimsc }
%    \end{macrocode}
% \end{macro}
%
% \begin{macro}{\c_@@_end_env_tl,\l_@@_env_name_tl}
%   A token list \cs{c_@@_end_env_tl} match when ending environments
%   defined by \cs{newenvsc}, \cs{l_@@_env_name_tl} storing the name
%   of environments defined by \cs{newenvsc}.
%
%    \begin{macrocode}
\tl_new:N \l_@@_env_name_tl
\tl_const:Nx \c_@@_end_env_tl
  {
    \c_backslash_str
%<latex|plain>    end
%<context>    stop
%<latex>    \c_left_brace_str
      \exp_not:N \l_@@_env_name_tl
%<latex>    \c_right_brace_str
  }
%    \end{macrocode}
% \end{macro}
%
% Now we load the core \mypkg*{scontents} code:
%    \begin{macrocode}
\file_input:n { scontents-code.tex }
%    \end{macrocode}
%
% \begin{macro}{\@@_format_case:nnn}
%   Sometimes we need to detect the format from within a macro:
%    \begin{macrocode}
\cs_new:Npn \@@_format_case:nnn #1 #2 #3
%<latex>  {#1} % LaTeX
%<plain>  {#2} % Plain/Generic
%<context>  {#3} % ConTeXt
%    \end{macrocode}
% \end{macro}
%
% Checking that the package was loaded with the proper loader code. This code
% was copied from \texttt{expl3-code.tex}.
%    \begin{macrocode}
%</loader>
%<*core>
\begingroup
  \catcode32=10
  \endlinechar=32
  \def\next{\endgroup}%
  \expandafter\ifx\csname PackageError\endcsname\relax
    \begingroup
      \def\next{\endgroup\endgroup}%
      \def\PackageError#1#2#3%
        {%
          \endgroup
          \errhelp{#3}%
          \errmessage{#1 Error: #2!}%
        }%
  \fi
  \expandafter\ifx\csname ScontentsFileDate\endcsname\relax
    \def\next
      {%
        \PackageError{scontents}{No scontents loader detected}
          {%
            You have attempted to use the scontents code directly rather than using
            the correct loader. Loading of scontents will abort.
          }%
        \endgroup
        \endinput
      }%
  \else
    \ifx\ScontentsFileDate\ScontentsCoreFileDate
    \else
      \def\next
        {%
          \PackageError{scontents}{Mismatched scontents files detected}
            {%
              You have attempted to load scontents with mismatched files:
              probably you have one or more files 'locally installed' which
              are in conflict. Loading of scontents will abort.
            }%
          \endgroup
          \endinput
        }%
    \fi
  \fi
\next
%    \end{macrocode}
%
% \subsection{Definition of temporary variables}
%
% \begin{macro}{\l_@@_macro_tmp_tl,\l_@@_temp_tl,\g_@@_temp_tl,\l_@@_tmpa_int,
%               \l_@@_temp_bool}
%   The token list \cs{l_@@_macro_tmp_tl} is a temporary token list to
%   hold the contents of the macro/environment. \cs{l_@@_temp_tl},
%   \cs{g_@@_temp_tl}, \cs{l_@@_tmpa_int} and \cs{l_@@_temp_bool} are
%   generic temporary vars.
%    \begin{macrocode}
\tl_new:N \l_@@_macro_tmp_tl
\tl_new:N \l_@@_temp_tl
\tl_new:N \g_@@_temp_tl
\int_new:N \l_@@_tmpa_int
\bool_new:N \l_@@_temp_bool
%    \end{macrocode}
% \end{macro}
%
% \begin{macro}{\l_@@_keys_tl}
%   Stores unused keys to be frwarded to other commands.
%    \begin{macrocode}
\tl_new:N \l_@@_keys_tl
%    \end{macrocode}
% \end{macro}
%
% \subsection{Compatibility layer with \hologo{plainTeX} and \hologo{ConTeXt}}
%
% When loading the package outside of \LaTeX{} we can't usually use
% \pkg{xparse}. However since \pkg{xparse} now \pkg{ltcmd} is part of
% the \LaTeX{} kernel is loadable in any format.
%
%    \begin{macrocode}
%</core>
%<*loader&!latex>
\int_set:Nn \l_@@_tmpa_int { \char_value_catcode:n { `\@ } }
\char_set_catcode_letter:N \@
\file_input:n { xparse-generic.tex }
\char_set_catcode:nn { `\@ } { \l_@@_tmpa_int }
%</loader&!latex>
%<*core>
%    \end{macrocode}
%
% \subsection{Definition of keys for the package}
%
% We create some common \mymeta{keys} that will be used by the options
% passed to the package as well as by the environments and commands
% defined.
%
%    \begin{macrocode}
\keys_define:nn { scontents }
  {
    store-env .tl_set:N         = \l_@@_name_seq_env_tl,
    store-env .initial:n        = contents,
    store-env .value_required:n = true,
    store-cmd .tl_set:N         = \l_@@_name_seq_cmd_tl,
    store-cmd .initial:n        = contents,
    store-cmd .value_required:n = true,
    verb-font .tl_set:N         = \l_@@_verb_font_tl,
    verb-font .value_required:n = true,
    print-env .bool_set:N       = \l_@@_print_env_bool,
    print-env .initial:n        = false,
    print-env .default:n        = true,
    print-cmd .bool_set:N       = \l_@@_print_cmd_bool,
    print-cmd .initial:n        = false,
    print-cmd .default:n        = true,
    force-eol .bool_set:N       = \l_@@_forced_eol_bool,
    force-eol .initial:n        = false,
    force-eol .default:n        = true,
    overwrite .bool_set:N       = \l_@@_overwrite_bool,
    overwrite .initial:n        = false,
    overwrite .default:n        = true,
    width-tab .int_set:N        = \l_@@_tab_width_int,
    width-tab .initial:n        = 1,
    width-tab .value_required:n = true,
    print-all .meta:n           = { print-env = #1 , print-cmd = #1 },
    print-all .default:n        = true,
    store-all .meta:n           = { store-env = #1 , store-cmd = #1 },
    store-all .value_required:n = true
  }
%</core>
%<loader>\keys_define:nn { scontents }
%<latex>  { verb-font .initial:n = \ttfamily }
%<plain|context>  { verb-font .initial:n = \tt }
%    \end{macrocode}
%
% In \LaTeX{} mode we load \mypkg{l3keys2e} process the \mymeta{keys}
% as options passed on to the package, the package \pkg{l3keys2e} will
% verify the \mymeta{keys} and will return an error when they are
%    \begin{macrocode}
%<*latex>
\IfFormatAtLeastTF { 2022-06-01 }
  {
    \ProcessKeyOptions [ scontents ]
  }
  { \ProcessKeysOptions { scontents } }
%</latex>
%<*core>
%    \end{macrocode}
%
% \subsection{Internal variables and utility functions}
%
% \begin{macro}{\l_@@_fname_out_tl,\l_@@_every_line_env_tl,\l_@@_file_iow}
%   The token list \cs{l_@@_fname_out_tl} is used for store the name of
%   the \meta{output file}, when there's one. Its value is set by the
%   keys §write-env§, §write-out§ and §write-cmd§.
%
%   The token list \cs{l_@@_every_line_env_tl} holds the contents of an
%   environment, \env{scontents} by default, as it's being read.
%   \cs{l_@@_file_iow} is an output stream for saving the contents of an
%   environment (or command) to a file.
%
%   This variables is used by the function \cs{@@_file_tl_write_start:n}
%   (see \ref{the:environment:itself}).
%
%    \begin{macrocode}
\tl_new:N \l_@@_fname_out_tl
\tl_new:N \l_@@_every_line_env_tl
\iow_new:N \l_@@_file_iow
%    \end{macrocode}
% \end{macro}
%
% \begin{macro}{\l_@@_foreach_name_seq_tl,\l_@@_foreach_before_tl,\l_@@_foreach_after_tl}
%   \cs{l_@@_foreach_name_seq_tl} is the name assigned to the sequence on
%   which the loop will be made, \cs{l_@@_foreach_before_tl} and \cs{l_@@_foreach_after_tl}
%   are token lists in which the assigned material will be placed before
%   and after the execution of the §\foreachsc§ loop.
%
%    \begin{macrocode}
\tl_new:N \l_@@_foreach_name_seq_tl
\tl_new:N \l_@@_foreach_before_tl
\tl_new:N \l_@@_foreach_after_tl
%    \end{macrocode}
% \end{macro}
%
% \begin{macro}{\l_@@_env_nesting_int,\l_@@_foreach_stop_int}
%   \cs{l_@@_seq_item_seq} stores the indexes in the sequence of the items
%   requested to §\typestored§ or §\meaningsc§. \cs{l_@@_env_nesting_int}
%   stores the current nesting level of the \env{scontents} environment.
%   \cs{l_@@_foreach_stop_int} will save the value at which the §\foreachsc§
%   loop will stop.
%
%    \begin{macrocode}
\int_new:N \l_@@_foreach_stop_int
\seq_new:N \l_@@_seq_item_seq
\int_new:N \l_@@_env_nesting_int
%    \end{macrocode}
% \end{macro}
%
% \begin{macro}{\l_@@_writing_bool,\l_@@_storing_bool,\l_@@_writable_bool}
%   The boolean \cs{l_@@_writing_bool} keeps track of whether we should write
%   to a file, and \cs{l_@@_storing_bool} determines whether it is in write-only
%   mode when the key §write-out§ is used.
%
%    \begin{macrocode}
\bool_new:N \l_@@_writing_bool
\bool_set_false:N \l_@@_writing_bool
\bool_new:N \l_@@_storing_bool
\bool_set_true:N  \l_@@_storing_bool
\bool_new:N \l_@@_writable_bool
%    \end{macrocode}
% \end{macro}
%
% \begin{macro}{\l_@@_foreach_before_bool,\l_@@_foreach_after_bool,
%               \l_@@_foreach_stop_bool,\l_@@_foreach_wrapper_bool}
%   Boolean variables used by the §\foreachsc§ loop.
%
%    \begin{macrocode}
\bool_new:N \l_@@_foreach_before_bool
\bool_set_false:N \l_@@_foreach_before_bool
\bool_new:N \l_@@_foreach_after_bool
\bool_set_false:N \l_@@_foreach_after_bool
\bool_new:N \l_@@_foreach_stop_bool
\bool_set_false:N \l_@@_foreach_stop_bool
\bool_new:N \l_@@_foreach_wrapper_bool
\bool_set_false:N \l_@@_foreach_wrapper_bool
%    \end{macrocode}
% \end{macro}
%
% \begin{macro}{\l_@@_foreach_print_seq}
%   The \cs{l_@@_foreach_print_seq} is the sequence used by §\foreachsc§.
%
%    \begin{macrocode}
\seq_new:N \l_@@_foreach_print_seq
\seq_new:c { g_@@_name_sc!internal_seq }
%    \end{macrocode}
% \end{macro}
%
% \begin{macro}{\c_@@_hidden_space_str}
%   \cs{c_@@_hidden_space_str} is a constant \emph{string} to used to hide
%   the \mymeta{forced space} added by \hologo{TeX} when recording content
%   in a macro. This \emph{string} contains the \emph{reserved phrase}
%   ``§%§§^^Ascheol%§'' which is added to the end of the argument stored
%   in |seq| when the key §force-eol§ is false.
%
%    \begin{macrocode}
\str_const:Nx \c_@@_hidden_space_str
  { \c_percent_str \c_circumflex_str \c_circumflex_str A scheol \c_percent_str }
%    \end{macrocode}
% \end{macro}
%
% \begin{macro}{\q_@@_stop,\q_@@_mark}
%   Some quarks used along the code as macro delimiters.
%
%    \begin{macrocode}
\quark_new:N \q_@@_stop
\quark_new:N \q_@@_mark
%    \end{macrocode}
% \end{macro}
%
% \begin{macro}{\s_@@_stop,\s_@@_mark}
%    \begin{macrocode}
\scan_new:N \s_@@_stop
\scan_new:N \s_@@_mark
%    \end{macrocode}
% \end{macro}
%
% \begin{macro}{\l_@@_cur_seq_name_str}
%    \begin{macrocode}
\str_new:N \l_@@_cur_seq_name_str
%    \end{macrocode}
% \end{macro}
%
% \begin{macro}{\@@_use_i_delimit_by_s_stop:nw,\@@_use_none_delimit_by_s_stop:w}
%    \begin{macrocode}
\cs_new:Npn \@@_use_delimit_by_s_stop:nw #1 \s_@@_stop {#1}
\cs_new:Npn \@@_use_i_delimit_by_s_stop:nw #1 #2 \s_@@_stop {#1}
\cs_new:Npn \@@_use_none_delimit_by_s_stop:w #1 \s_@@_stop { }
%    \end{macrocode}
% \end{macro}
%
% \begin{macro}{\l_@@_save_sf_int,\l_@@_save_skip}
%   Internal variables used by functions \cs{@@_bsphack:} and \cs{@@_esphack:}.
%
%    \begin{macrocode}
\int_new:N \l_@@_save_sf_int
\skip_new:N \l_@@_save_skip
%    \end{macrocode}
% \end{macro}
%
% \begin{macro}{\@@_rescan_tokens:n,\@@_rescan_tokens:x,\@@_rescan_tokens:V}
%   The function \cs{tl_rescan:nn} provided by \pkg{expl3} doesn't fit
%   the needs of this package because it does not allow catcode changes
%   inside the argument, so verbatim commands used inside one of
%   \mypkg*{scontents}'s commands/environments will not work. Here we
%   create a private copy of \cs{tex_scantokens:D} which will serve our
%   purposes. See the answer by Ulrich Diez in
%   \href{https://tex.stackexchange.com/a/577183/7832}{How do use
%   \{<setup>\} in \textbackslash tl_set_rescan:Nnn to replace
%   \textbackslash scantokens?}
%
%    \begin{macrocode}
\cs_new_protected:Npn \@@_rescan_tokens:n #1 { \tex_scantokens:D {#1} }
\cs_generate_variant:Nn \@@_rescan_tokens:n { V, x }
%    \end{macrocode}
% \end{macro}
%
% \begin{macro}{\@@_tab:,\@@_par:}
%   Control sequences to replace tab (§^^I§) and form feed (§^^L§) characters.
%
%    \begin{macrocode}
\cs_new:Npx \@@_tab: { \c_space_tl }
\cs_new:Npn \@@_par: { ^^J ^^J }
%    \end{macrocode}
% \end{macro}
%
% \begin{macro}[int]{\tl_remove_once:NV,\tl_replace_all:Nxx,
%     \tl_replace_all:Nxn,\tl_replace_all:Nnx,\tl_if_empty:fTF}
%   Some nonstandard kernel variants.
%
%    \begin{macrocode}
\cs_generate_variant:Nn \tl_remove_once:Nn { NV }
\cs_generate_variant:Nn \tl_replace_all:Nnn { Nx, Nxx, Nnx }
\cs_generate_variant:Nn \msg_error:nnnn { nnx }
\prg_generate_conditional_variant:Nnn \tl_if_empty:n { f } { p, TF }
%    \end{macrocode}
% \end{macro}
%
% \subsection{Defining keys for the environment and commands}
%
% We add the \mymeta{keys} divided into subgroups to handle errors and
% \emph{unknown} \mymeta{keys} separately.
%
% \subsubsection{Keys for environment \env{scontents}}
%
% We define a set of \mymeta{keys} for environment §scontents§.
%
%    \begin{macrocode}
\keys_define:nn { scontents / scontents }
  {
    write-env .code:n           = {
                                    \bool_set_true:N \l_@@_writing_bool
                                    \tl_set:Nn \l_@@_fname_out_tl {#1}
                                  },
    write-out .code:n           = {
                                    \bool_set_false:N \l_@@_storing_bool
                                    \bool_set_true:N  \l_@@_writing_bool
                                    \tl_set:Nn \l_@@_fname_out_tl {#1}
                                  },
    write-env .value_required:n = true,
    write-out .value_required:n = true,
    print-env .meta:nn          = { scontents } { print-env = #1 },
    print-env .default:n        = true,
    store-env .meta:nn          = { scontents } { store-env = #1 },
    force-eol .meta:nn          = { scontents } { force-eol = #1 },
    force-eol .default:n        = true,
    overwrite .meta:nn          = { scontents } { overwrite = #1 },
    overwrite .default:n        = true,
    unknown   .code:n           = { \@@_parse_environment_keys:n {#1} }
  }
%    \end{macrocode}
%
% \subsubsection{Keys for command \cs{Scontents}}
%
% We define a set of \mymeta{keys} for commands §\Scontents§ and
% §\Scontents*§.
%
%    \begin{macrocode}
\keys_define:nn { scontents / Scontents }
  {
    write-cmd .code:n           = {
                                    \bool_set_true:N \l_@@_writing_bool
                                    \tl_set:Nn \l_@@_fname_out_tl {#1}
                                  },
    write-out .code:n           = {
                                    \bool_set_false:N \l_@@_storing_bool
                                    \bool_set_true:N  \l_@@_writing_bool
                                    \tl_set:Nn \l_@@_fname_out_tl {#1}
                                  },
    write-cmd .value_required:n = true,
    write-out .value_required:n = true,
    print-cmd .meta:nn          = { scontents } { print-cmd = #1 },
    print-cmd .default:n        = true,
    store-cmd .meta:nn          = { scontents } { store-cmd = #1 },
    force-eol .meta:nn          = { scontents } { force-eol = #1 },
    force-eol .default:n        = true,
    overwrite .meta:nn          = { scontents } { overwrite = #1 },
    overwrite .default:n        = true,
    unknown   .code:n           = { \@@_parse_command_keys:n {#1} }
  }
%    \end{macrocode}
%
% \subsubsection{Keys for command \cs{foreachsc}}
%
% We define a set of \mymeta{keys} for command §\foreachsc§.
%
%    \begin{macrocode}
\keys_define:nn { scontents / foreachsc }
  {
    before  .code:n           = {
                                  \bool_set_true:N \l_@@_foreach_before_bool
                                  \tl_set:Nn \l_@@_foreach_before_tl {#1}
                                },
    before  .value_required:n = true,
    after   .code:n           = {
                                  \bool_set_true:N \l_@@_foreach_after_bool
                                  \tl_set:Nn \l_@@_foreach_after_tl {#1}
                                },
    after   .value_required:n = true,
    start   .int_set:N        = \l_@@_foreach_start_int,
    start   .value_required:n = true,
    start   .initial:n        = 1,
    stop    .code:n           = {
                                  \bool_set_true:N \l_@@_foreach_stop_bool
                                  \int_set:Nn \l_@@_foreach_stop_int {#1}
                                 },
    stop    .value_required:n = true,
    step    .int_set:N        = \l_@@_foreach_step_int,
    step    .value_required:n = true,
    step    .initial:n        = 1,
    wrapper .code:n           = {
                                  \bool_set_true:N \l_@@_foreach_wrapper_bool
                                  \cs_set_protected:Npn
                                  \@@_foreach_wrapper:n ##1 {#1}
                                },
    wrapper .value_required:n = true,
    sep     .tl_set:N         = \l_@@_foreach_sep_tl,
    sep     .initial:n        = {},
    sep     .value_required:n = true,
    unknown .code:n           = { \@@_parse_foreach_keys:n {#1} }
  }
%    \end{macrocode}
%
% \subsubsection{Key for commands \cs{typestored} and \cs{meaningsc}}
%
% We define a \mymeta{key} for command §\typestored§ and §\meaningsc§.
% Both commands accept the same type of optional arguments, just define
% a common \mymeta{key}.
%
%    \begin{macrocode}
\bool_new:N \l_@@_print_aux_bool
\bool_set_true:N  \l_@@_print_aux_bool
\keys_define:nn { scontents / typemeaning }
  {
    width-tab .meta:nn   = { scontents } { width-tab = #1 },
    write-out .code:n    = {
                             \bool_set_false:N \l_@@_storing_bool
                             \bool_set_true:N  \l_@@_writing_bool
                             \tl_set:Nn \l_@@_fname_out_tl {#1}
                           },
    overwrite .meta:nn   = { scontents } { overwrite = #1 },
    overwrite .default:n = true,
    unknown   .code:n    = { \@@_parse_type_meaning_key:n {#1} }
  }
%    \end{macrocode}
%
% \subsection{Handling undefined keys}
%
% The \mymeta{keys} are stored in the string variable \cs{l_keys_key_str},
% and the value (if any) is passed as an argument to each \mymeta{function}.
%
% \subsubsection{Undefined keys for environment \env{scontents}}
%
% \begin{macro}{\@@_parse_environment_keys:n,
%               \@@_parse_environment_keys:nn}
%
%   We check the \mymeta{keys} passed to the environment
%   §scontents§ and process it with \cs{@@_parse_environment_keys:n}
%   if the \mymeta{key} is \emph{unknown} we return an error message.
%
%    \begin{macrocode}
\cs_new_protected:Npn \@@_parse_environment_keys:n #1
  { \exp_args:NV \@@_parse_environment_keys:nn \l_keys_key_str {#1} }
\cs_new_protected:Npn \@@_parse_environment_keys:nn #1#2
  {
    \tl_if_blank:nTF {#2}
      { \msg_error:nnn { scontents } { env-key-unknown } {#1} }
      { \msg_error:nnnn { scontents } { env-key-value-unknown } {#1} {#2} }
  }
%    \end{macrocode}
% \end{macro}
%
% \subsubsection{Undefined keys for \cs{Scontents} and \cs{Scontents*}}
%
% \begin{macro}{\@@_parse_command_keys:n,
%               \@@_parse_command_keys:nn}
%
%   We check the \meta{keys} passed to commands §\Scontents§ or
%   §\Scontents*§ and process it with \cs{@@_parse_command_keys:n}
%   if the \mymeta{key} is \emph{unknown} we return an error message.
%
%    \begin{macrocode}
\cs_new_protected:Npn \@@_parse_command_keys:n #1
  { \exp_args:NV \@@_parse_command_keys:nn \l_keys_key_str {#1} }
\cs_new_protected:Npn \@@_parse_command_keys:nn #1#2
  {
    \tl_if_blank:nTF {#2}
      { \msg_error:nnn { scontents } { cmd-key-unknown } {#1} }
      { \msg_error:nnnn { scontents } { cmd-key-value-unknown } {#1} {#2} }
  }
%    \end{macrocode}
% \end{macro}
%
% \subsubsection{Undefined keys for \cs{foreachsc}}
%
% \begin{macro}{\@@_parse_foreach_keys:n,
%               \@@_parse_foreach_keys:nn}
%
%   We check the \meta{keys} passed to command §\foreachsc§ and process
%   it with \cs{@@_parse_foreach_keys:n}, if the \mymeta{key} is
%   \emph{unknown} we return an error message.
%
%    \begin{macrocode}
\cs_new_protected:Npn \@@_parse_foreach_keys:nn #1#2
  {
    \tl_if_blank:nTF {#2}
      { \msg_error:nnn { scontents } { for-key-unknown } {#1} }
      { \msg_error:nnnn { scontents } { for-key-value-unknown } {#1} {#2} }
  }
\cs_new_protected:Npn \@@_parse_foreach_keys:n #1
  { \exp_args:NV \@@_parse_foreach_keys:nn \l_keys_key_str {#1} }
%    \end{macrocode}
% \end{macro}
%
% \subsubsection{Undefined keys for \cs{typestored} and \cs{meaningsc}}
%
% \begin{macro}{\@@_parse_type_meaning_key:n,
%               \@@_parse_type_meaning_key:nn}
%
%   The commands §\typestored§ and §\meaningsc§ accept an optional
%   argument for setting the §width-tab§ to print the stored
%   contents. However their optional argument also contains the number
%   of the item to retrieve from the stored sequence. To avoid the
%   awkward §\typestored[]§\myoarg[type=tt]{options}§{...}§ syntax, we'll make the
%   commands have a single optional argument which is processed by
%   \pkg{l3keys}, and the unknown keys are brought here to
%   \cs{@@_parse_typemeaning_key:n} to process.
%
%   First we check if the \mymeta{key} is an integer using \cs{int_to_roman:n}.
%   If it is, we check that the value passed to the key is blank
%   (otherwise something odd as §1=1§ might have been used). If everything
%   is correct, then set the value of the integer which holds the \mymeta{index}.
%   Otherwise raise an error about an \emph{unknown} option.
%
%    \begin{macrocode}
\cs_new_protected:Npn \@@_parse_type_meaning_key:n #1
  { \exp_args:NV \@@_parse_type_meaning_key:nn \l_keys_key_str {#1} }
\cs_new_protected:Npn \@@_parse_type_meaning_key:nn #1#2
  {
    \tl_if_blank:nTF {#2}
      { \@@_parse_type_meaning_range:w #1 - \q_@@_mark - \s_@@_mark }
      { \msg_error:nnnn { scontents } { type-key-value-unknown } {#1} {#2} }
  }
\cs_new_protected:Npn \@@_parse_type_meaning_range:w #1 - #2 - #3 \s_@@_mark
  {
    \@@_range_parser:nnxn {#1} {#2}
      { \seq_count:c { g_@@_name_\l_@@_cur_seq_name_str _seq } }
      { \msg_error:nnn { scontents } { type-key-unknown } }
  }
\cs_new_protected:Npn \@@_range_parser:nnnn #1 #2 #3 #4
  {
    \exp_args:Nxx \@@_range_parser_aux:nnn
      { \str_if_eq:nnTF {#1} { end } {#3} { \exp_not:n {#1} } }
      { \str_if_eq:nnTF {#2} { end } {#3} { \exp_not:n {#2} } }
      { #4 }
  }
\cs_generate_variant:Nn \@@_range_parser:nnnn { nnxn }
\cs_new_protected:Npn \@@_range_parser_aux:nnn #1 #2 #3
  {
    \@@_tl_if_head_is_q_mark:nTF {#2}
      {
        \tl_if_empty:fTF { \int_to_roman:n { -0 #1 } }
          { \seq_put_right:Nx \l_@@_seq_item_seq { \int_eval:n {#1} } }
          { #3 {#1} }
      }
      {
        \bool_lazy_and:nnTF
            { \tl_if_empty_p:f { \int_to_roman:n { -0 #1 } } }
            { \tl_if_empty_p:f { \int_to_roman:n { -0 #2 } } }
          {
            \int_compare:nNnTF {#2} > {#1}
              { \int_step_inline:nnnn {#1} { 1 } {#2} }
              { \int_step_inline:nnnn {#1} { -1 } {#2} }
                { \seq_put_right:Nn \l_@@_seq_item_seq {##1} }
          }
          { #3 { #1-#2 } }
      }
  }
%    \end{macrocode}
% \end{macro}
%
% \subsection{Programming of the sequences}
%
%   The storage of the package is done using |seq| variables.  Here we
%   set up the macros that will manage the variables.
%
% \begin{macro}{\@@_append_contents:nn,\@@_append_contents:Vx}
%   The function \cs{@@_append_contents:nn} creates a |seq| variable if
%   one didn't exist and appends the contents in the argument to the
%   right of the sequence.
%
%    \begin{macrocode}
\cs_new_protected:Npn \@@_append_contents:nn #1#2
  {
    \tl_if_blank:nT {#1}
      { \msg_error:nn { scontents } { empty-store-cmd } }
    \seq_if_exist:cF { g_@@_name_#1_seq }
      { \seq_new:c { g_@@_name_#1_seq } }
    \seq_gput_right:cn { g_@@_name_#1_seq } {#2}
  }
\cs_generate_variant:Nn \@@_append_contents:nn { Vx }
%    \end{macrocode}
% \end{macro}
%
% \begin{macro}{\@@_getfrom_seq:nn,\@@_getfrom_seq:Nn,\@@_getfrom_seq:nnn}
%   The function \cs{@@_getfrom_seq:nn} retrieves the saved item from the
%   sequence.
%
%    \begin{macrocode}
\cs_new:Npn \@@_getfrom_seq:Nn #1#2
  {
    \seq_if_exist:cTF { g_@@_name_#2_seq }
      {
        \exp_args:Nf \@@_getfrom_seq:nNn
          { \seq_count:c { g_@@_name_#2_seq } } #1 {#2}
      }
      { \msg_expandable_error:nnn { scontents } { undefined-storage } {#2} }
  }
\cs_new:Npn \@@_getfrom_seq:nNn #1 #2 #3
  { \seq_map_tokens:Nn #2 { \@@_getfrom_seq_aux:nnn {#1} {#3} } }
\cs_new:Npn \@@_getfrom_seq_aux:nnn #1 #2 #3
  { \exp_args:Nnf \use:n { \@@_getfrom_seq:nnn {#1} } { \int_eval:n {#3} } {#2} }
\cs_new:Npn \@@_getfrom_seq:nn #1#2
  {
    \seq_if_exist:cTF { g_@@_name_#2_seq }
      {
        \exp_args:Nf \@@_getfrom_seq:nnn
          { \seq_count:c { g_@@_name_#2_seq } }
          {#1} {#2}
      }
      { \msg_expandable_error:nnn { scontents } { undefined-storage } {#2} }
  }
\cs_new:Npn \@@_getfrom_seq:nnn #1#2#3
  {
    \bool_lazy_or:nnTF
      { \int_compare_p:nNn {#2} = { 0 } }
      { \int_compare_p:nNn { \int_abs:n {#2} } > {#1} }
      { \msg_expandable_error:nnnnn { scontents } { index-out-of-range } {#2} {#3} {#1} }
      { \seq_item:cn { g_@@_name_#3_seq } {#2} }
  }
%    \end{macrocode}
% \end{macro}
%
% \begin{macro}{\@@_lastfrom_seq:n,\@@_lastfrom_seq:V}
%   The function \cs{@@_lastfrom_seq:n} retrieves the last saved item from the
%   sequence when \cs{l_@@_print_env_bool} or \cs{l_@@_print_cmd_bool}
%   is true.
%
%    \begin{macrocode}
\cs_new_protected:Npn \@@_lastfrom_seq:n #1
  {
    \tl_gset:Nx \g_@@_temp_tl { \seq_item:cn { g_@@_name_#1_seq } {-1} }
    \group_insert_after:N \@@_rescan_tokens:V
    \group_insert_after:N \g_@@_temp_tl
    \group_insert_after:N \tl_gclear:N
    \group_insert_after:N \g_@@_temp_tl
  }
\cs_generate_variant:Nn \@@_lastfrom_seq:n { V }
%    \end{macrocode}
% \end{macro}
%
% \begin{macro}{\@@_store_to_seq:NN}
%   The function \cs{@@_store_to_seq:NN} writes the recorded contents
%   in §#1§ to the log and stores it in §#2§.
%
%    \begin{macrocode}
\cs_new_protected:Npn \@@_store_to_seq:NN #1#2
  {
    \tl_log:N #1
    \@@_append_contents:Vx #2 { \exp_not:V #1 }
  }
%    \end{macrocode}
% \end{macro}
%
% \subsection{The command \cs{newenvsc} and environment \env{scontents}}
%
% In order to be able to define environments that behave similarly to
% \myenv{scontents}, we define a generic environment and make all other
% environment as wrappers around that one.
%
% \subsubsection{The command \cs{newenvsc}}
%
% \begin{macro}{\newenvsc,\@@_env_setting:nn,\@@_env_define:nnn}
%   The internal function \cs{@@_env_setting:nn} defines two
%   functions \cs[no-index]{@@_\#1_env_begin:} and
%   \cs[no-index]{@@_\#1_env_end:}, which set the current environment's
%   name in §#1§ and \cs{l_@@_env_name_tl} and default properties in §#2§
%   then call \cs{@@_setup_verb_processor:}, the generic
%   \cs{@@_env_generic_begin:} and \cs{@@_env_generic_end:}.
%
%   Finally the function \cs{@@_env_define:nnn} will create the environments.
%    \begin{macrocode}
\cs_new_protected:Npn \@@_env_setting:nn #1 #2
  {
    \cs_new_protected:cpn { @@_#1_env_begin: }
      {
        \tl_set:Nn \l_@@_env_name_tl {#1}
        \keys_set:nn { scontents } {#2}
        \@@_setup_verb_processor:
        \@@_env_generic_begin:
      }
    \cs_new_protected:cpn { @@_#1_env_end: }
      { \@@_env_generic_end: }
    \exp_args:Nooo % http://nooooooooooooooo.com :) jeje
    \@@_env_define:nnn { \tl_to_str:n {#1} }
      { \cs:w @@_#1_env_begin: \cs_end: }
      { \cs:w @@_#1_env_end: \cs_end: }
  }
%</core>
%<*loader>
\NewDocumentCommand \newenvsc { m O{} }
  {
%<latex|plain>    \cs_if_exist:cTF { #1 }
%<context>    \cs_if_exist:cTF { start #1 }
      { \msg_error:nnn { scontents } { env-already-defined } {#1} }
      { \@@_env_setting:nn {#1} {#2} }
  }
\cs_new_protected:Npn \@@_env_define:nnn #1 #2 #3
  {
%<latex|plain>    \NewDocumentEnvironment {#1} { }
%<context>    \cs_new_protected:cpn { start #1 }
      {
%<!latex>        \group_begin:
          #2
      }
%<context>    \cs_new_protected:cpn { stop #1 }
      {
          #3
%<!latex>        \group_end:
      }
  }
%</loader>
%<*core>
%    \end{macrocode}
% \end{macro}
%
% \subsubsection{Generic definition of the environment}
%
% \begin{macro}{\@@_env_generic_begin:,\@@_env_generic_end:}
%
%   Now we define the generic environment functions
%   \cs{@@_env_generic_begin:} and \cs{@@_env_generic_end:}.
%    \begin{macrocode}
\cs_new_protected:Npn \@@_env_generic_begin:
  {
    \char_set_catcode_active:N \^^M
    \@@_start_environment:w
  }
\cs_new_protected:Npn \@@_env_generic_end:
  {
    \@@_stop_environment:
    \@@_finish_storing:NNN \l_@@_macro_tmp_tl
    \l_@@_name_seq_env_tl \l_@@_print_env_bool
  }
%    \end{macrocode}
% \end{macro}
%
% \subsubsection{Definition of the environment \env{scontents}}
%
% \begin{macro}{
%     scontents,
%     \scontents,
%     \endscontents,
%     \startscontents,
%     \stopscontents,
%   }
%   Finally defining the \myenv{scontents} environment should be easy :)
%    \begin{macrocode}
%</core>
%<loader>\newenvsc{scontents}
%<*core>
%    \end{macrocode}
% \end{macro}
%
% \subsubsection{key val for environment}
%
% \begin{macro}{\@@_grab_optional:n,\@@_grab_optional:w}
%   The macro \cs{@@_grab_optional:w} is called from the
%   \env{scontents} environment with the tokens following the
%   §\begin{scontents}§ when the next character is a §[§. This function
%   is defined using \mypkg{xparse} to exploit its delimited argument
%   processor.
%
%   The function is called from a context where §^^M§ is active, so
%   \cs{@@_normalise_line_ends:N} is used to replace active §^^M§
%   characters by spaces.
%    \begin{macrocode}
%</core>
%<*loader>
\NewDocumentCommand \@@_grab_optional:w { r[] }
  { \@@_grab_optional:n {#1} }
%</loader>
%<*core>
\cs_new_protected:Npn \@@_grab_optional:n #1
  {
    \tl_if_novalue:nF {#1}
      {
        \tl_set:Nn \l_@@_temp_tl {#1}
        \@@_normalise_line_ends:N \l_@@_temp_tl
        \keys_set:nV { scontents / scontents } \l_@@_temp_tl
      }
    \@@_start_after_option:w
  }
%    \end{macrocode}
% \end{macro}
%
% \subsubsection{The environment itself}
% \label{the:environment:itself}
%
% \begin{macro}{\@@_start_environment:w,\@@_start_after_option:w,
%               \@@_check_line_process:xn,\@@_stop_environment:}
%   Here we make §^^I§, §^^L§ and §^^M§ active characters so that the end of
%   line can be \enquote{seen} to be used as a delimiter, and \hologo{TeX} doesn't try to
%   eliminate space-like characters.
%
%   First we check if the immediate next token after §\begin{scontents}§
%   is a §[§. If it is, then \cs{@@_grab_optional:w} is called
%   to do the heavy lifting. \cs{@@_grab_optional:w} processes the optional
%   argument and calls \cs{@@_start_after_option:w}.
%
%   The function \cs{@@_start_after_option:w} also checks for trailing
%   tokens after the optional argument and issues an error if any.
%
%   In all cases, the function \cs{@@_check_line_process:xn} checks that
%   everything past §\begin{scontents}§ is empty and then process
%   the environment.
%
%   The function \cs{@@_check_line_process:xn} calls the function
%   \cs{@@_file_tl_write_start:V} which will then read the contents
%   of the environment and optionally store them in a token list or write to
%   an external file.
%
%   When that's done, the function \cs{@@_file_write_stop:N} does the
%   cleanup. This part of the code is inspired and adapted from the code of
%   the package \pkg{xsimverb} by Clemens Niederberger.
%
%    \begin{macrocode}
\group_begin:
  \char_set_catcode_active:N \^^I
  \char_set_catcode_active:N \^^L
  \char_set_catcode_active:N \^^M
  \cs_new_protected:Npn \@@_normalise_line_ends:N #1
    { \tl_replace_all:Nnn #1 { ^^M } { ~ } }
  \cs_new_protected:Npn \@@_start_environment:w #1 ^^M
    {
      \tl_if_head_is_N_type:nTF {#1}
        {
          \str_if_eq:eeTF { \tl_head:n {#1} } { [ }
            { \@@_grab_optional:w #1 ^^M }
            { \@@_check_line_process:xn { } {#1} }
        }
        { \@@_check_line_process:xn { } {#1} }
    }
  \cs_new_protected:Npn \@@_start_after_option:w #1 ^^M
    { \@@_check_line_process:xn { [...] } {#1} }
  \cs_new_protected:Npn \@@_check_line_process:xn #1 #2
    {
      \tl_if_blank:nF {#2}
        {
          \msg_error:nnxn { scontents } { junk-after-begin }
            { after~\c_backslash_str begin { \l_@@_env_name_tl } #1 } {#2}
        }
      \@@_make_control_chars_active:
      \@@_file_tl_write_start:V \l_@@_fname_out_tl
    }
  \cs_new_protected:Npn \@@_stop_environment:
    {
      \@@_file_write_stop:N \l_@@_macro_tmp_tl
      \bool_lazy_and:nnT
        { \l_@@_storing_bool }
        { \tl_if_empty_p:N \l_@@_macro_tmp_tl }
        {
          \msg_warning:nnx { scontents } { empty-environment } { \l_@@_env_name_tl }
        }
    }
%    \end{macrocode}
% \end{macro}
%
% \begin{macro}{
%     \@@_file_tl_write_start:n,
%     \@@_file_tl_write_start:V,
%     \@@_verb_processor_iterate:w,
%     \@@_verb_processor_iterate:nnn,
%     \@@_setup_verb_processor:,
%     \@@_file_write_stop:N,
%     \@@_remove_leading_nl:n,
%     \@@_remove_leading_nl:w,
%   }
%   This is the main macro to collect the contents of a verbatim environment.
%   The macro starts a group, opens the \mymeta{output file}, if necessary,
%   sets verbatim catcodes, and then issues §^^M§ (set equal to \cs{@@_ret:w})
%   to read the environment line by line until reaching its end. The output
%   token list will be appended with an active §^^J§ character and the line
%   just read, and this line is written to the output file, if any. At the end
%   of the environment the \mymeta{output file} is closed (if it was open),
%   and the output token list is smuggled out of the verbatim group. A leading
%   §^^J§ is removed from the token list using \cs{@@_remove_leading_nl:n}
%  (which expects an active §^^J§ token at the head of the token list; a low
%  level \hologo{TeX} error is raised otherwise).
%
%    \begin{macrocode}
  \cs_new_protected:Npn \@@_file_tl_write_start:n #1
    {
      \group_begin:
        \@@_file_if_writable:nTF {#1}
          {
            \bool_set_true:N \l_@@_writable_bool
            \iow_open:Nn \l_@@_file_iow {#1}
          }
          { \bool_set_false:N \l_@@_writable_bool }
        \tl_clear:N \l_@@_every_line_env_tl
        \seq_map_function:NN \l_char_special_seq \char_set_catcode_other:N
        \int_step_function:nnnN { 128 } { 1 } { 255 } \char_set_catcode_letter:n
        \cs_set_protected:Npx \@@_ret:w ##1 ^^M
          {
            \exp_not:N \@@_verb_processor_iterate:w
            ##1 \c_@@_end_env_tl
                \c_@@_end_env_tl
                \exp_not:N \q_@@_stop
          }
        \@@_make_control_chars_active:
        \@@_ret:w
    }
  \cs_new:Npn \@@_setup_verb_processor:
    {
      \use:x
        {
          \cs_set:Npn \exp_not:N \@@_verb_processor_iterate:w
            ####1 \c_@@_end_env_tl
            ####2 \c_@@_end_env_tl
            ####3 \exp_not:N \q_@@_stop
        }   { \@@_verb_processor_iterate:nnn {##1} {##2} {##3} }
    }
  \cs_new:Npn \@@_verb_processor_iterate:nnn #1 #2 #3
    {
      \tl_if_blank:nTF {#3}
        {
          \@@_analyse_nesting:n {#1}
          \@@_verb_processor_output:n {#1}
        }
        {
          \@@_if_nested:TF
            {
              \@@_nesting_decr:
              \@@_verb_processor_output:x
                { \exp_not:n {#1} \c_@@_end_env_tl \exp_not:n {#2} }
            }
            {
              \tl_if_blank:nF {#1}
                { \@@_verb_processor_output:n {#1} }
              \cs_set_protected:Npx \@@_ret:w
                {
                  \@@_env_end_function:
                  \bool_lazy_or:nnF
                    { \tl_if_blank_p:n {#2} }
                    { \str_if_eq_p:ee {#2} { \c_percent_str } }
                    {
                      \str_if_eq:VnF \c_@@_hidden_space_str {#2}
                        {
                          \msg_warning:nnnn { scontents } { rescanning-text }
                            {#2} { \tl_use:N \l_@@_env_name_tl }
                        }
                      \@@_rescan_tokens:n {#2}
                    }
                }
              \char_set_active_eq:NN ^^M \@@_ret:w
            }
        }
      ^^M
    }
  \cs_new:Npn \@@_env_end_function:
    {
      \@@_format_case:nnn
        { \exp_not:N \end { \if_false: } \fi: }
        { \exp_after:wN \exp_not:N \cs:w end }
        { \exp_after:wN \exp_not:N \cs:w stop }
      \tl_use:N \l_@@_env_name_tl
      \@@_format_case:nnn
        { \if_false: { \fi: } }
        { \cs_end: }
        { \cs_end: }
    }
  \cs_new_protected:Npn \@@_file_write_stop:N #1
    {
      \bool_if:NT \l_@@_writable_bool
        { \iow_close:N \l_@@_file_iow }
      \use:x
        {
          \group_end:
          \bool_if:NT \l_@@_storing_bool
            {
              \tl_set:Nn \exp_not:N #1
                { \exp_args:NV \@@_remove_leading_nl:n \l_@@_every_line_env_tl }
            }
        }
    }
  \cs_new:Npn \@@_remove_leading_nl:n #1
    {
      \tl_if_head_is_N_type:nTF {#1}
        {
          \exp_args:Nf
            \@@_remove_leading_nl:nn
              { \tl_head:n {#1} } {#1}
        }
        { \exp_not:n {#1} }
    }
  \cs_new:Npn \@@_remove_leading_nl:nn #1 #2
    {
      \token_if_eq_meaning:NNTF ^^J #1
        { \exp_not:o { \@@_remove_leading_nl:w #2 } }
        { \exp_not:n {#2} }
    }
  \cs_new:Npn \@@_remove_leading_nl:w ^^J { }
%    \end{macrocode}
% \end{macro}
%
% \begin{macro}{\@@_verb_processor_output:n,\@@_verb_processor_output:x}
%   The function \cs{@@_verb_processor_output:n} does the output of
%   each line read, to a token list and to a file, depending on the
%   booleans \cs{l_@@_writing_bool} and \cs{l_@@_storing_bool}.
%    \begin{macrocode}
  \cs_new_protected:Npn \@@_verb_processor_output:n #1
    {
      \bool_if:NT \l_@@_writable_bool
        { \iow_now:Nn \l_@@_file_iow {#1} }
      \bool_if:NT \l_@@_storing_bool
        { \tl_put_right:Nn \l_@@_every_line_env_tl { ^^J #1 } }
    }
\group_end:
\cs_generate_variant:Nn \@@_verb_processor_output:n { x }
\cs_generate_variant:Nn \@@_file_tl_write_start:n { V }
%    \end{macrocode}
% \end{macro}
%
% \begin{macro}{
%     \@@_analyse_nesting:n,
%     \@@_analyse_nesting:w,
%     \@@_nesting_decr:,
%     \@@_use_none_delimit_by_q_stop:w,
%   }
% \begin{macro}[TF]{\@@_if_nested:}
%   \cs{@@_analyse_nesting:n} scans nested §\begin{scontents}§
%   and steps a \cs{l_@@_env_nesting_int} counter. The \cs{@@_if_nested:}
%   conditional tests if we're in a nested environment, and
%   \cs{@@_nesting_decr:} reduces the nesting level, if an §\end{scontents}§
%   is found.
%
%   Multiple §\end{scontents}§ in the same line are not supported\dots
%    \begin{macrocode}
\cs_new_protected:Npn \@@_analyse_nesting:n #1
  {
    \int_zero:N \l_@@_tmpa_int
    \@@_analyse_nesting_format:n {#1}
    \int_compare:nNnT { \l_@@_tmpa_int } > { 1 }
      { \msg_warning:nn { scontents } { multiple-begin } }
  }
\cs_new_protected:Npn \@@_nesting_incr:
  {
    \int_incr:N \l_@@_env_nesting_int
    \int_incr:N \l_@@_tmpa_int
  }
\cs_new_protected:Npn \@@_nesting_decr:
  { \int_decr:N \l_@@_env_nesting_int }
\prg_new_protected_conditional:Npnn \@@_if_nested: { TF }
  {
    \int_compare:nNnTF { \l_@@_env_nesting_int } > { \c_zero_int }
      { \prg_return_true: }
      { \prg_return_false: }
  }
\cs_new:Npn \@@_use_none_delimit_by_q_stop:w #1 \q_@@_stop { }
%    \end{macrocode}
%
%   In \LaTeX{}, environments start with §\begin{<env>}§, so checking if a
%   string contains §\begin{scontents}§ is straightforward. Since no §}§ can
%   appear inside §<env>§, then just a macro delimited by §}§ is enough.
%    \begin{macrocode}
\use:x
  {
    \cs_new_protected:Npn \exp_not:N \@@_analyse_nesting_latex:w ##1
      \c_backslash_str begin \c_left_brace_str ##2 \c_right_brace_str
  }   {
        \@@_tl_if_head_is_q_mark:nTF {#2}
          { \@@_use_none_delimit_by_q_stop:w }
          {
            \str_if_eq:VnT \l_@@_env_name_tl {#2}
              { \@@_nesting_incr: }
            \@@_analyse_nesting_latex:w
          }
      }
\cs_new_protected:Npx \@@_analyse_nesting_latex:n #1
  {
    \@@_analyse_nesting_latex:w #1
      \c_backslash_str begin
        \c_left_brace_str \exp_not:N \q_@@_mark \c_right_brace_str
    \exp_not:N \q_@@_stop
  }
%    \end{macrocode}
%
%   In other formats, however, we don't have an \enquote{end anchor} to delimit the
%   environment name, so a delimited macro won't help. We have to search for
%   the entire environment command (usually §\scontents§ and §\startscontents§).
%    \begin{macrocode}
\cs_new_protected:Npn \@@_analyse_nesting_generic_process:nn #1 #2
  {
    \tl_if_head_is_N_type:nTF {#2}
      {
        \@@_tl_if_head_is_q_mark:nF {#2}
          {
            \@@_nesting_incr:
            \@@_analyse_nesting_generic:w #2 \q_@@_stop
          }
      }
      { \@@_analyse_nesting_generic:w #2 \q_@@_stop }
  }
\cs_new_protected:Npn \@@_analyse_nesting_generic:nn #1 #2
  {
    \@@_define_generic_nesting_function:n {#1}
    \use:x
      {
        \exp_not:N \@@_analyse_nesting_generic:w #2
          \c_backslash_str #1 \tl_use:N \l_@@_env_name_tl
            \exp_not:N \q_@@_mark \exp_not:N \q_@@_stop
      }
  }
\cs_new_protected:Npn \@@_define_generic_nesting_function:n #1
  {
    \use:x
      {
        \cs_set_protected:Npn \exp_not:N \@@_analyse_nesting_generic:w ####1
          \c_backslash_str #1 \tl_use:N \l_@@_env_name_tl
            ####2 \exp_not:N \q_@@_stop
      }   { \@@_analyse_nesting_generic_process:nn {##1} {##2} }
  }
%</core>
%<*loader>
%<latex>\cs_new_eq:NN \@@_analyse_nesting_format:n
%<latex>  \@@_analyse_nesting_latex:n
%<!latex>\cs_new_protected:Npn \@@_analyse_nesting_format:n
%<plain>  { \@@_analyse_nesting_generic:nn { } }
%<context>  { \@@_analyse_nesting_generic:nn { start } }
%</loader>
%<*core>
%    \end{macrocode}
% \end{macro}
% \end{macro}
%
% \subsubsection{Recording of the content in the sequence}
%
% \begin{macro}{\@@_finish_storing:NNN}
%   Finishes the environment by optionally calling \cs{@@_store_to_seq:}
%   and then clearing the temporary token list.
%
%    \begin{macrocode}
\cs_new_protected:Npn \@@_finish_storing:NNN #1 #2 #3
  {
    \bool_if:NT \l_@@_storing_bool
      {
        \bool_if:NF \l_@@_forced_eol_bool
          { \tl_put_right:Nx #1 { \c_@@_hidden_space_str } }
        \@@_store_to_seq:NN #1 #2
        \bool_if:NT #3 { \@@_lastfrom_seq:V #2 }
      }
  }
%</core>
%    \end{macrocode}
% \end{macro}
%
% \subsection{The environment \env{verbatimsc}}
%
% \begin{macro}[int]{\verbatimsc,\endverbatimsc,\@@_verbatimsc_aux:,
%                    \@@_vobeyspaces:,\@@_xverb:,\@@_nolig_list:,\@@_xobeysp:}
%   In \hologo{plainTeX} we emulate \LaTeX's \env{verbatim} environment.
%
%    \begin{macrocode}
%<*plain>
\cs_new_protected:Npn \verbatimsc
  {
    \group_begin:
      \@@_verbatimsc_aux: \frenchspacing \@@_vobeyspaces:
      \@@_xverb:
  }
\cs_new_protected:Npn \endverbatimsc
  { \group_end: }
\cs_new_protected:Npn \@@_verbatimsc_aux:
  {
    \skip_vertical:N \parskip
    \dim_zero:N \parindent
    \skip_set:Nn \parfillskip { 0pt plus 1fil }
    \skip_set:Nn \parskip { 0pt plus0pt minus0pt }
    \tex_par:D
    \bool_set_false:N \l_@@_temp_bool
    \cs_set:Npn \par
      {
        \bool_if:NTF \l_@@_temp_bool
          {
            \mode_leave_vertical:
            \null
            \tex_par:D
            \penalty \interlinepenalty
          }
          {
            \bool_set_true:N \l_@@_temp_bool
            \mode_if_horizontal:T
              { \tex_par:D \penalty \interlinepenalty }
          }
      }
    \cs_set_eq:NN \do \char_set_catcode_other:N
    \dospecials \obeylines
    \tl_use:N \l_@@_verb_font_tl
    \cs_set_eq:NN \do \@@_do_noligs:N
    \@@_nolig_list:
    \tex_everypar:D \exp_after:wN
      { \tex_the:D \tex_everypar:D \tex_unpenalty:D }
  }
\cs_new_protected:Npn \@@_nolig_list:
  { \do\`\do\<\do\>\do\,\do\'\do\- }
\cs_new_protected:Npn \@@_vobeyspaces:
  { \@@_set_active_eq:NN \  \@@_xobeysp: }
\cs_new_protected:Npn \@@_xobeysp:
  { \mode_leave_vertical: \nobreak \ }
%</plain>
%    \end{macrocode}
% \end{macro}
%
% \begin{macro}[int]{\dospecials}
%   \pkg{xparse} also requires \LaTeX's §\dospecials§. In case it doesn't
%   exist (at the time \pkg{scontents} is loaded) we define §\dospecials§
%   to use the \cs{l_char_special_seq}.
%    \begin{macrocode}
%<*!latex>
\cs_if_exist:NF \dospecials
  {
    \cs_new:Npn \dospecials
      { \seq_map_function:NN \l_char_special_seq \do }
  }
%</!latex>
%    \end{macrocode}
% \end{macro}
%
%
% \subsection{The command \cs{Scontents}}
%
% User command to \mymeta{stored content}, adapted from code by Ulrich
% Diez in \href{https://tex.stackexchange.com/a/472044/7832}{Stringify
% input - \textbackslash string on token list}
% and code by user siracusa in
% \href{https://tex.stackexchange.com/a/500281/7832}{Convert a macro from
% Latex2e to expl3}
%
% \begin{macro}{\@@_bsphack:,\@@_esphack:}
%    We emulate §\@bsphack§ and §\@esphack§ for \hologo{plainTeX}.
%    This is necessary to prevent unwanted spaces when the §print-cmd§
%    key is false.
%    \begin{macrocode}
%<*core>
\cs_new_protected:Npn \@@_bsphack:
  {
    \scan_stop:
    \mode_if_horizontal:T
      {
        \skip_set_eq:NN \l_@@_save_skip \tex_lastskip:D
        \int_set_eq:NN \l_@@_save_sf_int \tex_spacefactor:D
      }
  }
\cs_new_protected:Npn \@@_esphack:
  {
    \scan_stop:
    \mode_if_horizontal:T
      {
        \int_set_eq:NN \tex_spacefactor:D \l_@@_save_sf_int
        \dim_compare:nNnT { \l_@@_save_skip } > { \c_zero_skip }
          {
            \skip_if_eq:nnT { \tex_lastskip:D } { \c_zero_skip }
              {
                \nobreak
                \skip_horizontal:n { \c_zero_skip }
              }
            \tex_ignorespaces:D
          }
      }
  }
%</core>
%<*latex>
\cs_gset_eq:NN \@@_bsphack: \@bsphack
\cs_gset_eq:NN \@@_esphack: \@esphack
%</latex>
%    \end{macrocode}
% \end{macro}
%
% \begin{macro}{
%     \Scontents,
%     \@@_Scontents_internal:nn,
%     \@@_norm_arg:n,
%     \@@_verb_arg:w,
%   }
%   The §\Scontents§ command starts by parsing an optional argument to
%   the function \cs{@@_Scontents_internal:nn} then delegates to
%   \cs{@@_verb_arg:w} or \cs{@@_norm_arg:n} depending whether a star
%   (§*§) argument is present.
%
%    \begin{macrocode}
%<*loader>
\NewDocumentCommand \Scontents { !s !O{} }
  { \@@_Scontents_internal:nn {#1} {#2} }
%</loader>
%<*core>
\cs_new_protected:Npn \@@_Scontents_internal:nn #1 #2
  {
    \@@_bsphack:
    \group_begin:
      \tl_if_novalue:nF {#2}
        { \keys_set:nn { scontents / Scontents } {#2} }
      \char_set_catcode_active:n { 9 }
      \bool_if:NTF #1
        { \@@_verb_arg:w }
        { \@@_norm_arg:n }
  }
%    \end{macrocode}
%
%   The function \cs{@@_norm_arg:n} grabs a normal argument, adds it to
%   the |seq| variable and optionally prints it.
%
%    \begin{macrocode}
\cs_new_protected:Npn \@@_norm_arg:n #1
  {
      \tl_set:Nn \l_@@_temp_tl {#1}
      \@@_Scontents_finish:
  }
%    \end{macrocode}
%
%   The function \cs{@@_verb_arg:w} grabs a verbatim argument using
%   \pkg{xparse}'s |+v| argument parser.
%
%    \begin{macrocode}
%</core>
%<*loader>
\NewDocumentCommand \@@_verb_arg:w { +v }
  { \@@_verb_arg_internal:n {#1} }
%</loader>
%<*core>
%    \end{macrocode}
% \end{macro}
%
% \begin{macro}{
%     \@@_verb_arg_internal:n,
%     \@@_Scontents_finish:,
%     \@@_file_write_cmd:nn,
%     \@@_file_write_cmd:VV,
%   }
%
%   The function \cs{@@_verb_arg_internal:n} replace all §\^^M§ and 
%   §\obeyedline§ added by §+v§ argument added in \hologo{LaTeX} release 2024-06-01 by
%   §\^^J§ then adds it to the |seq| variable.
%
%    \begin{macrocode}
\cs_new_protected:Npn \@@_verb_arg_internal:n #1
  {
      \tl_set:Nn \l_@@_temp_tl {#1}
      \cs_if_exist:NT \obeyedline
        {
          \tl_replace_all:Nxx \l_@@_temp_tl { \obeyedline } { \iow_char:N \^^M }
        }
      \tl_replace_all:Nxx \l_@@_temp_tl { \iow_char:N \^^M } { \iow_char:N \^^J }
      \@@_Scontents_finish:
  }
\cs_new_protected:Npn \@@_Scontents_finish:
  {
      \@@_file_write_cmd:VV \l_@@_fname_out_tl \l_@@_temp_tl
      \@@_finish_storing:NNN
        \l_@@_temp_tl
        \l_@@_name_seq_cmd_tl
        \l_@@_print_cmd_bool
      \use:x
        {
    \group_end:
    \bool_if:NF \l_@@_print_cmd_bool { \@@_esphack: }
        }
  }
\cs_new_protected:Npn \@@_file_write_cmd:nn #1#2
  {
    \@@_file_if_writable:nT {#1}
      {
        \iow_open:Nn \l_@@_file_iow {#1}
        \iow_now:Nn  \l_@@_file_iow {#2}
        \iow_close:N \l_@@_file_iow
      }
  }
\cs_generate_variant:Nn \@@_file_write_cmd:nn { VV }
\prg_new_protected_conditional:Npnn \@@_file_if_writable:n #1 { T, F, TF }
  {
    \bool_if:NTF \l_@@_writing_bool
      {
        \file_if_exist:nTF {#1}
          {
            \bool_if:NTF \l_@@_overwrite_bool
              {
                \msg_warning:nnx { scontents } { overwrite-file } {#1}
                \prg_return_true:
              }
              {
                \msg_warning:nnx { scontents } { not-writing } {#1}
                \prg_return_false:
              }
          }
          {
            \msg_warning:nnx { scontents } { writing-file } {#1}
            \prg_return_true:
          }
      }
      { \prg_return_false: }
  }
%    \end{macrocode}
% \end{macro}
%
% \subsection{The command \cs{getstored}}
%
% \begin{macro}{\getstored,\@@_getstored_internal:nn}
%   User command §\getstored§ to extract \mymeta{stored content} in |seq|
%   (robust).
%
%    \begin{macrocode}
%</core>
%<*loader>
\NewDocumentCommand \getstored { O{-1} m }
  { \@@_getstored_internal:nn {#1} {#2} }
%</loader>
%<*core>
\cs_new_protected:Npn \@@_getstored_internal:nn #1 #2
  {
    \group_begin:
      \int_set:Nn \tex_newlinechar:D { `\^^J }
      \@@_rescan_tokens:x
        {
    \endgroup % This assumes \catcode`\\=0... Things might go off otherwise.
    \@@_getfrom_seq:nn {#1} {#2}
        }
  }
%    \end{macrocode}
% \end{macro}
%
% \subsection{The command \cs{foreachsc}}
%
% \begin{macro}{\foreachsc,\@@_foreachsc_internal:nn,\@@_foreach_add_body:n}
%   User command §\foreachsc§ to loop over \mymeta{stored content} in |seq|.
%
%    \begin{macrocode}
%</core>
%<*loader>
\NewDocumentCommand \foreachsc { o m }
  { \@@_foreachsc_internal:nn {#1} {#2} }
%</loader>
%<*core>
\cs_new_protected:Npn \@@_foreachsc_internal:nn #1 #2
  {
    \group_begin:
      \tl_if_novalue:nF {#1} { \keys_set:nn { scontents / foreachsc } {#1} }
      \tl_set:Nn \l_@@_foreach_name_seq_tl {#2}
      \seq_clear:N \l_@@_foreach_print_seq
      \bool_if:NF \l_@@_foreach_stop_bool
        {
          \int_set:Nn \l_@@_foreach_stop_int
            { \seq_count:c { g_@@_name_#2_seq } }
        }
      \int_step_function:nnnN
        { \l_@@_foreach_start_int }
        { \l_@@_foreach_step_int }
        { \l_@@_foreach_stop_int }
        \@@_foreach_add_body:n
      \tl_gset:Nx \g_@@_temp_tl
        {
          \exp_args:NNV \seq_use:Nn
            \l_@@_foreach_print_seq \l_@@_foreach_sep_tl
        }
    \group_end:
    \exp_after:wN \tl_gclear:N
    \exp_after:wN \g_@@_temp_tl
      \g_@@_temp_tl
  }
\cs_new_protected:Npn \@@_foreach_add_body:n #1
  {
    \seq_put_right:Nx \l_@@_foreach_print_seq
      {
        \bool_if:NT \l_@@_foreach_before_bool
          { \exp_not:V \l_@@_foreach_before_tl }
        \bool_if:NTF \l_@@_foreach_wrapper_bool
          { \@@_foreach_wrapper:n }
          { \use:n }
            { \getstored [#1] { \tl_use:N \l_@@_foreach_name_seq_tl } }
        \bool_if:NT \l_@@_foreach_after_bool
          { \exp_not:V \l_@@_foreach_after_tl }
      }
  }
%    \end{macrocode}
% \end{macro}
%
% \subsection{The command \cs{typestored}}
%
% \begin{macro}{\typestored,\@@_typestored_internal:nn,\@@_verb_print:N,
%               \@@_xverb:w}
%   The §\typestored§ commands fetches a buffer from memory, prints it
%   to the log file, and then calls \cs{@@_verb_print:N}.
%
%    \begin{macrocode}
%</core>
%<*loader>
\NewDocumentCommand \typestored { o m }
  { \@@_typestored_internal:nn {#1} {#2} }
%</loader>
%<*core>
\cs_new_protected:Npn \@@_typestored_internal:nn #1 #2
  {
    \@@_bsphack:
    \group_begin:
      \seq_clear:N \l_@@_seq_item_seq
      \str_set:Nx \l_@@_cur_seq_name_str {#2}
      \tl_if_novalue:nF {#1} { \keys_set:nn { scontents / typemeaning } {#1} }
      \seq_if_empty:NT \l_@@_seq_item_seq
        { \seq_set_from_clist:Nn \l_@@_seq_item_seq { 1 } }
      \tl_set:Nx \l_@@_temp_tl
        { \@@_getfrom_seq:Nn \l_@@_seq_item_seq {#2} }
      \@@_remove_trailing_eol:N \l_@@_temp_tl
      \tl_replace_all:Nxn \l_@@_temp_tl \c_@@_hidden_space_str { ^^J }
      \tl_log:N \l_@@_temp_tl
      \tl_if_empty:NF \l_@@_temp_tl
        {
          \bool_if:NT \l_@@_print_aux_bool
            {
              \@@_verb_print:N \l_@@_temp_tl
            }
        }
      \@@_file_write_cmd:VV \l_@@_fname_out_tl \l_@@_temp_tl
      \use:x
        {
    \group_end:
    \bool_if:NF \l_@@_print_aux_bool { \@@_esphack: }
        }
  }
%    \end{macrocode}
%
%   The \cs{@@_verb_print:N} macro is defined with active carriage return
%   (\textsc{ascii} 13) characters to mimic an actual verbatim environment
%   \enquote{on the loose}. The contents of the environment are placed in a
%   §verbatimsc§ environment and rescanned using \cs{@@_rescan_tokens:x}.
%
%    \begin{macrocode}
\group_begin:
  \char_set_catcode_active:N \^^M
  \cs_new_protected:Npn \@@_verb_print:N #1
    {
      \tl_if_blank:VT #1
        { \msg_error:nnn { scontents } { empty-variable } {#1} }
      \cs_set_eq:NN \@@_verb_print_EOL: ^^M
      \cs_set_eq:NN ^^M \scan_stop:
      \cs_set_eq:cN { do@noligs } \@@_do_noligs:N
      \int_set:Nn \tex_newlinechar:D { `\^^J }
      \@@_rescan_tokens:x
        {
          \@@_format_case:nnn
            { \exp_not:N \begin{verbatimsc} } % LaTeX
            { \verbatimsc } % Plain/Generic
            { \startverbatimsc } % ConTeXt
            ^^M
          \exp_not:V #1 ^^M
          \g_@@_end_verbatimsc_tl
        }
      \cs_set_eq:NN ^^M \@@_verb_print_EOL:
    }
\group_end:
\cs_new_protected:Npn \@@_xverb:
  {
    \char_set_catcode_active:n { 9 }
    \char_set_active_eq:nN { 9 } \@@_tabs_to_spaces:
    \@@_xverb:w
  }
%</core>
%    \end{macrocode}
% \end{macro}
%
% \subsection{The command \cs{mergesc}}
%
% \begin{macro}{\mergesc}
%   The §\mergesc§ command parses a list given as argument, and just assembles
%   it as a temporary internal sequence, then passes it to the requested command.
%    \begin{macrocode}
%<*loader>
\NewDocumentCommand \mergesc { o m }
  { \@@_mergesc_internal:nn {#1} {#2} }
%</loader>
%<*core>
\keys_define:nn { scontents / mergesc }
  {
    , typestored .code:n =
        { \cs_set_eq:NN \@@_mergesc_output_cmd:nn \@@_typestored_internal:nn }
    , meaningsc .code:n =
        { \cs_set_eq:NN \@@_mergesc_output_cmd:nn \@@_meaningsc_internal:nn }
  }
\cs_new_protected:Npn \@@_mergesc_output_cmd:nn #1 #2
  { \msg_error:nn { scontents } { mergesc-missing-cmd } }
\msg_new:nnn { scontents } { mergesc-missing-cmd }
  { Missing~output~command~for~\iow_char:N\\mergesc~\msg_line_context:. }
\cs_new_protected:Npn \@@_mergesc_internal:nn #1 #2
  {
    \group_begin:
      \tl_clear:N \l_@@_keys_tl
      \tl_if_novalue:nF {#1}
        {
          % Add print-cmd here :D
          \keys_define:nn { scontents / typemeaning }
            {
              print-cmd .bool_set:N = \l_@@_print_aux_bool,
              print-cmd .initial:n = false,
              print-cmd .default:n = true,
            }
          \keys_set_known:nnN { scontents / mergesc } {#1} \l_@@_keys_tl
         }
      \seq_gclear:c { g_@@_name_sc!internal_seq }
      \@@_mergesc_parse_list:n {#2}
      \exp_args:Nx \@@_mergesc_output_cmd:nn
        { 1-end, \exp_not:V \l_@@_keys_tl } { sc!internal }
    \group_end:
  }
%    \end{macrocode}
%
%    \begin{macrocode}
\cs_new_protected:Npn \@@_mergesc_parse_list:n #1
  {
    \clist_map_inline:nn {#1} { \@@_parse_mergesc:nw ##1 \s_@@_stop }
    \seq_gpop_right:cN { g_@@_name_sc!internal_seq } \l_@@_temp_tl
    \@@_remove_trailing_eol:N \l_@@_temp_tl
    \seq_gput_right:cV { g_@@_name_sc!internal_seq } \l_@@_temp_tl
  }
\cs_new_protected:Npx \@@_remove_trailing_eol:N #1
  {
    \exp_not:N \exp_after:wN \exp_not:N \@@_remove_trailing_eol:w
      #1 \s_@@_stop \c_@@_hidden_space_str \s_@@_stop \s_@@_mark #1
  }
\use:e
  {
    \cs_new_protected:Npn \exp_not:N \@@_remove_trailing_eol:w #1
        \c_@@_hidden_space_str \s_@@_stop #2 \s_@@_mark #3
  }   {
        \tl_set:Nx #3
          {
            \tl_if_empty:nTF {#2}
              { \exp_not:o { \@@_use_delimit_by_s_stop:nw #1 } }
              { \exp_not:n {#1} }
          }
      }
\cs_new_protected:Npn \@@_parse_mergesc:nw #1
  {
    \peek_remove_spaces:n
      {
        \peek_charcode:NTF [ % ]
          { \@@_parse_mergesc_aux:nw {#1} }
          { \@@_parse_mergesc_aux:nw {#1} [ 1-\seq_count:c { g_@@_name_#1_seq } ] }
      }
  }
\cs_new_protected:Npn \@@_parse_mergesc_aux:nw #1 [#2]
  {
    \seq_clear:N \l_@@_seq_item_seq
    \clist_map_inline:nn {#2}
      { \@@_parse_mergesc_range:nw {#1} ##1 - \q_@@_mark - \s_@@_mark }
    \seq_map_inline:Nn \l_@@_seq_item_seq
      {
        \seq_gput_right:cx { g_@@_name_sc!internal_seq }
          { \seq_item:cn { g_@@_name_#1_seq } {##1} }
      }
    \@@_use_none_delimit_by_s_stop:w
  }
\cs_new_protected:Npn \@@_parse_mergesc_range:nw #1 #2 - #3 - #4 \s_@@_mark
  {
    \cs_set_protected:Npn \@@_tmp:w ##1
      {
        \msg_error:nnxxx { scontents } { index-out-of-range }
          {##1} {#1} { \seq_count:c { g_@@_name_#1_seq } }
      }
    \@@_range_parser:nnxn {#2} {#3}
      { \seq_count:c { g_@@_name_#1_seq } }
      { \@@_tmp:w }
  }
%</core>
%    \end{macrocode}
% \end{macro}
%
% \begin{macro}{verbatimsc,\startverbatimsc,\stopverbatimsc}
%   Finally the \hologo{LaTeX} and \hologo{ConTeXt} version of
%   §verbatimsc§ environment is defined.
%
%   The macro §\endverbatim§ in the second argument of the §verbatimsc§
%   environment is only needed for compatibility with the \pkg{verbatim}
%   package.
%
%    \begin{macrocode}
%<*loader>
%<*!context>
\use:x
  {
    \cs_new_protected:Npn \exp_not:N \@@_xverb:w
      ##1 \g_@@_end_verbatimsc_tl
%<latex>      { ##1 \exp_not:N \end{verbatimsc} }
%<plain>      { ##1 \exp_not:N \endverbatimsc }
%<context>      { ##1 \exp_not:N \stopverbatimsc }
  }
%</!context>
%<*latex>
\NewDocumentEnvironment { verbatimsc } { }
  {
    \cs_set_eq:cN { @xverbatim } \@@_xverb:
    \verbatim
  }
  { \endverbatim }
%</latex>
%<context>\definetyping[verbatimsc]
%</loader>
%<*core>
%    \end{macrocode}
% \end{macro}
%
% \subsubsection{Some auxiliaries functions}
%
% \begin{macro}{\@@_tabs_to_spaces:}
%   In a verbatim context the |TAB| character is made active and set
%   equal to \cs{@@_tabs_to_spaces:}, to produce as many spaces as
%   the §width-tab§ key was set to.
%
%    \begin{macrocode}
\cs_new:Npn \@@_tabs_to_spaces:
  { \prg_replicate:nn { \l_@@_tab_width_int } { ~ } }
%    \end{macrocode}
% \end{macro}
%
% \begin{macro}{\@@_do_noligs:N}
%   \cs{@@_do_noligs:N} is an alternative definition for \LaTeXe's
%   §\do@noligs§ which makes sure to not consume following space
%   tokens. The \LaTeXe{} version ends with §\char`#1§, which
%   leaves \TeX{} still looking for an \mymeta{optional~space}.
%^^A   This version uses \cs{char_generate:nn} to ensure that doesn't
%^^A   happen.
%^^A   Maintain documentation at this point.
%    \begin{macrocode}
\cs_new_protected:Npn \@@_do_noligs:N #1
  {
    \char_set_catcode_active:N #1
    \cs_set:cpx { @@_active_char_ \token_to_str:N #1 : }
      {
        \mode_leave_vertical:
        \tex_kern:D \c_zero_dim
        \tex_char:D `\exp_not:N #1
      }
    \char_set_active_eq:Nc #1 { @@_active_char_ \token_to_str:N #1 : }
  }
%    \end{macrocode}
% \end{macro}
%
% \begin{macro}[TF]{\@@_tl_if_head_is_q_mark:n}
%   Tests if the head of the token list is \cs{q_@@_mark}.
%    \begin{macrocode}
\prg_new_protected_conditional:Npnn \@@_tl_if_head_is_q_mark:n #1
  { T, F, TF }
  {
    \exp_after:wN \if_meaning:w
      \exp_after:wN \q_@@_mark \@@_use_i_delimit_by_s_stop:nw #1 ? \s_@@_stop
      \prg_return_true:
    \else:
      \prg_return_false:
    \fi:
  }
%    \end{macrocode}
% \end{macro}
%
% \begin{macro}{
%     \@@_set_active_eq:NN, \@@_make_control_chars_active:,
%     \@@_plain_disable_outer_par:}
%   Shortcut definitions for common catcode changes.
%   The §^^L§ needs a special treatment in non-\LaTeX{} mode
%   because in Plain\,\TeX{} it is an §\outer§ token.
%    \begin{macrocode}
\cs_new_protected:Npn \@@_set_active_eq:NN #1
  {
    \char_set_catcode_active:N #1
    \char_set_active_eq:NN #1
  }
%</core>
%<*loader>
\group_begin:
%<plain>  \char_set_catcode_active:n { `\* }
  \cs_new_protected:Npn \@@_plain_disable_outer_par:
%<*plain>
    {
      \group_begin:
        \char_set_lccode:nn { `\* } { `\^^L }
        \tex_lowercase:D { \group_end:
        \tex_let:D * \scan_stop:
      }
    }
%</plain>
%<latex|context>    { }
\group_end:
%</loader>
%<*core>
\group_begin:
  \char_set_catcode_active:N \*
  \cs_new_protected:Npn \@@_make_control_chars_active:
    {
      \@@_plain_disable_outer_par:
      \@@_set_active_eq:NN \^^I \@@_tab:
      \@@_set_active_eq:NN \^^L \@@_par:
      \@@_set_active_eq:NN \^^M \@@_ret:w
    }
\group_end:
%    \end{macrocode}
% \end{macro}
%
% \subsection{The command \cs{setupsc}}
%
% User command §\setupsc§ to setup module.
%
% \begin{macro}{\setupsc}
%   A user-level wrapper for \cs{keys_set:nn}§{ scontents }§.
%
%    \begin{macrocode}
%</core>
%<*loader>
\NewDocumentCommand \setupsc { +m }
  { \keys_set:nn { scontents } {#1} }
%</loader>
%<*core>
%    \end{macrocode}
% \end{macro}
%
% \subsection{The command \cs{meaningsc}}
%
% \begin{macro}{\meaningsc,\@@_meaningsc_internal:nn,\@@_meaningsc:n}
%   User command §\meaningsc§ to see content stored in |seq|.
%
%    \begin{macrocode}
%</core>
%<*loader>
\NewDocumentCommand \meaningsc { o m }
  { \@@_meaningsc_internal:nn {#1} {#2} }
%</loader>
%<*core>
\cs_new_protected:Npn \@@_meaningsc_internal:nn #1 #2
  {
    \group_begin:
      \seq_clear:N \l_@@_seq_item_seq
      \str_set:Nx \l_@@_cur_seq_name_str {#2}
      \tl_if_novalue:nF {#1} { \keys_set:nn { scontents / typemeaning } {#1} }
      \seq_if_empty:NT \l_@@_seq_item_seq
        { \seq_set_from_clist:Nn \l_@@_seq_item_seq { 1 } }
      \@@_meaningsc:n {#2}
    \group_end:
  }
\group_begin:
  \char_set_catcode_active:N \^^I
  \cs_new_protected:Npn \@@_meaningsc:n #1
    {
      \tl_set:Nx \l_@@_temp_tl
        { \@@_getfrom_seq:Nn \l_@@_seq_item_seq {#1} }
      \tl_replace_all:Nxn \l_@@_temp_tl { \iow_char:N \^^J } { ~ }
      \tl_replace_all:Nxn \l_@@_temp_tl \c_@@_hidden_space_str { ~ }
      \tl_log:N \l_@@_temp_tl
      \tl_use:N \l_@@_verb_font_tl
      \tl_replace_all:Nnx \l_@@_temp_tl { ^^I } { \@@_tabs_to_spaces: }
      \cs_replacement_spec:N \l_@@_temp_tl
    }
\group_end:
%    \end{macrocode}
% \end{macro}
%
% \subsection{The command \cs{countsc}}
%
% \begin{macro}{\countsc}
%   User command §\countsc§ to count number of contents stored in |seq|.
%
%    \begin{macrocode}
%</core>
%<*loader>
\NewExpandableDocumentCommand \countsc { m }
  { \seq_count:c { g_@@_name_#1_seq } }
%</loader>
%<*core>
%    \end{macrocode}
% \end{macro}
%
% \subsection{The command \cs{cleanseqsc}}
%
%
% \begin{macro}{\cleanseqsc}
%   A user command §\cleanseqsc§ to clear (remove) a defined |seq|.
%
%    \begin{macrocode}
%</core>
%<*loader>
\NewDocumentCommand \cleanseqsc { m }
  { \seq_gclear_new:c { g_@@_name_#1_seq } }
%</loader>
%<*core>
%    \end{macrocode}
% \end{macro}
%
% \subsection{Warning and error messages}
%
% Warning and error messages used throughout the package.
%
%    \begin{macrocode}
\msg_new:nnn { scontents } { junk-after-begin }
  {
    Junk~characters~#1~\msg_line_context: :
    \\ \\
    #2
  }
\msg_new:nnnn { scontents } { env-already-defined }
  { Environment~'#1'~already~defined! }
  {
    You~have~used~\newenvsc
    with~an~environment~that~already~has~a~definition. \\ \\
    The~existing~definition~of~'#1'~will~not~be~altered.
  }
\msg_new:nnn { scontents } { empty-stored-content }
  { Empty~value~for~key~'getstored'~\msg_line_context:. }
\msg_new:nnn { scontents } { empty-variable }
  { Variable~'#1'~empty~\msg_line_context:. }
\msg_new:nnn { scontents } { overwrite-file }
  { Overwriting~file~'#1'. }
\msg_new:nnn { scontents } { writing-file }
  { Writing~file~'#1'. }
\msg_new:nnn { scontents } { not-writing }
  { File~`#1'~already~exists.~Not~writing. }
\msg_new:nnn { scontents } { rescanning-text }
  { Rescanning~text~'#1'~after~\c_backslash_str end{#2}~\msg_line_context:.}
\msg_new:nnn { scontents } { multiple-begin }
  { Multiple~\c_backslash_str begin{ \l_@@_env_name_tl }~\msg_line_context:.}
\msg_new:nnn { scontents } { undefined-storage }
  { Storage~named~'#1'~is~not~defined. }
\msg_new:nnn { scontents } { index-out-of-range }
  {
    \int_compare:nNnTF {#1} = { 0 }
      { Index~of~sequence~cannot~be~zero. }
      {
        Index~'#1'~out~of~range~for~'#2'.~
        \int_compare:nNnTF {#1} > { 0 }
          { Max = } { Min = -} #3.
      }
  }
\msg_new:nnnn { scontents } { env-key-unknown }
  {
    The~key~'#1'~is~unknown~by~environment~
    '\l_@@_env_name_tl'~and~is~being~ignored.
  }
  {
    The~environment~'\l_@@_env_name_tl'~does~not~have~a~key~called~'#1'.\\
    Check~that~you~have~spelled~the~key~name~correctly.
  }
\msg_new:nnnn { scontents } { env-key-value-unknown }
  {
    The~key~'#1=#2'~is~unknown~by~environment~
    '\l_@@_env_name_tl'~and~is~being~ignored.
  }
  {
    The~environment~'\l_@@_env_name_tl'~does~not~have~a~key~called~'#1'.\\
    Check~that~you~have~spelled~the~key~name~correctly.
  }
\msg_new:nnnn { scontents } { cmd-key-unknown }
  { The~key~'#1'~is~unknown~by~'\c_backslash_str Scontents'~and~is~being~ignored.}
  {
    The~command~'\c_backslash_str Scontents'~does~not~have~a~key~called~'#1'.\\
    Check~that~you~have~spelled~the~key~name~correctly.
  }
\msg_new:nnnn { scontents } { cmd-key-value-unknown }
  { The~key~'#1=#2'~is~unknown~by~'\c_backslash_str Scontents'~and~is~being~ignored. }
  {
    The~command~'\c_backslash_str Scontents'~does~not~have~a~key~called~'#1'.\\
    Check~that~you~have~spelled~the~key~name~correctly.
  }
\msg_new:nnnn { scontents } { for-key-unknown }
  { The~key~'#1'~is~unknown~by~'\c_backslash_str foreachsc'~and~is~being~ignored.}
  {
    The~command~'\c_backslash_str foreachsc'~does~not~have~a~key~called~'#1'.\\
    Check~that~you~have~spelled~the~key~name~correctly.
  }
\msg_new:nnnn { scontents } { for-key-value-unknown }
  { The~key~'#1=#2'~is~unknown~by~'\c_backslash_str foreachsc'~and~is~being~ignored. }
  {
    The~command~'\c_backslash_str foreachsc'~does~not~have~a~key~called~'#1'.\\
    Check~that~you~have~spelled~the~key~name~correctly.
  }
\msg_new:nnnn { scontents } { type-key-unknown }
  { The~key~'#1'~is~unknown~and~is~being~ignored. }
  {
    This~command~does~not~have~a~key~called~'#1'.\\
    This~command~only~accepts~the~key~'width-tab'.
  }
\msg_new:nnnn { scontents } { type-key-value-unknown }
  { The~key~'#1'~to~which~you~passed~'#2'~is~unknown~and~is~being~ignored. }
  {
    This~command~does~not~have~a~key~called~'#1'.\\
    This~command~only~accepts~the~key~'width-tab'.
  }
\msg_new:nnn { scontents } { empty-environment }
  { environment~'#1'~empty~\msg_line_context:. }
\msg_new:nnnn { scontents } { verbatim-newline }
  { Verbatim~argument~of~#1~ended~by~end~of~line. }
  {
    The~verbatim~argument~of~the~#1~cannot~contain~more~than~one~line,~
    but~the~end~
    of~the~current~line~has~been~reached.~You~may~have~forgotten~the~
    closing~delimiter.
    \\ \\
    LaTeX~will~ignore~'#2'.
  }
\msg_new:nnnn { scontents } { verbatim-tokenized }
  { The~verbatim~#1~cannot~be~used~inside~an~argument. }
  {
    The~#1~takes~a~verbatim~argument.~
    It~may~not~appear~within~the~argument~of~another~function.~
    It~received~an~illegal~token \tl_if_empty:nF {#3} { ~'#3' } .
    \\ \\
    LaTeX~will~ignore~'#2'.
  }
%    \end{macrocode}
%
% \subsection{Finish package}
%
% Finish package implementation.
%
%    \begin{macrocode}
%</core>
%<plain|context>\ExplSyntaxOff
%    \end{macrocode}
%
% \Finale
