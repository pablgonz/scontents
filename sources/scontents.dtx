% \iffalse meta-comment
%
% Copyright (C) 2019 by Pablo González L <pablgonz@educarchile.cl>
%
% This work may be distributed and/or modified under the conditions of the
% LaTeX Project Public License, either version 1.3c of this license or (at
% your option) any later version. The latest version of this license is in
%
%  http://www.latex-project.org/lppl.txt
%
% and version 1.3c or later is part of all distributions of LaTeX version
% 2005/12/01 or later.
%
% This work is "maintained" (as per the LPPL maintenance status)
% by Pablo González Luengo.
%
% This work consists of the files scontents.dtx and
%                                 scontents.ins,
% and the derived files           scontents.sty,
%                                 scontents.tex,
%                                 t-scontents.mkiv and
%                                 scontents-code.tex.
%
%<*driver>
\documentclass[full]{l3doc}
% imakeidx also opens a write stream for the .idx file, and that conflicts
% with the one opened by l3doc. Here we close that write stream:
\makeatletter
\immediate\closeout\@indexfile
\makeatother
% and later we will copy the write stream opened by imakeidx into \@indexfile
% so that entries written to both streams end up in the same file.
\usepackage[english]{babel}
\usepackage[top=0.5in,bottom=0.3in,left=2in,right=0.7in,footskip=0.2in,headheight=1cm,headsep=0.27cm]{geometry}
\usepackage[osf,mono=false,scale=0.95,llscaled=0.95]{libertine}
\setmonofont[
    Ligatures      = TeX,
    Numbers        = OldStyle,
    Scale          = 0.80,
    UprightFont    = *-Regular,
    ItalicFont     = *-RegularIt,
    BoldFont       = *-Semibold,
    BoldItalicFont = *-SemiboldIt,
    RawFeature     = {+zero,+ss06},
    FontFace       = {m}{sc}{LinBiolinum_R.otf},
    Extension      = .otf]{SourceCodePro}
\newfontfamily\sourcecodeprolight{SourceCodePro-Regular.otf}[Scale = 0.80,FakeStretch =0.75]
\RenewDocumentCommand{\MacroLongFont}{}
  {
    \sourcecodeprolight\small
  }
% The character of visible space is now taken from Latin Modern Mono
% to prevent fonts in T1. The original definition for xetex/luatex is
% \def\verbvisiblespace{\usefont{OT1}{cmtt}{m}{n}\asciispace}
\def\verbvisiblespace{{\fontfamily{lmtt}\selectfont\char"2423}}

\usepackage{unicode-math}
% Unicode-math changes the position of the glyphs, so the position 26A is now
% a letter `j'. The doc package ensures that mathcode and the | actually prints
% a j inside docstrip guards. This changes the mathcode of | to 07C:
\makeatletter
\def\mod@math@codes{\mathcode`\|="207C \mathcode`\&="2026
                    \mathcode`\-="702D \mathcode`\+="702B
                    \mathcode`\:="703A \mathcode`\=="703D }
\makeatother

\setmathfont[Scale = 0.95]{latinmodern-math.otf}
\usepackage{fontawesome5}
\newfontfamily\fetamontotf{ffmw10.otf}[
   Scale             = 0.95,%
   RawFeature        = {+latn,+rand,+kern,+size},%
   FontFace          = {bx}{n}{ffmw10.otf},
   ]
\usepackage[svgnames]{xcolor}
\usepackage[sf,bf,compact,medium,pagestyles]{titlesec}
\usepackage{lastpage,imakeidx,microtype,attachfile2,upquote}
\usepackage{adjustbox,multicol,listings,accsupp,titletoc}
\usepackage{scontents} % main
\usepackage[contents]{colordoc}
\def\textnew#1{\color{color-def}#1\/}
\def\AltMacroFont{\fontencoding\encodingdefault
                  \fontfamily\ttdefault
                  \fontseries\mddefault
                  \fontshape\updefault
                  \small
                  }%

% Patching colordoc.sty to work with l3doc.cls
\ExplSyntaxOn
\cs_new_eq:cN { liii@xmacro@code } \__codedoc_xmacro_code:n
\ExplSyntaxOff
\begingroup
\makeatletter
\catcode`\[\@ne\catcode`\]\tw@
\@makebracesactive
\gdef{[\@openingbrace[\char'173]]
\gdef}[\@closingbrace[\char'175]]
\catcode`\|\z@\catcode`\%12
\catcode`\ \active\catcode`\\\active
|gdef|xmacro@code#1%    \end{macrocode}[|liii@xmacro@code[#1]|end[macrocode]]
|catcode`| 12|gdef|sxmacro@code#1%    \end{macrocode*}[|liii@xmacro@code[#1]|end[macrocode*]]
|endgroup

\EnableCrossrefs
% \PageIndex % \CodelineIndex undoes what \PageIndex does
% \CodelineIndex tries to open another write stream for the index file. We don't
% want that, so we temporarily make \makeindex a no-op:
\let\ORGmakeindex\makeindex
\def\makeindex{}
\CodelineIndex
\let\makeindex\ORGmakeindex
\DoNotIndex{\ , \1,\^}
\expandafter\DoNotIndex\expandafter{\string\{}
\expandafter\DoNotIndex\expandafter{\string\}}
\expandafter\DoNotIndex\expandafter{\string\begin}
\newcommand{\HP}[1]{\emph{\hyperpage{#1}}\normalsize}
\ExplSyntaxOn
\cs_new_protected:Npn \StartImplementation
  { \bool_set_true:N \l__codedoc_in_implementation_bool }
\bool_set_false:N \l__codedoc_in_implementation_bool
\cs_new_protected:Npn \MYSortIndex #1 #2
  {
    \bool_if:NTF \l__codedoc_in_implementation_bool
      { \index{#1\actualchar#2|HP} }
      { \index[userdoc]{#1\actualchar#2|HP} }
  }
\ExplSyntaxOff

\indexsetup{level=\section,firstpagestyle=myheader,othercode=\pagestyle{myheader}}
\makeindex[name=userdoc,options=-s gind.ist,columnsep=15pt,title={Index of Documentation}]
\makeindex[options=-s gind.ist,columnsep=15pt,title={Index of Implementation}]
% Now, after imakeidx opens the write stream for the index file, we copy
% the reference to \@indexfile:
\makeatletter
\expandafter\let\expandafter\@indexfile\csname \jobname @idxfile\endcsname
\makeatother

\setlength{\parindent}{0pt}

% Colors
\definecolor{optcolor}{rgb}{0.281,0.275,0.485}
\definecolor{pkgcolor}{rgb}{0,0.5,0.5}
\definecolor{araracolor}{rgb}{0, 0.72, 0.28}
\definecolor{linkcolor}{rgb}{0.04,0.38,0.04}

% Only for arara...I like arara :)
\usepackage[scale=0.85]{comfortaa}
\newcommand{\araratext}[1]{{\small\normalfont\comfortaa\color{araracolor}\bfseries#1}}
\newcommand*\arara{\araratext{ar\kern-.03emar\kern-.03ema}}

% Logo with fetamont font for title
\makeatletter
\newsavebox{\logobox}
\savebox{\logobox}{%
  \normalsize\fetamontotf{\textcolor{pkgcolor}{s}\textcolor{OrangeRed}{content}\textcolor{pkgcolor}{s}}}
\newcommand{\mylogo}{%
  \settoheight{\@tempdima}{L}%
  \resizebox{!}{\@tempdima}{\usebox{\logobox}}%
 }
\makeatother

% Custom \meta[...]{...}, \marg[...]{...} and \oarg[...]{...} with color
\ExplSyntaxOn
% User level commands
\NewDocumentCommand{\mymeta}{O{}m}
  {
   \userdoc_meta_generic:Nnn \userdoc_meta:n { #1 } { #2 }
  }
\NewDocumentCommand{\mymarg}{O{}m}
  {
   \userdoc_meta_generic:Nnn \userdoc_marg:n { #1 } { #2 }
  }
\NewDocumentCommand{\myoarg}{O{}m}
  {
   \userdoc_meta_generic:Nnn \userdoc_oarg:n { #1 } { #2 }
  }
% Variables and keys
\tl_new:N \l_userdoc_meta_font_tl

\keys_define:nn { userdoc / mymeta }
  {
   type .choice:,
   type / tt .code:n = \tl_set:Nn \l_userdoc_meta_font_tl { \ttfamily },
   type / rm .code:n = \tl_set:Nn \l_userdoc_meta_font_tl { \rmfamily },
   type .initial:n = rm,
   cf .tl_set:N = \l_userdoc_meta_color_tl,
   cf .initial:n = optcolor,
   ac .tl_set:N = \l_userdoc_meta_anglecolor_tl,
   ac .initial:n = lightgray,
   sbc .tl_set:N = \l_userdoc_meta_brackcolor_tl,
   sbc .initial:n = gray,
   cbc .tl_set:N = \l_userdoc_meta_bracecolor_tl,
   cbc .initial:n = gray,
  }
% Internal commands
\cs_new_protected:Npn \userdoc_meta_generic:Nnn #1 #2 #3
  {
   \group_begin:
   \keys_set:nn { userdoc / mymeta } { #2 }
   \color{ \l_userdoc_meta_color_tl }
   \l_userdoc_meta_font_tl
   #1 { #3 } % #1 is \userdoc_meta:n, \userdoc_marg:n or \userdoc_oarg:n
   \group_end:
  }
\cs_new_protected:Npn \userdoc_meta:n #1
  {
   \userdoc_meta_angle:n { \textlangle }
   \userdoc_meta_meta:n { #1 }
   \userdoc_meta_angle:n { \textrangle }
  }
\cs_new_protected:Npn \userdoc_marg:n #1
  {
   \userdoc_meta_brace:n { \textbraceleft }
   \userdoc_meta:n { #1 }
   \userdoc_meta_brace:n { \textbraceright }
  }
\cs_new_protected:Npn \userdoc_oarg:n #1
  {
   \userdoc_meta_brack:n { [ }
   \userdoc_meta:n { #1 }
   \userdoc_meta_brack:n { ] }
  }
\cs_new_protected:Npn \userdoc_meta_meta:n #1
  {
   \textnormal{\textit{#1}}
  }
\cs_new_protected:Npn \userdoc_meta_angle:n #1
  {
   \group_begin:
   \fontfamily{lmr}\selectfont
   \textcolor{\l_userdoc_meta_anglecolor_tl}{#1}
   \group_end:
  }
\cs_new_protected:Npn \userdoc_meta_brace:n #1
  {
   \group_begin:
   \color{\l_userdoc_meta_bracecolor_tl}
    #1
   \group_end:
  }
\cs_new_protected:Npn \userdoc_meta_brack:n #1
  {
   \textcolor{\l_userdoc_meta_brackcolor_tl}{#1}
  }
% \envexamp{m}
\newsavebox{\boxexaenv}
\NewDocumentCommand{\envexamp}{m}
  {
   \begin{lrbox}{\boxexaenv}%
    \begin{minipage}[t]{\marginparwidth}%
     \raggedright\ttfamily\small
     \textcolor{gray}{\textbackslash begin\{\textcolor{pkgcolor}{{#1}}\}\myoarg
     [type=tt,sbc=gray,ac=lightgray,cf=optcolor]{keyval ~ list}}\par%
     \hspace{0.5cm}\mymeta[type=tt,ac=lightgray,cf=gray]{env ~ contents}\par%
     \textcolor{gray}{\textbackslash end\{{\textcolor{pkgcolor}{{#1}}}\}}\par
    \end{minipage}%
   \end{lrbox}%
   \usebox{\boxexaenv}
  }
\NewDocumentCommand{\envexaplain}{O{} m O{end}}
  {
   \begin{lrbox}{\boxexaenv}%
    \begin{minipage}[t]{\marginparwidth}%
     \raggedright\ttfamily\small
     \textcolor{pkgcolor}{{\textbackslash#1#2}}\myoarg
     [type=tt,sbc=gray,ac=lightgray,cf=optcolor]{keyval ~ list}\par%
     \hspace{0.5cm}\mymeta[type=tt,ac=lightgray,cf=gray]{env ~ contents}\par%
     \textcolor{pkgcolor}{{\textbackslash #3#2}}\par
    \end{minipage}%
   \end{lrbox}%
   \usebox{\boxexaenv}
  }

% \cmdexamp{s m o m o}
\DeclareDocumentCommand{\cmdexamp}{o m o m o}
  {
  \group_begin:
  \small\ttfamily
  \textcolor{pkgcolor}{\textbackslash#2}
  \IfValueT{#1}{ \textcolor{MediumOrchid}{#1} }
  \IfValueT{#3}{ \myoarg[type=tt,sbc=gray,ac=lightgray,cf=optcolor]{#3} }
  \IfValueTF{#5}
    {
     \mymeta[ac=gray,type=tt,cf=MediumOrchid]{#5}%
     \mymeta[type=tt,ac=gray,cf=optcolor]{#4}
     \mymeta[ac=gray,type=tt,cf=MediumOrchid]{#5}%
    }
    { \mymarg[type=tt,cbc=gray,ac=lightgray,cf=optcolor]{#4} }
  \par
  \group_end:
  \MYSortIndex{Commands}{Commands ~ provide  ~ by ~
  {\fetamontotf{\textcolor{pkgcolor}{s}\textcolor{OrangeRed}{content}\textcolor{pkgcolor}{s}}}:}%
  }

% \keyexamp{mmm}
\DeclareDocumentCommand{\keyexamp}{ m m m }
  {
    \par
    \adjustbox{outer=-\marginparsep}{\textcolor{black}{\small\ttfamily{#1}}}
    \textcolor{gray}{\,\bfseries\texttt{=}}\,{}
    \mymarg[type=tt,cbc=gray,ac=lightgray,cf=optcolor]{\small{#2}}
    \hfill\textcolor{gray}{\small\textsf{default}:~\emph{#3}}
    \par
    \MYSortIndex{Keys}{Keys>\texttt{#1}}%
  }

% \mykey{m}
\DeclareDocumentCommand{\mykey}{ m }
  {
    \textcolor{optcolor}{\texttt{#1}}
    \MYSortIndex{Keys}{Keys>\texttt{#1}}%
  }

% \mypkg{sm}
\NewDocumentCommand{\mypkg}{sm}
  {
    \group_begin:
    \IfBooleanTF{#1}
      {
        \fetamontotf{\textcolor{pkgcolor}{s}\textcolor{OrangeRed}{content}\textcolor{pkgcolor}{s}}
        \MYSortIndex{packages}{Packages>\texttt{#2}}
      }
      {
        \textcolor{darkgray}{\textsf{#2}}
        \MYSortIndex{packages}{Packages>\texttt{#2}}%
      }
     \group_end:
  }

% \myenv{sm}
\DeclareDocumentCommand{\myenv}{sm}
  {
    \IfBooleanTF{#1}
      {
        \textcolor{pkgcolor}{\ttfamily{#2}}%
        \MYSortIndex{environment}{ Environment ~ provide ~ by ~
        {\fetamontotf{\textcolor{pkgcolor}{s}\textcolor{OrangeRed}{content}\textcolor{pkgcolor}{s}}}:>\texttt{#2}}
      }
      {
       \textcolor{darkgray}{\ttfamily{#2}}%
       \MYSortIndex{environment}{Environments>\texttt{#2}}
      }
  }

% \ics{sm}
\DeclareDocumentCommand{\ics}{sm}
  {
    \IfBooleanTF{#1}
      {
        \tl_set:Nn \l_tmpa_tl { #2 }
        \regex_replace_once:nnN { (\*) } { \c{textcolor}\cB\{MediumOrchid\cE\}\cB\{\1\cE\} } \l_tmpa_tl
        \textcolor{pkgcolor}{\ttfamily\textbackslash{\tl_use:N \l_tmpa_tl}}
        \MYSortIndex{Commands}{ Commands ~ provide  ~ by ~
        {
        \fetamontotf{
        \textcolor{pkgcolor}{s}\textcolor{OrangeRed}{content}\textcolor{pkgcolor}{s}}
        }:>\texttt{\textbackslash#2}}
      }
      {
        \textcolor{darkgray}{\ttfamily\textbackslash{#2}}
        \MYSortIndex{#2}{\texttt{\textbackslash#2}}
      }
  }
\ExplSyntaxOff

% email https://tex.stackexchange.com/a/663
\catcode`\_=11\relax%
\newcommand\email[1]{\_email #1\q_nil}%
\def\_email#1@#2\q_nil{%
  \href{mailto:#1@#2}{{\emailfont #1\emailampersat #2}}%
}%
\newcommand\emailfont{\sffamily}%
\newcommand\emailampersat{{\color{optcolor}\footnotesize @}}%
\catcode`\_=8\relax%

% Custom vertamimsc
\makeatletter
\let\verbatimsc\@undefined
\let\endverbatimsc\@undefined
\makeatother
\usepackage{listings}
\lstnewenvironment{verbatimsc}
  {
    \lstset{
       language=,%
       basicstyle= \ttfamily\small,
       gobble    = 2,%
           }
   }{}
\makeatother

% Don't copy numbers in code example
\newcommand*{\noaccsupp}[1]{\BeginAccSupp{ActualText={}}#1\EndAccSupp{}}

% Create a language for documentation
\lstdefinelanguage{scontents-doc}{
    texcsstyle=*,%
    escapechar=`,%
    extendedchars=true,%
    breakatwhitespace=true,%
    breaklines=true,%
    keepspaces=true,%
    columns=fullflexible,%
    stringstyle= {\color{red}},%
    basicstyle=\ttfamily\small,%
    alsoletter={-,@,.},%
% comments
    morecomment=[l]{\%},%
    commentstyle=\ttfamily\small\color{lightgray},%
% Important words 1
    keywordstyle=[1]{\color{gray}},%
    keywords=[1]{begin,end,Verbatim},%
% Other words 2
    keywordstyle=[2]{\color{blue!75}},%
    keywords=[2]{usepackage,section,documentclass,input,usemodule,
                 subsection,ExplSyntaxOff,ExplSyntaxOn,RequirePackage,
                 ProvidesExplPackage},%
% Other words 3
    keywordstyle=[3]{\color{optcolor!85}},%
    keywords=[3]{document,article,setlength,pagestyle,definecolor},%
% Reserved words 4 (scontents pkg)
    keywordstyle=[4]{\color{pkgcolor}},%
    keywords=[4]{scontents,Scontents,getstored,typestored,verbatimsc,%
                 endverbatimsc,countsc,meaningsc,foreachsc,macros
                 endscontents,startscontents,stopscontents,scontents.tex},%
% Reserved in red
    keywordstyle=[5]{\color{red}},%
    keywords=[5]{makeatletter,makeatother,let,verb,@undefined,makebox,def,%
                par,item,parindent,lstinline,@ifpackagelater,ver@xparse.sty,%
                dospecials,do@noligs,char,outer,jobname,inner},%
% Reserved in orange
    keywordstyle=[6]{\color{optcolor}},%
    keywords=[6]{store-cmd,store-env,print-env,print-cmd,write-env,%
                 start,stopt,wrapper,sep,step,before,after,write-out,%
                 overwrite,width-tab,force-eol,inner,outer,I,J,M,L,Ascheol},%
% Reserved in orange
    keywordstyle=[7]{\color{OrangeRed}},%
    keywords=[7]{DefineVerbatimEnvironment,lstnewenvironment,newminted,%
                 newtheorem,newenvironment,VerbatimEnvironment},%
% literate
    literate=*{\{}{{{\color{gray}{\char`\{}}}}{1}
              {\}}{{{\color{gray}{\char`\}}}}}{1}
              {\|}{{{\color{gray}{\char`\|}}}}{1}
              {\$}{{{\color{gray}{\char`\$}}}}{1}
              {\#}{{{\color{optcolor}{\char`\#}}}}{1}
              {[}{{\textcolor{gray}{[}}}{1}
              {]}{{\textcolor{gray}{]}}}{1}
              {,}{{\textcolor{gray}{,}}}{1}
              {;}{{\textcolor{gray}{;}}}{1}
              {\&}{{\textcolor{gray}{\&}}}{1}
              {>}{{\textcolor{OrangeRed}{\guillemotright}}}{1}
              {*}{{\textcolor{MediumOrchid}{*}}}{1}
              {\^}{{\textcolor{optcolor}{\textasciicircum}}}{1}
              {0}{{\textcolor{MediumOrchid}{0}}}{1}
              {1}{{\textcolor{MediumOrchid}{1}}}{1}
              {2}{{\textcolor{MediumOrchid}{2}}}{1}
              {3}{{\textcolor{MediumOrchid}{3}}}{1}% expl3
              {4}{{\textcolor{MediumOrchid}{4}}}{1}
              {5}{{\textcolor{MediumOrchid}{5}}}{1}
              {6}{{\textcolor{MediumOrchid}{6}}}{1}
              {7}{{\textcolor{MediumOrchid}{7}}}{1}
              {8}{{\textcolor{MediumOrchid}{8}}}{1}
              {9}{{\textcolor{MediumOrchid}{9}}}{1}
              {.0}{{\textcolor{MediumOrchid}{.0}}}{2}% Following is to ensure that only periods
              {.1}{{\textcolor{MediumOrchid}{.1}}}{2}% followed by a digit are changed.
              {.2}{{\textcolor{MediumOrchid}{.2}}}{2}
              {.3}{{\textcolor{MediumOrchid}{.3}}}{2}
              {.4}{{\textcolor{MediumOrchid}{.4}}}{2}
              {.5}{{\textcolor{MediumOrchid}{.5}}}{2}
              {.6}{{\textcolor{MediumOrchid}{.6}}}{2}
              {.7}{{\textcolor{MediumOrchid}{.7}}}{2}
              {.8}{{\textcolor{MediumOrchid}{.8}}}{2}
              {.9}{{\textcolor{MediumOrchid}{.9}}}{2}
              {=}{{\textcolor{gray}{=}}}{1},%
}[keywords,tex,comments,strings]% end languaje

% \begin{examplecode}[key=val]...\end{examplecode}
\lstnewenvironment{examplecode}[1][]{%
\lstset{
    language=scontents-doc,%
    stringstyle= {\color{red}},%
    basicstyle=\ttfamily\small,%
    numbersep=1em,%
    numberstyle=\tiny\color{lightgray}\noaccsupp,%
    rulecolor=\color{gray!50},%
    framesep=\fboxsep,%
    framerule=\fboxrule,%
    xleftmargin=\dimexpr\fboxsep+\fboxrule\relax,%
    xrightmargin=\dimexpr\fboxsep+\fboxrule\relax,%
           #1,%
    }% close lstset
}{}% close examplecode

% \lstinline[style=inline]|...|
\lstdefinestyle{inline}
  {
   language=scontents-doc,%
   escapechar=`,%
   upquote=true,%
   numbersep=1em,%
   columns=fullflexible,%
   numberstyle=\tiny\color{lightgray}\noaccsupp,%
   literate=*{\%}{{\bfseries\textcolor{gray}{\%}}}{1}
  }

% Set default \lstinline style
\lstset{style=inline}
\lstMakeShortInline[language=scontents-doc,basicstyle=\ttfamily]§

% Get file info
\GetFileInfo{\jobname.sty}

% Config hyperref
\hypersetup{
   allcolors          = linkcolor,
   colorlinks         = true,%
   linktoc            = all,%
   pdftitle           = {.:: The scontents package --- \fileinfo{} ::.},%
   pdfauthor          = {Pablo González L},
   pdfsubject         = {Documentation for \fileversion{} [\filedate] },%
   pdfkeywords        = {sequences, contents, external files, expl3, xparse, l3seq, store contents},
   bookmarksopenlevel = 1,%
  }

% Configuration titleps and titlesec
\AtBeginDocument{\edef\scontentsrightmargin{\the\oddsidemargin}}
\settitlemarks{section}
\renewpagestyle{plain}[\color{optcolor}\small\sffamily]{%
\setfoot{}{}{\rlap{\parbox{\scontentsrightmargin}{\hspace{20pt}\thepage\,/\,\pageref{LastPage}}}}%
}
\newpagestyle{myheader}[\color{optcolor}\small\sffamily]{%
\renewcommand\makeheadrule{%
\rlap{\hskip\dimexpr-\oddsidemargin-1in\relax
      \color{WhiteSmoke}\rule[0.3cm]{\paperwidth}{0.7cm}}\hss
}
\setfoot{}%
        {}%
        {\rlap{\parbox{\dimexpr\scontentsrightmargin\relax}{\hspace{20pt}\thepage\,/\,\pageref{LastPage}}}}%
\sethead{\llap{\raisebox{0.55cm}{\parbox{\dimexpr\oddsidemargin+1in\relax}{\makebox[0pt][l]{\hspace{10pt}\scalebox{0.8}{\mylogo}\space\fileversion}}}}}
        {\raisebox{0.55cm}{\parbox{\textwidth}{\hspace*{-\oddsidemargin}\centering\scshape\small\S.\thesection\space\sectiontitle}}}%
        {}%
}

\titlecontents{section}[0mm]{}%
    {\bfseries\contentspush{\makebox[5mm][l]{\thecontentslabel\hfill}}}%
    {\hspace*{-5mm}}% numberless
    {\hspace{0.25em}\titlerule*[6pt]{.}\contentspage}%
\titlecontents{subsection}[5mm]{}%
    {\contentspush{\makebox[6mm][l]{\thecontentslabel\hfill}}}
    {\hspace*{-11mm}}% numberless
    {\hspace{0.25em}\titlerule*[6pt]{.}\contentspage}%

% Table of contents
\makeatletter
\renewcommand\tableofcontents{%
\begingroup%
\section*{\contentsname\quad{\color{gray}\leaders\hrule height 5pt depth -4.4pt\hfill}%
  \@mkboth{%
    \MakeUppercase\contentsname}{\MakeUppercase\contentsname}}%
\vspace*{-14pt}
\setlength{\columnsep}{10pt}%
 \begin{multicols}{2}%
    \@starttoc{toc}%
\end{multicols}%
\vspace*{-3pt}{\color{gray}\hrule height 0.6pt}%
\vspace*{5pt}
\endgroup
}
\makeatother
\begin{document}
  \DocInput{\jobname.dtx}
\end{document}
%</driver>
% \fi
%
% \title{
%    \scalebox{1.13}{\mylogo}\\[2pt]
%    \Large
%    \textsf{Stores \hologo{LaTeX}}{ \fetamontotf{contents} }\\[3pt]
%    \fileversion{} --- \filedate\thanks{
%    This file describes a documentation for \fileversion, last revised
%    \filedate.}\\[25pt]
%    \author{
%    \large
%    \raisebox{-1pt}{\textcopyright}{}2019 by Pablo González
%    \thanks{
%    E-mail: \textcolor{optcolor}{\textsf{\guillemotleft}}\email{pablgonz@educarchile.cl}\textcolor{optcolor}{\textsf{\guillemotright}}.
%       }%
%    }
% \small
% \textsc{ctan}: \url{https://www.ctan.org/pkg/scontents}\\
% \textcolor{gray}{\scriptsize\faIcon[regular]{github}}\,\,\,\url{https://github.com/pablgonz/scontents}
% \vspace*{-2cm}
% }
% \date{}
% \maketitle
%
% \begin{scontents}[store-env=abstract, print-env=true]
% \begin{abstract}
% This package allows to store \hologo{LaTeX} code, including \enquote{\emph{verbatim}},
% in \mymeta{sequences} using the \mypkg{l3seq} module of \mypkg{expl3}. The \mymeta{stored
% content} can be used as many times as desired in the document, additionally you can write
% to \mymeta{external files} or show it in \mymeta{verbatim style}.
% \end{abstract}
% \end{scontents}
%
% \tableofcontents
%
% \setlength{\parskip}{3pt}
%
% \section{Motivation and Acknowledgments}
%
% In \hologo{LaTeX} there is no direct way to record content for later use, although
% you can do this using §\macros§, recording \mymeta[type=rm]{verbatim content}  is a
% problem, usually you can avoid this by creating external files or boxes. The
% general idea of this package is to try to imitate this implementation
% \enquote{\emph{buffers}} that has \hologo{ConTeXt} which allows you to save content in
% memory, including \emph{verbatim}, to be used later. The package
% \mypkg{filecontentsdef} solves this problem and since \mypkg{expl3} has an
% excellent way to manage data, ideas were combined giving rise to this package.
%
% This package would not be possible without the great work of \textsc{Jean
% Fran\c{c}ois Burnol} who was kind enough to take my requirements into account
% and add the \myenv{filecontentsdefmacro} environment. Also a special thanks to
% Phelype Oleinik who has collaborated and adapted a large part of the code and
% all \hologo{LaTeX3} team for their great work and to the different members
% of the \href{https://tex.stackexchange.com}{TeX-SX} community who have provided
% great answers and ideas. Here a note of the main ones:
%
% \begin{enumerate}[nosep]
% \item \href{https://tex.stackexchange.com/q/45946/7832}{Stack datastructure using LaTeX}
%
% \item \href{https://tex.stackexchange.com/q/5338/7832}{LaTeX equivalent of ConTeXt buffers}
%
% \item \href{https://tex.stackexchange.com/q/215563/7832}{Storing an array of strings in a command}
%
% \item \href{https://tex.stackexchange.com/q/184503/7832}{Collecting contents of environment and store them for later retrieval}
%
% \item \href{https://tex.stackexchange.com/q/373647/7832}{Collect contents of an environment (that contains verbatim content)}
% \end{enumerate}
%
% \section{License and Requirements}
% \label{sec:licence}
%
% Permission is granted to copy, distribute and/or modify this software under
% the terms of the LaTeX Project Public License (lppl), version 1.3 or later
% (\url{http://www.latex-project.org/lppl.txt}). The software has the status
% \enquote{maintained}.
%
% \medskip
%
% The \mypkg*{scontents} package loads \mypkg{expl3}, \mypkg{xparse} and
% \mypkg{l3keys2e}. This package can be used with "plain", "context", "xelatex",
% "lualatex", "pdflatex" and the classical workflow §latex>dvips>ps2pdf§.
%
% \thispagestyle{plain}
%
% \newpage
%
% \pagestyle{myheader}
%
% \section{The \texttt{scontents} package}
% \label{sec:pkgscontents}
%
% \subsection{Description of the package and load}
%
% \begin{scontents}[store-env=description, print-env=true]
% The \mypkg*{scontents} package allows to \mymeta{store contents} in \mymeta{sequences}
% or \mymeta{external files}. In some ways it is similar to the \mypkg{filecontentsdef}
% package, with the difference in which the \mymeta{content} is stored. The idea behind
% this package is to get an approach to \hologo{ConTeXt} \enquote{\emph{buffers}} by
% making use \mymeta{sequences}.
% \end{scontents}
%
% The package is loaded in the usual way:
%
% \subsubsection*{For \hologo{LaTeX} users}
% \iffalse
%<*example>
% \fi
\begin{examplecode}[frame=single]
\usepackage{scontents}
\end{examplecode}
% \iffalse
%</example>
% \fi
% \vspace*{-5pt}
% or
% \iffalse
%<*example>
% \fi
\begin{examplecode}[frame=single]
\usepackage`\small\myoarg[type=tt]{key \textnormal{\textcolor{gray}{=}} val}`{scontents}
\end{examplecode}
% \iffalse
%</example>
% \fi
%
% The package options are not available for \hologo{plainTeX} and
% \hologo{ConTeXt}, see \ref{sec:confopt}.
%
% \subsubsection*{For \hologo{plainTeX} users}
%
% \iffalse
%<*example>
% \fi
\begin{examplecode}[frame=single]
\input scontents.tex
\end{examplecode}
% \iffalse
%</example>
% \fi
%
% \subsubsection*{For \hologo{ConTeXt} users}
%
% \iffalse
%<*example>
% \fi
\begin{examplecode}[frame=single]
\usemodule{scontents}
\end{examplecode}
% \iffalse
%</example>
% \fi
%
% \subsection{The TAB character}
% \label{sec:tabcharacter}
%
% Some users use horizontal "TAB"s \enquote{\LKeyTab} from keyboard to indented the source
% code of the document and depending on the text editor used, some will use real "TAB"s
% (\enquote{hard tabs}), others with \enquote{soft tabs}(\verbvisiblespace\verbvisiblespace{}
% or \verbvisiblespace\verbvisiblespace\verbvisiblespace\verbvisiblespace) or both.
%
% At first glance it may seem the same, but the way in which "TAB"s (\enquote{hard tabs})
% are processed according to the context in which they are found within
% a file, both in \mymeta{reading}\footnote{Check the answer given by Ulrich Diez
% in \href{https://tex.stackexchange.com/a/508103/7832}{Keyboard TAB
% character in argument v (xparse)}.} and \mymeta{writing}\footnote{Check the answer
% given by Enrico Gregorio in \href{https://tex.stackexchange.com/a/58740/7832}{How to output a
% tabulation into a file}.} are different and may have adverse consequences.
%
% In a standard \hologo{LaTeX} document, the character "TAB" \enquote{\LKeyTab}
% are treated as explicit spaces (in most contexts) and is the behavior when
% \mymeta{stored contents}, but when \mymeta{writing files} these are preserved.
%
% With a \hologo{TeX}Live distribution, the "TAB" character is \enquote{printable}
% for "latex", "pdflatex" and "lualatex", but if you use "xelatex" you
% must add the \texttt{\textcolor{optcolor}{-8bit}} option on the command
% line, otherwise you will get \hologo{TeX}-"TAB" (§^^I§) in the \mymeta{output file}.
%
% As a general recommendation \enquote{Do not use "TAB" character unless strictly
% necessary}, for example within a \emph{verbatim} environment that supports
% this character such as \myenv{Verbatim} of the package \mypkg{fancyvrb} or \myenv{lstlisting}
% of the package \mypkg{listings} or when you want to generate a §MakeFile§ file.
%
% \subsection{Configuration of the options}
% \label{sec:confopt}
%
% Most of the options can be passed directly to the package or using the
% command \ics*{setupsc}. All boolean keys can be passed using the
% equal sign \enquote{\textnormal{\textcolor{gray}{=}}} or just the
% name of the key, all unknown keys will return an error. In this section
% are described some of the options, a summary of all options is shown in
% section \ref{sec:optover}.
%
% \vspace*{-10pt}
%
% \begin{function}{\setupsc}
%   \begin{syntax}
%       \cmdexamp{setupsc}{keyval list}
%   \end{syntax}
%
% The command \ics*{setupsc} sets the \mymeta{keys} in a global way,
% it can be used both in the preamble and in the body of the document
% as many times as desired.
% \end{function}
%
% \keyexamp{verb-font}{font family}{\textnormal{\ttfamily\textbackslash{}ttfamily}}
% Sets the \mymeta{font family} used to display the \mymeta{stored content} for the
% \ics*{typestored} and \ics*{meaningsc} commands. This key is only available
% as a package option or using \ics*{setupsc}.
%
% \medskip
%
% \keyexamp{store-all}{seq name}{not used}
% It is a \mymeta{meta-key} that sets the §store-env§ key of the \myenv*{scontents}
% environment and the §store-cmd§ key of the \ics*{Scontents} command. This key is
% only available as a package option or using \ics*{setupsc}.
%
% \medskip
%
% \keyexamp{overwrite}{true \textnormal{\textcolor{lightgray}{\textbar}} false}{false}
% Sets whether the \mymeta{files} generated by §write-out§ and §write-env§
% from the \myenv*{scontents} environment will be rewritten. This key is
% available as a package option or using \ics*{setupsc} or \myenv*{scontents}
% environment.
% \medskip
%
% \keyexamp{print-all}{true \textnormal{\textcolor{lightgray}{\textbar}} false}{false}
% It is a \mymeta{meta-key} that sets the §print-env§ key of the \myenv*{scontents}
% environment and the §print-cmd§ key of the \ics*{Scontents} command. This key is
% only available as a package option or using \ics*{setupsc}.
%
% \keyexamp{force-eol}{true \textnormal{\textcolor{lightgray}{\textbar}} false}{false}
% Sets if the end of line for the \mymeta{stored content} is hidden or not.
% This key is necessary only if the last line is the closing of some
% environment defined by the \mypkg{fancyvrb} package as §\end{Verbatim}§
% or another environment that does not support a comments \enquote{\textcolor{gray}{\%}}
% after closing §\end{§\textcolor{gray}{...}§}§\textcolor{gray}{\%}. This key is available for the \myenv*{scontents}
% environment and the \ics*{Scontents} command.
%
% \keyexamp{width-tab}{integer}{1}
% Sets the equivalence in \mymeta{spaces} for the character "TAB"
% used when displaying stored content in \emph{verbatim style}. The value
% must be a \mymeta{positive integer}. This key is available for the \ics*{typestored}
% and the \ics*{meaningsc} commands.
%
% \subsection{Options Overview}
% \label{sec:optover}
%
% \newcommand*{\xmark}{\textcolor{OrangeRed}{✘}}%
% \newcommand*{\cmark}{\textcolor{linkcolor}{✔}}%
%
% Summary of available options.
%
% \setlength{\tabcolsep}{0.25em}
% \begin{center}
% \small
% \begin{tabular}{cccccccc}
% \toprule
% \texttt{key} &package &\ics*{setupsc} &\myenv*{scontents} &\ics*{Scontents} &\ics*{Scontents*} &\ics*{typestored} &\ics*{meaningsc}\\
% \midrule
%  \mykey{store-env} & \cmark  & \cmark  &  \cmark  & \xmark  & \xmark  & \xmark  & \xmark \\
%  \mykey{store-cmd} & \cmark  & \cmark  &  \xmark  & \cmark  & \cmark  & \xmark  & \xmark \\
%  \mykey{print-env} & \cmark  & \cmark  &  \cmark  & \xmark  & \xmark  & \xmark  & \xmark \\
%  \mykey{print-cmd} & \cmark  & \cmark  &  \xmark  & \cmark  & \cmark  & \xmark  & \xmark \\
%  \mykey{print-all} & \cmark  & \cmark  &  \xmark  & \xmark  & \xmark  & \xmark  & \xmark \\
%  \mykey{store-all} & \cmark  & \cmark  &  \xmark  & \xmark  & \xmark  & \xmark  & \xmark \\
%  \mykey{write-env} & \xmark  & \xmark  &  \cmark  & \xmark  & \xmark  & \xmark  & \xmark \\
%  \mykey{write-out} & \xmark  & \xmark  &  \cmark  & \xmark  & \xmark  & \xmark  & \xmark \\
%  \mykey{overwrite} & \cmark  & \cmark  &  \cmark  & \xmark  & \xmark  & \xmark  & \xmark \\
%  \mykey{width-tab} & \cmark  & \cmark  &  \xmark  & \xmark  & \xmark  & \cmark  & \cmark \\
%  \mykey{force-eol} & \cmark  & \cmark  &  \cmark  & \xmark  & \cmark  & \xmark  & \xmark \\
%  \mykey{verb-font} & \cmark  & \cmark  &  \xmark  & \xmark  & \xmark  & \xmark  & \xmark \\
% \bottomrule
% \end{tabular}
% \end{center}
%
% \section{User interface}
% \label{sec:interface}
%
% The user interface consists in \myenv*{scontents} environment, \ics*{Scontents}
% and \ics*{Scontents*} commands to \mymeta{stored content} and \ics*{getstored}
% command to get the \mymeta{stored content} along with other utilities described
% in this documentation.
%
% \subsection{The environment \texttt{scontents}}
% \label{sec:envscontents}
%
% \vspace*{-10pt}
%
% \begin{function}[label=SCONTENTS]{scontents}
%   \begin{syntax}
%   \envexamp{scontents}
%   \end{syntax}
%
% The \myenv*{scontents} environment allows you to \mymeta{store} and \mymeta{write}
% content, including \emph{verbatim} material. After the package has been
% loaded, the environment can be used both in the preamble and in the body
% of the document.
%
% For the correct operation §\begin{scontents}§ and §\end{scontents}§ must
% be in different lines, all \mymeta{keys} must be passed separated by
% commas and \enquote{without separation} of the start of the environment.
% \end{function}
%
% Comments \enquote{\textcolor{gray}{\%}} or \enquote{any character} after
% §\begin{scontents}§ or \myoarg{keyval list} on the same line are not
% supported, the package will return an \enquote{error} message if this happens.
% In a similar way comments \enquote{\textcolor{gray}{\%}} or \enquote{any character} after
% §\end{scontents}§ on the same line the package will return a \enquote{warning} message.
%
% The environment can be \mymeta{nested} if it is properly balanced and
% does not appear \enquote{literally} in commented lines or in some \emph{verbatim}
% environment or command. As an example:
%
% \iffalse
%<*example>
% \fi
\begin{examplecode}[frame=single]
\begin{scontents}[store-env=outer]
This text is in the outer environment (before nested).
\begin{scontents}[store-env=inner]
This text is found in the inner environment (inside of nested).
\end{scontents}
This text is in the outer environment (after nested).
\end{scontents}
\end{examplecode}
% \iffalse
%</example>
% \fi
%
% Of course, content stored in the \mymeta{inner} sequence is only available
% after content stored in the \mymeta{outer} sequence one has been retrieved,
% either by using the key §print-env§ or §\getstored§ command.
%
% It is advisable to store content within sequences with different names,
% so as not to get lost in the order in which content is stored.
%
% \subsection*{Notes for \hologo{plainTeX} and \hologo{ConTeXt} users}
%
% In \hologo{plainTeX} there is not environments as in \hologo{LaTeX}.
% Instead of using the environment \myenv*{scontents}, one should use a
% \emph{pseudo environment} delimited by \ics*{scontents} and \ics*{endscontents}.
%
% \vspace*{-10pt}
% \begin{function}{\scontents,\endscontents}
%   \begin{syntax}
%   \envexaplain{scontents}
%   \end{syntax}
% \end{function}
%
% \hologo{ConTeXt} users should use \ics*{startscontents} and \ics*{stopscontents}.
% \vspace*{-10pt}
% \begin{function}{\startscontents,\stopscontents}
%   \begin{syntax}
%   \envexaplain[start]{scontents}[stop]
%   \end{syntax}
% \end{function}
% \subsection*{Options for environment}
%
% The environment options can be configured globally using option
% in package or the \ics*{setupsc} command and locally using
% \myoarg{key \textnormal{\textcolor{gray}{=}} val} in the environment.
% The key §force-eol§ is available for this environment.
%
% \keyexamp{store-env}{seq name}{contents}
% Sets the name of the \mymeta{sequence} in which the contents will be
% stored. If the sequence does not exist, it will be created globally.
%
% \medskip
%
% \keyexamp{print-env}{true \textnormal{\textcolor{lightgray}{\textbar}} false}{false}
% Sets if the \mymeta{stored content} is displayed or not at the time of
% running the environment. The content is extracted from the \mymeta{sequence}
% in which it is stored.
%
% \medskip
%
% \keyexamp{write-env}{file.ext}{not used}
% Sets the name of the \mymeta{external file} in which the \mymeta{contents} of
% the environment will be written. The \mymeta{file.ext} will be created
% in the working directory, if \mymeta{file.ext} exists it will be overwritten,
% relative or absolute paths are not supported.
% The characters "TAB"s will be written in \mymeta{file.ext} and the \mymeta{contents}
% will be stored in the sequence established at that time. \hologo{XeLaTeX} users
% using the "TAB" character must add \texttt{\textcolor{optcolor}{-8bit}}
% at the command line, otherwise you will get \hologo{TeX}-"TAB" (§^^I§)
% in \mymeta{file.ext}.
%
% \medskip
%
% \keyexamp{write-out}{file.ext}{not used}
% Sets the name of the \mymeta{external file} in which the \mymeta{contents} of
% the environment will be written. The \mymeta{file.ext} will be created
% in the working directory, if \mymeta{file.ext} exists it will be overwritten,
% relative or absolute paths are not supported.
% The characters "TAB"s will be written in \mymeta{file.ext}, the rest of the
% \mymeta{keys} will not be available and the \mymeta{contents} will NOT
% be stored in any sequence. \hologo{XeLaTeX} users using the "TAB" character
% must add \texttt{\textcolor{optcolor}{-8bit}} at the command line,
% otherwise you will get \hologo{TeX}-"TAB" (§^^I§) in \mymeta{file.ext}.
%
% \subsection{The command \cs{Scontents}}
% \label{sec:Scontents}
%
% \vspace*{-10pt}
%
% \begin{function}{\Scontents}
%   \begin{syntax}
%     \cmdexamp{Scontents}[key \textnormal{\textcolor{gray}{=}} val]{argument}
%     \cmdexamp[*]{Scontents}[key \textnormal{\textcolor{gray}{=}} val]{argument}
%     \cmdexamp[*]{Scontents}[key \textnormal{\textcolor{gray}{=}} val]{argument}[del]
%   \end{syntax}
% \end{function}
%
% The \ics*{Scontents} command reads the \mymarg{argument} in standard
% mode. It is not possible to pass environments such as \emph{verbatim},
% but it is possible to use the implementation of \ics{Verb} provided by
% the \mypkg{fvextra} package for contents on one line and \ics{lstinline}
% from \mypkg{listings} package, but it is preferable to use the starred
% version.
%
% The \ics*{Scontents*} command reads the \mymarg{argument} under \emph{verbatim}
% category code regimen. If its first delimiter is a brace, it will be
% assumed that the \mymarg{argument} is nested into braces. Otherwise it
% will be assumed that the ending of that \mymeta{argument} is delimited by that
% first delimiter \mymeta[cf=MediumOrchid]{del} like command \ics{verb}.
%
% Blank lines are preserved, escaped braces \enquote{\texttt{\textcolor{gray}{\textbackslash\hspace{-1pt}\{}}} and
% \enquote{\texttt{\textcolor{gray}{\textbackslash\}}}} must also be
% balanced if the argument is used with braces and "TAB"s characters typed
% from the keyboard are converted into spaces.
%
% Both versions can be used anywhere in the document and cannot be used
% as an \mymeta{argument} for other command.
%
% \subsection*{Options for command}
% \label{sec:optcmdsc}
%
% The command options can be configured globally using option in package
% or the \ics*{setupsc} command and locally using \myoarg{key \textnormal{\textcolor{gray}{=}} val}.
% The key §force-eol§ is available for this command.
%
% \keyexamp{store-cmd}{seq name}{contents}
% Sets the name of the \mymeta{sequence} in which the contents will be stored.
% If the sequence does not exist, it will be created globally.
%
% \medskip
%
% \keyexamp{print-cmd}{true \textnormal{\textcolor{lightgray}{\textbar}} false}{false}
% Sets if the \mymeta{stored content} is displayed or not at the time of
% running the command. The content is extracted from the \mymeta{sequence}
% in which it is stored.
%
% \subsection{The command \cs{getstored}}
% \label{sec:getstored}
%
% \vspace*{-10pt}
%
% \begin{function}{\getstored}
%   \begin{syntax}
%      \cmdexamp{getstored}[index]{seq name}
%   \end{syntax}
% \end{function}
%
% The command \ics*{getstored} gets the content stored in \mymarg{seq name}
% according to the \mymeta{index} in which it was stored. The command is
% robust and can be used as an \mymeta{argument} for another command. If the
% optional argument is not passed it defaults to the first element stored
% in the \mymarg{seq name}.
%
% \subsection{The command \cs{foreachsc}}
% \label{sec:foreachsc}
%
% \vspace*{-10pt}
%
% \begin{function}{\foreachsc}
%   \begin{syntax}
%     \cmdexamp{foreachsc}[key \textnormal{\textcolor{gray}{=}} val]{seq name}
%   \end{syntax}
% \end{function}
%
% The command \ics*{foreachsc} goes through and executes the command §\getstored§
% on the contents stored in \mymarg{seq name}. If you pass without options run
% §\getstored§ on all contents stored in \mymarg{seq name}.
%
% \subsection*{Options for command}
% \label{sec:optcmdfor}
%
% \keyexamp{sep}{code}{empty}
% Establishes the separation between each content stored in \mymarg{seq name}.
% For example, you can use §sep={\\[10pt]}§ for vertical separation of stored
% contents.
%
% \keyexamp{step}{integer}{1}
% Sets the increment (\mymeta{step}) applied to the value set by key §start§
% for each element stored in the \mymarg{seq name}. The value must be a
% \mymeta{positive integer}.
%
% \keyexamp{start}{integer}{1}
% Sets the \mymeta{index} number of the \mymarg{seq name} from which execution
% will start. The value must be a \mymeta{positive integer}.
%
% \keyexamp{stop}{integer}{total}
% Sets the \mymeta{index} number of the \mymarg{seq name} from which execution
% it will finish executing. The value must be a \mymeta{positive integer}.
%
% \keyexamp{before}{code}{empty}
% Sets the \mymarg{code} that will be executed \mymeta{before} each content stored
% in \mymarg{seq name}. The \mymarg{code} must be passed between braces.
%
% \medskip
%
% \keyexamp{after}{code}{empty}
% Sets the \mymarg{code} that will be executed \mymeta{after} each content stored
% in \mymarg{seq name}. The \mymarg{code} must be passed between braces.
%
% \keyexamp{wrapper}{code \textnormal{\textcolor{gray}{\{\#1\}}} more code}{empty}
% Wraps the content stored in \mymarg{seq name} referenced by \mymarg{\#1}.
% The \mymarg{code} must be passed between braces. For example
% §\foreachsc[wrapper={\makebox[1em][l]{#1}}]{contents}§.
%
%
% \subsection{The command \cs{typestored}}
% \label{sec:typestored}
%
% \vspace*{-10pt}
%
% \begin{function}{\typestored}
%   \begin{syntax}
%      \cmdexamp{typestored}[index\textnormal{\textcolor{gray}{,}} width-tab \textnormal{\textcolor{gray}{=}} number]{seq name}
%   \end{syntax}
% The command \ics*{typestored} internally places the content stored in
% the \mymarg{seq name} into the \myenv*{verbatimsc} environment. The \mymeta{index}
% corresponds to the position in which the content is stored in the \mymarg{seq name}.
% \end{function}
%
% If the optional argument is not passed it defaults to the first element
% stored in the \mymarg{seq name}. The key §width-tab§ is available for
% this command.
%
%
% \subsection{The environment \env{verbatimsc}}
% \label{sec:verbatimsc}
%
% \vspace*{-10pt}
%
% \begin{function}{verbatimsc}
% Internal environment used by \ics*{typestored} to display \mymeta{verbatim style}
% contents.
% \end{function}
%
% One consideration to keep in mind is that this is a \emph{representation}
% of the \mymeta{stored content} in a \emph{verbatim} environment and not
% a real \emph{verbatim} environment. The \mypkg{verbatim} package is not
% compatible with the implementation of the \myenv*{verbatimsc} environment.
%
% \newpage
%
% The \myenv*{verbatimsc} environment can be customized in the following ways:
%
% Using the package \mypkg{fancyvrb}:
% \iffalse
%<*example>
% \fi
\begin{examplecode}[frame=single]
\makeatletter
\let\verbatimsc\@undefined
\let\endverbatimsc\@undefined
\makeatother
\DefineVerbatimEnvironment{verbatimsc}{Verbatim}{numbers=left}
\end{examplecode}
% \iffalse
%</example>
% \fi
% Using the package \mypkg{minted}:
% \iffalse
%<*example>
% \fi
\begin{examplecode}[frame=single]
\makeatletter
\let\verbatimsc\@undefined
\let\endverbatimsc\@undefined
\makeatother
\usepackage{minted}
\newminted{tex}{linenos}
\newenvironment{verbatimsc}{\VerbatimEnvironment\begin{texcode}}{\end{texcode}}
\end{examplecode}
% \iffalse
%</example>
% \fi
% Using the package \mypkg{listings}:
% \iffalse
%<*example>
% \fi
\begin{examplecode}[frame=single]
\makeatletter
\let\verbatimsc\@undefined
\let\endverbatimsc\@undefined
\makeatother
\usepackage{listings}
\lstnewenvironment{verbatimsc}
  {
   \lstset{
           basicstyle=\small\ttfamily,
           columns=fullflexible,
           language=[LaTeX]TeX,
           numbers=left,
           numberstyle=\tiny\color{gray},
           keywordstyle=\color{red}
          }
  }{}
\end{examplecode}
% \iffalse
%</example>
% \fi
%
% \section{Other commands provided}
%
% \subsection{The command \cs{meaningsc}}
% \label{sec:meaningsc}
%
% \vspace*{-10pt}
%
% \begin{function}{\meaningsc}
%   \begin{syntax}
%      \cmdexamp{meaningsc}[index\textnormal{\textcolor{gray}{,}} width-tab \textnormal{\textcolor{gray}{=}} number]{seq name}
%   \end{syntax}
% The command \ics*{meaningsc} executes \ics{meaning} on the content stored
% in \mymarg{seq name}. The \mymeta{index} corresponds to the position
% in which the content is stored in the \mymarg{seq name}.
% \end{function}
%
% If the optional argument is not passed it defaults to the first element
% stored in the \mymarg{seq name}. The key §width-tab§ is available for
% this command.
%
% \subsection{The command \cs{countsc}}
% \label{sec:countsc}
%
% \vspace*{-10pt}
%
% \begin{function}{\countsc}
%   \begin{syntax}
%      \cmdexamp{countsc}{seq name}
%   \end{syntax}
% The command \ics*{countsc} count a number of contents stored in \mymarg{seq name}.
% \end{function}
%
% \subsection{The command \cs{cleanseqsc}}
% \label{sec:cleansc}
%
% \vspace*{-10pt}
%
% \begin{function}{\cleanseqsc}
%   \begin{syntax}
%      \cmdexamp{cleanseqsc}{seq name}
%   \end{syntax}
% The command \ics*{cleanseqsc} remove all contents stored in \mymarg{seq name}.
% \end{function}
%
% \section[The \textnormal{\texttt{scontents}} package in action]{The \mypkg*{scontents} package in action}
%
% Remember the abstract on the first page?, this is it:
%
% \getstored{abstract}
%
% And the description of the package?
%
% \getstored{description}
%
% \medskip
%
% I've only written:
%
% \typestored{abstract}
%
% and
%
% \typestored{description}
%
% Of course, I didn't copy and paste. The real code they were written with is:
%
% \iffalse
%<*example>
% \fi
\begin{examplecode}[numbers=left]
\begin{scontents}[store-env=abstract,print-env=true]
\begin{abstract}
This package allows to store \hologo{LaTeX} code, including \enquote{\emph{verbatim}},
in \mymeta{sequences} using the \mypkg{l3seq} module of \mypkg{expl3}. The \mymeta{stored
content} can be used as many times as desired in the document, additionally you can write
to \mymeta{external files} or show it in \mymeta{verbatim style}.
\end{abstract}
\end{scontents}
\end{examplecode}
% \iffalse
%</example>
% \fi
%
% and
%
% \iffalse
%<*example>
% \fi
\begin{examplecode}[numbers=left]
\begin{scontents}[store-env=description, print-env=true]
The \mypkg*{scontents} package allows to \mymeta{store contents} in \mymeta{sequences}
or \mymeta{external files}. In some ways it is similar to the \mypkg{filecontentsdef}
package, with the difference in which the \mymeta{content} is stored. The idea behind
this package is to get an approach to \hologo{ConTeXt} \enquote{\emph{buffers}} by
making use \mymeta{sequences}.
\end{scontents}
\end{examplecode}
% \iffalse
%</example>
% \fi
%
% I stored the content in memory and then ran §\getstored§ and
% §\typestored§. This is one of the ways you can use \mypkg*{scontents}.
%
% \section{Examples}
%
% These are some adapted examples that have served as inspiration for
% the creation of this package. The examples are attached to this documentation
% and can be extracted from your PDF viewer or from the command line by running:
% \iffalse
%<*example>
% \fi
\begin{examplecode}[frame=single]
$ pdfdetach -saveall scontents.pdf
\end{examplecode}
% \iffalse
%</example>
% \fi
% and then you can use the excellent \arara\footnote{The cool \TeX\ automation tool:
% \url{https://www.ctan.org/pkg/arara}} tool to compile them.
%
% \subsection{From \texttt{answers} package}
%
% \subsubsection*{Example 1}
%
% \iffalse
%<*example>
% \fi
\begin{scontents}[write-out=scexamp1.ltx]
% arara: pdflatex
% arara: clean: { extensions: [ aux, log] }
\documentclass{article}
\usepackage[store-cmd=solutions]{scontents}
\newtheorem{ex}{Exercise}
\begin{document}
\section{Problems}
\begin{ex}
First exercise
\Scontents{First solution.}
\end{ex}

\begin{ex}
Second exercise
\Scontents{Second solution.}
\end{ex}

\section{Solutions}
\foreachsc[sep={\\[10pt]}]{solutions}
\end{document}
\end{scontents}
% \iffalse
%</example>
% \fi
%
% Adaptation of example 1 of the package \mypkg{answers}
% \textattachfile[color=linkcolor,print=false]{scexamp1.ltx}{\faFile*[regular]}.
% \lstinputlisting[language=scontents-doc,numbers=left]{scexamp1.ltx}
%
% \subsection{From \texttt{filecontentsdef} package}
%
% \subsubsection*{Example 2}
%
% \iffalse
%<*example>
% \fi
\begin{scontents}[write-out=scexamp2.ltx]
% arara: pdflatex
% arara: clean: { extensions: [ aux, log] }
\documentclass{article}
\usepackage[store-env=defexercise,store-cmd=defexercise]{scontents}
\pagestyle{empty}
\begin{document}
% not starred
\Scontents{
Prove that \[x^n+y^n=z^n\] is not solvable in positive integers if $n$ is at
most $-3$.\par
}
% starred
\Scontents*|Refute the existence of black holes in less than $140$ characters.|
% write environment to \jobname.txt
\begin{scontents}[write-env=\jobname.txt]
\def\NSA{NSA}%
Prove that factorization is easily done via probabilistic algorithms and
advance evidence from knowledge of the names of its employees in the
seventies that the \NSA\ has known that for 40 years.\par
\end{scontents}
% see all stored
\begin{itemize}
\foreachsc[before={\item }]{defexercise}
\end{itemize}
% \getstored are robust :)
\section{\getstored[2]{defexercise}}
\end{document}
\end{scontents}
% \iffalse
%</example>
% \fi
%
% Adaptation of example from package \mypkg{filecontentsdef}
% \textattachfile[color=linkcolor,print=false]{scexamp2.ltx}{\faFile*[regular]}.
% \lstinputlisting[language=scontents-doc,numbers=left]{scexamp2.ltx}
%
% \subsection{From TeX-SX}
%
% \subsubsection*{Example 3}
%
% \iffalse
%<*example>
% \fi
\begin{scontents}[write-out=scexamp3.ltx]
% arara: pdflatex
% arara: clean: { extensions: [ aux, log] }
\documentclass{article}
\usepackage[store-cmd=tikz]{scontents}
\usepackage{tikz}
\setlength{\parindent}{0pt}
\pagestyle{empty}
\Scontents*{\matrix{ \node (a) {$a$} ; & \node (b) {$b$} ; \\ } ;}
\Scontents*{\matrix[ampersand replacement=\&]
{ \node (a) {$a$} ; \& \node (b) {$b$} ; \\ } ;}
\Scontents*{\matrix{\node (a) {$a$} ; & \node (b) {$b$} ; \\ } ; }
\begin{document}
\section{tikzpicture}
\begin{tikzpicture}
\getstored[1]{tikz}
\end{tikzpicture}

\begin{tikzpicture}
\getstored[2]{tikz}
\end{tikzpicture}

\begin{tikzpicture}
\getstored[3]{tikz}
\end{tikzpicture}

\begin{scontents}[store-env=buffer]
Hello World!

This is a \verb*|fake poor man's buffer :)|.
\end{scontents}

\section{source tikz}
\typestored[1]{tikz}
\typestored[2]{tikz}
\typestored[3]{tikz}

\section{fake buffer}
\subsection{real content}
\getstored[1]{buffer}
\subsection{verbatim style}
\typestored[1]{buffer}
\subsection{meaning}
\meaningsc[1]{buffer}

\section{tikz again}
\foreachsc[before={\begin{tikzpicture}},after={\end{tikzpicture}},sep={\\[10pt]}]{tikz}
\end{document}
\end{scontents}
% \iffalse
%</example>
% \fi
%
% Adapted from \href{https://tex.stackexchange.com/q/5338/7832}{LaTeX equivalent of ConTeXt buffers}
% \textattachfile[color=linkcolor,print=false]{scexamp3.ltx}{\faFile*[regular]}.
% \lstinputlisting[language=scontents-doc,numbers=left]{scexamp3.ltx}
%
% \subsubsection*{Example 4}
%
% \iffalse
%<*example>
% \fi
\begin{scontents}[write-out=scexamp4.ltx]
% arara: pdflatex
% arara: clean: { extensions: [ aux, log] }
\documentclass{article}
\usepackage{scontents}
\pagestyle{empty}
\begin{document}
\begin{scontents}[store-env=main]
Something for main A.
\end{scontents}

\begin{scontents}[store-env=main]
Something for \verb|main B|.
\end{scontents}

\begin{scontents}[store-env=other]
Something for \verb|other|.
\end{scontents}

\textbf{Let's print them}

This is first stored in main: \getstored[1]{main}\par
This is second stored in main: \getstored[2]{main}\par
This is stored in other: \getstored[1]{other}

\textbf{Print all of stored in main}\par
\foreachsc[sep={\\[10pt]}]{a}
\end{document}
\end{scontents}
% \iffalse
%</example>
% \fi
%
% Adapted from \href{https://tex.stackexchange.com/q/184503/7832}{Collecting contents of environment and store them for later retrieval}
% \textattachfile[color=linkcolor,print=false]{scexamp4.ltx}{\faFile*[regular]}.
% \lstinputlisting[language=scontents-doc,numbers=left]{scexamp4.ltx}
%
% \subsubsection*{Example 5}
%
% \iffalse
%<*example>
% \fi
\begin{scontents}[write-out=scexamp5.ltx]
% arara: pdflatex
% arara: clean: { extensions: [ aux, log] }
\documentclass{article}
\usepackage{scontents}
\pagestyle{empty}
\setlength{\parindent}{0pt}
\begin{document}
\section{Problem stated the first time}
\begin{scontents}[print-env=true,store-env=problem]
This is normal text.
\verb|This is from the verb command.|
\verb*|This is from the verb* command.|
This is normal text.
\begin{verbatim}
This is from the verbatim environment:
&%{}~
\end{verbatim}
\end{scontents}
\section{Problem restated}
\getstored[1]{problem}
\section{Problem restated once more}
\getstored[1]{problem}
\end{document}
\end{scontents}
% \iffalse
%</example>
% \fi
%
% Adapted from \href{https://tex.stackexchange.com/q/373647/7832}{Collect contents of an environment (that contains verbatim content)}
% \textattachfile[color=linkcolor,print=false]{scexamp5.ltx}{\faFile*[regular]}.
% \lstinputlisting[language=scontents-doc,numbers=left]{scexamp5.ltx}
%
% \subsection{Customization of \texttt{verbatimsc}}
%
% \subsubsection*{Example 6}
%
% \iffalse
%<*example>
% \fi
\begin{scontents}[write-out=scexamp6.ltx]
\documentclass{article}
% arara: pdflatex
% arara: clean: { extensions: [ aux, log] }
\usepackage{scontents}
\makeatletter
\let\verbatimsc\@undefined
\let\endverbatimsc\@undefined
\makeatother
\usepackage{fvextra}
\usepackage{xcolor}
\definecolor{mygray}{gray}{0.9}
\usepackage{tcolorbox}
\newenvironment{verbatimsc}%
{\VerbatimEnvironment
\begin{tcolorbox}[colback=mygray, boxsep=0pt, arc=0pt, boxrule=0pt]
\begin{Verbatim}[fontsize=\scriptsize, breaklines, breakafter=*, breaksymbolsep=0.5em,
breakaftersymbolpre={\,\tiny\ensuremath{\rfloor}}]}%
{\end{Verbatim}%
\end{tcolorbox}}
\setlength{\parindent}{0pt}
\pagestyle{empty}
\begin{document}
\section{Test \texttt{\textbackslash begin\{scontents\}} whit \texttt{fancyvrb}}
Test \verb+{scontents}+ \par

\begin{scontents}
Using \verb+scontents+ env no \verb+[key=val]+, save in seq \verb+contents+
with index 1.

Prove new \Verb*{ fancyvrb whit braces } and environment \verb+Verbatim*+
\begin{verbatim}
 verbatim  environment
\end{verbatim}
\end{scontents}

\section{Test \texttt{\textbackslash Scontents} whit \texttt{fancyvrb}}
\Scontents{ We have coded this in \LaTeX: $E=mc^2$.}

\section{Test \texttt{\textbackslash getstored}}
\getstored[1]{contents}\par
\getstored[2]{contents}

\section{Test \texttt{\textbackslash meaningsc}}
\meaningsc[1]{contents}\par
\meaningsc[2]{contents}

\section{Test \texttt{\textbackslash typestored}}
\typestored[1]{contents}
\typestored[2]{contents}
\end{document}
\end{scontents}
% \iffalse
%</example>
% \fi
%
% Customization of \myenv*{verbatimsc} using the \mypkg{fancyvrb} and
% \mypkg{tcolorbox} package
% \textattachfile[color=linkcolor,print=false]{scexamp6.ltx}{\faFile*[regular]}.
% \lstinputlisting[language=scontents-doc,numbers=left]{scexamp6.ltx}
%
% \subsubsection*{Example 7}
%
% \iffalse
%<*example>
% \fi
\begin{scontents}[write-out=scexamp7.ltx]
% arara: pdflatex
% arara: clean: { extensions: [ aux, log] }
\documentclass{article}
\usepackage{scontents}
\makeatletter
\let\verbatimsc\@undefined
\let\endverbatimsc\@undefined
\makeatother
\usepackage{xcolor}
\usepackage{listings}
\lstnewenvironment{verbatimsc}
  {
   \lstset{
           basicstyle=\small\ttfamily,
           breaklines=true,
           columns=fullflexible,
           language=[LaTeX]TeX,
           numbers=left,
           numbersep=1em,
           numberstyle=\tiny\color{gray},
           keywordstyle=\color{red}
          }
  }{}
\setlength{\parindent}{0pt}
\pagestyle{empty}
\begin{document}
\section{Test \texttt{\textbackslash begin\{scontents\}} whit \texttt{listings}}
Test \verb+{scontents}+ \par

\begin{scontents}
Using \verb+scontents+ env no \verb+[key=val]+, save in seq \verb+contents+ with index 1.\par

Prove \lstinline[basicstyle=\ttfamily]| lstinline | and environment \verb+Verbatim*+
\begin{verbatim}
  verbatim  environment
\end{verbatim}
\end{scontents}

\section{Test \texttt{\textbackslash Scontents*} whit \texttt{listings}}

\Scontents*+ We have coded this in \lstinline[basicstyle=\ttfamily]|\LaTeX: $E=mc^2$|
and more.+

\section{Test \texttt{\textbackslash getstored}}
\getstored[2]{contents}\par
\getstored[1]{contents}

\section{Test \texttt{\textbackslash typestored}}
\typestored[1]{contents}
\typestored[2]{contents}
\end{document}
\end{scontents}
% \iffalse
%</example>
% \fi
%
% Customization of \myenv*{verbatimsc} using the \mypkg{listings} package
% \textattachfile[color=linkcolor,print=false]{scexamp7.ltx}{\faFile*[regular]}.
% \lstinputlisting[language=scontents-doc,numbers=left]{scexamp7.ltx}
%
% \subsubsection*{Example 8}
%
% \iffalse
%<*example>
% \fi
\begin{scontents}[write-out=scexamp8.ltx]
% arara: xelatex : {shell: true, options: [-8bit]}
% arara: clean: { extensions: [ aux, log] }
\documentclass{article}
\usepackage{scontents}
\makeatletter
\let\verbatimsc\@undefined
\let\endverbatimsc\@undefined
\makeatother
\usepackage{minted}
\newminted{tex}{linenos}
\newenvironment{verbatimsc}{\VerbatimEnvironment\begin{texcode}}{\end{texcode}}
\pagestyle{empty}
\begin{document}
\section{Test \texttt{\textbackslash begin\{scontents\}} whit \texttt{minted}}
Test \verb+{scontents}+ \par

\begin{scontents}[overwrite,write-env=\jobname.tsc,force-eol=true]
Using \verb+scontents+ env no \verb+[key=val]+, save in seq \verb+contents+
with index 1.\par

Prove new \Verb*{ new fvextra whit braces } and environment \verb+Verbatim*+
\begin{Verbatim}[obeytabs, showtabs, tab=\rightarrowfill, tabcolor=red]
No tab
	One real tab
		Two real Tab plus 	one tab
\end{Verbatim}
\end{scontents}

\section{See \Verb{\jobname.tsc}}
Read \Verb{\jobname.tsc} (shows TABs as red arrows):
\VerbatimInput[obeytabs, showtabs, tab=\rightarrowfill, tabcolor=red]{\jobname.tsc}

\section{Test \texttt{\textbackslash Scontents} whit \texttt{minted}}

\Scontents{ We have coded \par this in \LaTeX: $E=mc^2$.}

\section{Test \texttt{\textbackslash getstored}}
\getstored[1]{contents}\par
\getstored[2]{contents}

\section{Test \texttt{\textbackslash typestored}}
\typestored[1]{contents}
\typestored[2]{contents}
\end{document}
\end{scontents}
% \iffalse
%</example>
% \fi
%
% Customization of \myenv*{verbatimsc} using the \mypkg{minted} package
% \textattachfile[color=linkcolor,print=false]{scexamp8.ltx}{\faFile*[regular]}.
% \lstinputlisting[language=scontents-doc,numbers=left]{scexamp8.ltx}
%
% \newpage
%
% \section{Change history}
% \label{sec:changes}
%
% In this section you will find some (not all) of the changes in \mypkg*{scontents}
% development, from the first public implementation using the \mypkg{filecontentsdef}
% package to the current version with only \mypkg{expl3}.
%
% \medskip
%
% \setlist[itemize,1]{label=\textendash,wide=0.5em,nosep,noitemsep,leftmargin=10pt}
% \newlength\descrwidth
% \settowidth{\descrwidth}{\textsf{v1.0, (ctan), 2019-07-30} }
%
% \begin{description}[font=\small\sffamily,wide=0pt,style=multiline,leftmargin=\descrwidth,nosep,noitemsep]
% \item [\fileversion{} (devp), \filedate]
%    \begin{itemize}
%    \item Add |overwrite| key to reduce |I/O| operations.
%    \item Deleted an unnecessary group in the code.
%    \end{itemize}
% \item [v1.7 (ctan), 2019-10-29]
%    \begin{itemize}
%    \item The |verbatimsc| environment was rewritten.
%    \item Minor adjustments in documentation.
%    \end{itemize}
% \item [v1.6 (ctan), 2019-10-26]
%    \begin{itemize}
%    \item The internal behavior of |\getstored| has been modified.
%    \item The internal behavior of |\foreachsc| has been modified.
%    \item Corrected file extension for \hologo{ConTeXt}.
%    \item Remove spurious warning.
%    \end{itemize}
% \item [v1.5 (ctan), 2019-10-24]
%    \begin{itemize}
%        \item Add support for \hologo{plainTeX} and \hologo{ConTeXt}.
%        \item Split internal code for optimization.
%        \item Add support for vertical spaces in |key=val|.
%        \item Add |\foreachsc| command.
%        \item Check if |verbatim| package is loaded.
%    \end{itemize}
% \item [v1.4 (ctan), 2019-10-03]
%    \begin{itemize}
%        \item Add |store-all| key.
%        \item Messages and keys were separated.
%        \item Restructuring of documentation.
%        \item Now the version of |expl3| is checked instead of |xparse|.
%        \item The internal behavior of |force-eol| has been modified.
%    \end{itemize}
% \item [v1.3 (ctan), 2019-09-24]
%    \begin{itemize}
%        \item The environment can now nest.
%        \item Added |force-eol|, |verb-font| and |width-tab| keys.
%        \item The extra space has been removed when you run |\getstored|.
%        \item Internal code has been rewritten more efficiently.
%        \item Remove |\typestored|.
%        \item Remove |filecontentsdef| dependency.
%        \item Changing |\regex_replace_all:| for |\tl_replace_all:|.
%    \end{itemize}
% \item [v1.2 (ctan), 2019-08-28]
%    \begin{itemize}
%        \item Restructuring of documentation.
%        \item Added copy of |\tex_scantokens:|.
%    \end{itemize}
% \item [v1.1 (ctan), 2019-08-12]
%    \begin{itemize}
%        \item Extension of documentation.
%        \item Replace |\tex_endinput:D| by |\file_input_stop:|.
%    \end{itemize}
% \item [v1.0 (ctan), 2019-07-30]
%    \begin{itemize}
%        \item First public release.
%    \end{itemize}
% \end{description}
%
% \newpage
%
% \indexprologue{
% The italic numbers denote the pages where the corresponding entry is
% described.}
%
% \printindex[userdoc]
%
% \newpage
%
% \StartImplementation
%
% \StopEventually{^^A
% \newgeometry{top=0.5in,bottom=0.3in,left=1.0in,right=0.5in,footskip=0.2in,headheight=1cm,headsep=0.27cm}
% \addtocontents{toc}{\protect\setcounter{tocdepth}{2}}
% \cleardoublepage
% \phantomsection
% \indexprologue{
% The italic numbers denote the pages where the corresponding entry is
% described, the numbers underlined and all others indicate the line on
% which they are implemented in the package code.
% }
% \printindex
% }
%
%
% \section{Implementation}
% \label{sec:Implementation}
% \addtocontents{toc}{\protect\setcounter{tocdepth}{0}}
%
% The most recent publicly released version of \mypkg*{scontents} is available at
% \textsc{ctan}: \url{https://www.ctan.org/pkg/scontents}. Historical and
% developmental versions are available at \textcolor{gray}{\scriptsize\faIcon[regular]{github}}
% \url{https://github.com/pablgonz/scontents}. While general feedback via email is
% welcomed, specific bugs or feature requests should be reported through the issue
% tracker: \url{https://github.com/pablgonz/scontents/issues}.
%
% \subsection{Declaration of the package}
%
% First we set up the module name for \pkg{l3doc}:
%    \begin{macrocode}
%<@@=scontents>
%    \end{macrocode}
%
% Now we define some common macros to hold the package date and version:
%    \begin{macrocode}
%<loader>\def\ScontentsFileDate{2019-12-25}%
%<core>\def\ScontentsCoreFileDate{2019-12-25}%
%<*loader>
\def\ScontentsFileVersion{1.8}%
\def\ScontentsFileDescription{Stores LaTeX contents in memory or files}%
%    \end{macrocode}
%
% The \LaTeX{} loader is fairly simple: just load the dependencies, load the
% core code, and then set interfaces up.
%
% We also check if the \pkg{verbatim} package is loaded and show a compatibility
% warning.
%    \begin{macrocode}
%<*latex>
\RequirePackage{expl3,xparse,l3keys2e}[2019/05/28]
\ProvidesExplPackage
  {scontents} {\ScontentsFileDate} {\ScontentsFileVersion} {\ScontentsFileDescription}
\@ifpackageloaded { verbatim }
  {
    \iow_term:n
      {
        The~implementation~of~the~‘verbatimsc’~environment~\\
        used~by~\tl_to_str:n{\typestored}~is~not~compatible~with~package\\
        ‘verbatim’.~Review~the~documentation~and~redefine~\\
        the~‘verbatimsc’~environment.
      }
  } { }
%</latex>
%    \end{macrocode}
% The Plain and \hologo{ConTeXt} loaders are similar (probably because I don't
% know how to make a proper \hologo{ConTeXt} module :-). We define a
% \LaTeX{}-style §\ver@scontents.sty§ macro with version info (just in case):
%    \begin{macrocode}
%<*!latex>
%<context>\writestatus{loading}{User Module scontents v\ScontentsFileVersion}
%<context>\unprotect
\input expl3-generic.tex
\ExplSyntaxOn
\tl_gset:cx { ver @ scontents . sty } { \ScontentsFileDate\space
  v\ScontentsFileVersion\space \ScontentsFileDescription }
\iow_log:x { Package: ~ scontents ~ \use:c { ver @ scontents . sty } }
%</!latex>
%    \end{macrocode}
%
% In Plain, check that the package isn't being loaded twice (\LaTeX{} and
% \hologo{ConTeXt} already defend against that):
%    \begin{macrocode}
%<*plain>
\msg_gset:nnn { scontents } { already-loaded }
  { The~‘scontents’~package~is~already~loaded.~Aborting~input~\msg_line_context:. }
\cs_if_exist:NT \@@_rescan_tokens:n
  {
    \msg_warning:nn { scontents } { already-loaded }
    \ExplSyntaxOff
    \file_input_stop:
  }
%</plain>
%    \end{macrocode}
%
% \begin{macro}{\g_@@_end_verbatimsc_tl,\c_@@_end_env_tl}
%   A token list to match when ending \env{verbatimsc} and \env{scontents}
%   environments.
%
%    \begin{macrocode}
\tl_new:N \g_@@_end_verbatimsc_tl
\tl_gset_rescan:Nnn
  \g_@@_end_verbatimsc_tl
  {
    \char_set_catcode_other:N \\
%<*latex>
    \char_set_catcode_other:N \{
    \char_set_catcode_other:N \}
%</latex>
  }
%<latex>  { \end{verbatimsc} }
%<plain>  { \endverbatimsc }
%<context>  { \stopverbatimsc }
\tl_const:Nx \c_@@_end_env_tl
  {
    \c_backslash_str
%<latex|plain>    end
%<context>    stop
%<latex>    \c_left_brace_str
      scontents
%<latex>    \c_right_brace_str
  }
%    \end{macrocode}
% \end{macro}
%
% Now we load the core \mypkg*{scontents} code:
%    \begin{macrocode}
\file_input:n { scontents-code.tex }
%    \end{macrocode}
%
% Sometimes we need to detect the format from within a macro:
%    \begin{macrocode}
\cs_new:Npn \@@_format_case:nnn #1 #2 #3
%<latex>  {#1} % LaTeX
%<plain>  {#2} % Plain/Generic
%<context>  {#3} % ConTeXt
%    \end{macrocode}
%
% Checking that the package was loaded with the proper loader code. This code
% was copied from \texttt{expl3-code.tex}.
%    \begin{macrocode}
%</loader>
%<*core>
\begingroup
  \def\next{\endgroup}%
  \expandafter\ifx\csname PackageError\endcsname\relax
    \begingroup
      \def\next{\endgroup\endgroup}%
      \def\PackageError#1#2#3%
        {%
          \endgroup
          \errhelp{#3}%
          \errmessage{#1 Error: #2!}%
        }%
  \fi
  \expandafter\ifx\csname ScontentsFileDate\endcsname\relax
    \def\next
      {%
        \PackageError{scontents}{No scontents loader detected}
          {%
            You have attempted to use the scontents code directly rather than using
            the correct loader. Loading of scontents will abort.
          }%
        \endgroup
        \endinput
      }
  \else
    \ifx\ScontentsFileDate\ScontentsCoreFileDate
    \else
      \def\next
        {%
          \PackageError{scontents}{Mismatched~scontents~files~detected}
            {%
              You~have~attempted~to~load~scontents~with~mismatched~files:~
              probably~you~have~one~or~more~files~‘locally~installed’~which~
              are~in~conflict.~Loading~of~scontents~will~abort.
            }%
          \endgroup
          \endinput
        }%
    \fi
\fi
\next
%    \end{macrocode}
%
% \subsection{Definition of common keys}
%
% We create some common \mymeta{keys} that will be used by the options passed to
% the package as well as by the environments and commands defined.
%
%    \begin{macrocode}
\keys_define:nn { scontents }
  {
    store-env .tl_set:N         = \l_@@_name_seq_env_tl,
    store-env .initial:n        = contents,
    store-env .value_required:n = true,
    store-cmd .tl_set:N         = \l_@@_name_seq_cmd_tl,
    store-cmd .initial:n        = contents,
    store-cmd .value_required:n = true,
    verb-font .tl_set:N         = \l_@@_verb_font_tl,
    verb-font .value_required:n = true,
    print-env .bool_set:N       = \l_@@_print_env_bool,
    print-env .initial:n        = false,
    print-env .default:n        = true,
    print-cmd .bool_set:N       = \l_@@_print_cmd_bool,
    print-cmd .initial:n        = false,
    print-cmd .default:n        = true,
    force-eol .bool_set:N       = \l_@@_forced_eol_bool,
    force-eol .initial:n        = false,
    force-eol .default:n        = true,
    overwrite .bool_set:N       = \l_@@_overwrite_bool,
    overwrite .initial:n        = false,
    overwrite .default:n        = true,
    width-tab .int_set:N        = \l_@@_tab_width_int,
    width-tab .initial:n        = 1,
    width-tab .value_required:n = true,
    print-all .meta:n           = { print-env = #1 , print-cmd = #1 },
    print-all .default:n        = true,
    store-all .meta:n           = { store-env = #1 , store-cmd = #1 },
    store-all .value_required:n = true
  }
%</core>
%<loader>\keys_define:nn { scontents }
%<latex>  { verb-font .initial:n = \ttfamily }
%<plain|context>  { verb-font .initial:n = \tt }
%    \end{macrocode}
%
% In \LaTeX{} mode we load \mypkg{l3keys2e} process the \mymeta{keys} as options
% passed on to the package, the package \pkg{l3keys2e} will verify the
% \mymeta{keys} and will return an error when they are \emph{unknown}.
%    \begin{macrocode}
%<latex>\ProcessKeysOptions { scontents }
%<*core>
%    \end{macrocode}
%
% \subsection{Internal variables}
%
% Now we declare the internal variables we will use.
%
% \begin{macro}{\l_@@_macro_tmp_tl,\l_@@_fname_out_tl,\l_@@_temp_tl,
%               \l_@@_file_tl,\g_@@_temp_tl,\l_@@_foreach_name_seq_tl,
%               \l_@@_foreach_before_tl,\l_@@_foreach_after_tl}
%   \cs{l_@@_macro_tmp_tl} is a temporary token list to hold the contents
%   of the macro/environment, \cs{l_@@_fname_out_tl} is used as the name
%   of the output file, when there's one, \cs{l_@@_file_tl} holds the contents
%   of an environment as it's being read, and \cs{l_@@_temp_tl} and
%   \cs{g_@@_temp_tl} are generic temporary token lists.
%
%   \cs{l_@@_foreach_name_seq_tl} is the name assigned to the sequence on
%   which the loop will be made, \cs{l_@@_foreach_before_tl} and \cs{l_@@_foreach_after_tl}
%   are token lists in which the assigned material will be placed before
%   and after the execution of the §\foreachsc§ loop.
%
%    \begin{macrocode}
\tl_new:N \l_@@_macro_tmp_tl
\tl_new:N \l_@@_fname_out_tl
\tl_new:N \l_@@_temp_tl
\tl_new:N \l_@@_file_tl
\tl_new:N \g_@@_temp_tl
\tl_new:N \l_@@_foreach_name_seq_tl
\tl_new:N \l_@@_foreach_before_tl
\tl_new:N \l_@@_foreach_after_tl
%    \end{macrocode}
% \end{macro}
%
% \begin{macro}{\l_@@_seq_item_int,\l_@@_env_nesting_int,\l_@@_tmpa_int,
%    \l_@@_foreach_stop_int}
%   \cs{l_@@_seq_item_int} stores the index in the sequence of the item
%   requested to \cs{typestored} or \cs{meaningsc}. \cs{l_@@_env_nesting_int}
%   stores the current nesting level of the \env{scontents} environment.
%  \cs{l_@@_foreach_stop_int} will save the value at which the §\foreachsc§
%  loop will stop.
%
%    \begin{macrocode}
\int_new:N \l_@@_foreach_stop_int
\int_new:N \l_@@_seq_item_int
\int_new:N \l_@@_env_nesting_int
\int_new:N \l_@@_tmpa_int
%    \end{macrocode}
% \end{macro}
%
% \begin{macro}{\l_@@_writing_bool,\l_@@_storing_bool}
%   The boolean \cs{l_@@_writing_bool} keeps track of whether we should write
%   to a file, and \cs{l_@@_storing_bool} determines whether it is in write-only
%   mode when the §write-out§ option is used.
%
%    \begin{macrocode}
\bool_new:N \l_@@_writing_bool
\bool_set_false:N \l_@@_writing_bool
\bool_new:N \l_@@_storing_bool
\bool_set_true:N  \l_@@_storing_bool
%    \end{macrocode}
% \end{macro}
%
%
% \begin{macro}{\l_@@_foreach_before_bool,\l_@@_foreach_after_bool,
%               \l_@@_foreach_stop_bool,\l_@@_foreach_wrapper_bool}
%   Boolean variables used by the §\foreachsc§ loop.
%
%    \begin{macrocode}
\bool_new:N \l_@@_foreach_before_bool
\bool_set_false:N \l_@@_foreach_before_bool
\bool_new:N \l_@@_foreach_after_bool
\bool_set_false:N \l_@@_foreach_after_bool
\bool_new:N \l_@@_foreach_stop_bool
\bool_set_false:N \l_@@_foreach_stop_bool
\bool_new:N \l_@@_foreach_wrapper_bool
\bool_set_false:N \l_@@_foreach_wrapper_bool
%    \end{macrocode}
% \end{macro}
%
% \begin{macro}{\l_@@_foreach_print_seq}
%   The \cs{l_@@_foreach_print_seq} is the sequence used by §\foreachsc§.
%
%    \begin{macrocode}
\seq_new:N \l_@@_foreach_print_seq
%    \end{macrocode}
% \end{macro}
%
% \begin{macro}{\c_@@_hidden_space_str}
%   \cs{c_@@_hidden_space_str} is a constant \emph{string} to used to hide
%   the \mymeta{forced space} added by \hologo{TeX} when recording content
%   in a macro. This \emph{string} contains the \emph{reserved phrase}
%   ``§%§§^^Ascheol%§'' which is added to the end of the argument stored
%   in |seq| when the key §force-eol§ is false.
%
%    \begin{macrocode}
\str_const:Nx \c_@@_hidden_space_str
  { \c_percent_str \c_circumflex_str \c_circumflex_str A scheol \c_percent_str }
%    \end{macrocode}
% \end{macro}
%
% \begin{macro}{\q_@@_stop,\q_@@_mark}
%   Some quarks used along the code as macro delimiters.
%
%    \begin{macrocode}
\quark_new:N \q_@@_stop
\quark_new:N \q_@@_mark
%    \end{macrocode}
% \end{macro}
%
% \begin{macro}{\l_@@_file_iow}
%   An output stream for saving the contents of an environment to a file.
%
%    \begin{macrocode}
\iow_new:N \l_@@_file_iow
%    \end{macrocode}
% \end{macro}
%
% \begin{macro}{\@@_rescan_tokens:n}
%   \cs{tl_rescan:nn} doesn't fit the needs of this package because
%   it does not allow catcode changes inside the argument, so verbatim
%   commands used inside one of \mypkg*{scontents}'s commands/environments
%   will not work. Here we create a private copy of \cs{tex_scantokens:D}
%   which will serve our purposes.
%
%    \begin{macrocode}
\cs_new_protected:Npn \@@_rescan_tokens:n #1 { \tex_scantokens:D {#1} }
\cs_generate_variant:Nn \@@_rescan_tokens:n { V, x }
%    \end{macrocode}
% \end{macro}
%
% \begin{macro}{\@@_tab:,\@@_par:}
%   Control sequences to replace tab (§^^I§) and form feed (§^^L§) characters.
%
%    \begin{macrocode}
\cs_new:Npx \@@_tab: { \c_space_tl }
\cs_new:Npn \@@_par: { ^^J ^^J }
%    \end{macrocode}
% \end{macro}
%
% \begin{macro}[int]{\tl_remove_once:NV,\tl_replace_all:Nxx,
%     \tl_replace_all:Nxn,\tl_replace_all:Nnx,\tl_if_empty:fTF}
%   Some nonstandard variants.
%
%    \begin{macrocode}
\cs_generate_variant:Nn \tl_remove_once:Nn { NV }
\cs_generate_variant:Nn \tl_replace_all:Nnn { Nx, Nxx, Nnx }
\cs_generate_variant:Nn \msg_error:nnnn { nnx }
\prg_generate_conditional_variant:Nnn \tl_if_empty:n { f } { TF }
%    \end{macrocode}
% \end{macro}
%
% \subsection{Defining keys for the environment and commands}
%
% We add the \mymeta{keys} divided into subgroups to handle errors and
% \emph{unknown} \mymeta{keys} separately.
%
% \subsubsection{Keys for environment \env{scontents}}
%
% We define a set of \mymeta{keys} for environment §scontents§.
%
%    \begin{macrocode}
\keys_define:nn { scontents / scontents }
  {
    write-env .code:n           = {
                                    \bool_set_true:N \l_@@_writing_bool
                                    \tl_set:Nn \l_@@_fname_out_tl {#1}
                                  },
    write-out .code:n           = {
                                    \bool_set_false:N \l_@@_storing_bool
                                    \bool_set_true:N  \l_@@_writing_bool
                                    \tl_set:Nn \l_@@_fname_out_tl {#1}
                                  },
    write-env .value_required:n = true,
    write-out .value_required:n = true,
    print-env .meta:nn          = { scontents } { print-env = #1 },
    print-env .default:n        = true,
    store-env .meta:nn          = { scontents } { store-env = #1 },
    force-eol .meta:nn          = { scontents } { force-eol = #1 },
    force-eol .default:n        = true,
    overwrite .meta:nn          = { scontents } { overwrite = #1 },
    overwrite .default:n        = true,
    unknown   .code:n           = { \@@_parse_environment_keys:n {#1} }
  }
%    \end{macrocode}
%
% \subsubsection{Keys for command \cs{Scontents}}
%
% We define a set of \mymeta{keys} for commands §\Scontents§ and
% §\Scontents*§.
%
%    \begin{macrocode}
\keys_define:nn { scontents / Scontents }
  {
    print-cmd .meta:nn   = { scontents } { print-cmd = #1 },
    print-cmd .default:n = true,
    store-cmd .meta:nn   = { scontents } { store-cmd = #1 },
    force-eol .meta:nn   = { scontents } { force-eol = #1 },
    force-eol .default:n = true,
    unknown   .code:n    = { \@@_parse_command_keys:n {#1} }
  }
%    \end{macrocode}
%
% \subsubsection{Keys for command \cs{foreachsc}}
%
% We define a set of \mymeta{keys} for command §\foreachsc§.
%
%    \begin{macrocode}
\keys_define:nn { scontents / foreachsc }
  {
    before  .code:n           = {
                                  \bool_set_true:N \l_@@_foreach_before_bool
                                  \tl_set:Nn \l_@@_foreach_before_tl {#1}
                                },
    before  .value_required:n = true,
    after   .code:n           = {
                                  \bool_set_true:N \l_@@_foreach_after_bool
                                  \tl_set:Nn \l_@@_foreach_after_tl {#1}
                                },
    after   .value_required:n = true,
    start   .int_set:N        = \l_@@_foreach_start_int,
    start   .value_required:n = true,
    start   .initial:n        = 1,
    stop    .code:n           = {
                                  \bool_set_true:N \l_@@_foreach_stop_bool
                                  \int_set:Nn \l_@@_foreach_stop_int {#1}
                                 },
    stop    .value_required:n = true,
    step    .int_set:N        = \l_@@_foreach_step_int,
    step    .value_required:n = true,
    step    .initial:n        = 1,
    wrapper .code:n           = {
                                  \bool_set_true:N \l_@@_foreach_wrapper_bool
                                  \cs_set_protected:Npn
                                    \@@_foreach_wrapper:n ##1 {#1}
                                },
    wrapper .value_required:n = true,
    sep     .tl_set:N         = \l_@@_foreach_sep_tl,
    sep     .initial:n        = {},
    sep     .value_required:n = true,
    unknown .code:n           = { \@@_parse_foreach_keys:n {#1} }
  }
%    \end{macrocode}
%
% \subsubsection{Key for commands \cs{typestored} and \cs{meaningsc}}
%
% We define a \mymeta{key} for command §\typestored§ and §\meaningsc§.
% Both commands accept the same type of optional arguments, just define
% a common \mymeta{key}.
%
%    \begin{macrocode}
\keys_define:nn { scontents / typemeaning }
  {
    width-tab .meta:nn = { scontents } { width-tab = #1 },
    unknown   .code:n  = { \@@_parse_type_meaning_key:n {#1} }
  }
%    \end{macrocode}
%
% \subsection{Handling undefined keys}
%
% The \mymeta{keys} are stored in the token list variable \cs{l_keys_key_tl},
% and the value (if any) is passed as an argument to each \mymeta{function}.
%
% \subsubsection{Undefined keys for environment \env{scontents}}
%
% \begin{macro}{\@@_parse_environment_keys:n,
%               \@@_parse_environment_keys:nn}
%
%   We check the \mymeta{keys} passed to the environment
%   §scontents§ and process it with \cs{@@_parse_environment_keys:n}
%   if the \mymeta{key} is \emph{unknown} we return an error message.
%
%    \begin{macrocode}
\cs_new_protected:Npn \@@_parse_environment_keys:n #1
  { \exp_args:NV \@@_parse_environment_keys:nn \l_keys_key_tl {#1} }
\cs_new_protected:Npn \@@_parse_environment_keys:nn #1#2
  {
    \tl_if_blank:nTF {#2}
      { \msg_error:nnn { scontents } { env-key-unknown } {#1} }
      { \msg_error:nnnn { scontents } { env-key-value-unknown } {#1} {#2} }
  }
%    \end{macrocode}
% \end{macro}
%
% \subsubsection{Undefined keys for \cs{Scontents} and \cs{Scontents*}}
%
% \begin{macro}{\@@_parse_command_keys:n,
%               \@@_parse_command_keys:nn}
%
%   We check the \meta{keys} passed to commands §\Scontents§ or
%   §\Scontents*§ and process it with \cs{@@_parse_command_keys:n}
%   if the \mymeta{key} is \emph{unknown} we return an error message.
%
%    \begin{macrocode}
\cs_new_protected:Npn \@@_parse_command_keys:n #1
  { \exp_args:NV \@@_parse_command_keys:nn \l_keys_key_tl {#1} }
\cs_new_protected:Npn \@@_parse_command_keys:nn #1#2
  {
    \tl_if_blank:nTF {#2}
      { \msg_error:nnn { scontents } { cmd-key-unknown } {#1} }
      { \msg_error:nnnn { scontents } { cmd-key-value-unknown } {#1} {#2} }
  }
%    \end{macrocode}
% \end{macro}
%
% \subsubsection{Undefined keys for \cs{foreachsc}}
%
% \begin{macro}{\@@_parse_foreach_keys:n,
%               \@@_parse_foreach_keys:nn}
%
%   We check the \meta{keys} passed to command §\foreachsc§ and process
%   it with \cs{@@_parse_foreach_keys:n}, if the \mymeta{key} is
%   \emph{unknown} we return an error message.
%
%    \begin{macrocode}
\cs_new_protected:Npn \@@_parse_foreach_keys:nn #1#2
  {
    \tl_if_blank:nTF {#2}
      { \msg_error:nnn { scontents } { for-key-unknown } {#1} }
      { \msg_error:nnnn { scontents } { for-key-value-unknown } {#1} {#2} }
  }
\cs_new_protected:Npn \@@_parse_foreach_keys:n #1
  { \exp_args:NV \@@_parse_foreach_keys:nn \l_keys_key_tl {#1} }
%    \end{macrocode}
% \end{macro}
%
% \subsubsection{Undefined keys for \cs{typestored} and \cs{meaningsc}}
%
% \begin{macro}{\@@_parse_type_meaning_key:n,
%               \@@_parse_type_meaning_key:nn}
%
%   The commands §\typestored§ and §\meaningsc§ accept an optional
%   argument for setting the §width-tab§ to print the stored
%   contents. However their optional argument also contains the number
%   of the item to retrieve from the stored sequence. To avoid the
%   awkward §\typestored[]§\myoarg[type=tt]{options}§{...}§ syntax, we'll make the
%   commands have a single optional argument which is processed by
%   \pkg{l3keys}, and the unknown keys are brought here to
%   \cs{@@_parse_typemeaning_key:n} to process.
%
%   First we check if the \mymeta{key} is an integer using \cs{int_to_roman:n}.
%   If it is, we check that the value passed to the key is blank
%   (otherwise something odd as §1=1§ might have been used). If everything
%   is correct, then set the value of the integer which holds the \mymeta{index}.
%   Otherwise raise an error about an \emph{unknown} option.
%
%    \begin{macrocode}
\cs_new_protected:Npn \@@_parse_type_meaning_key:n #1
  { \exp_args:NV \@@_parse_type_meaning_key:nn \l_keys_key_tl {#1} }
\cs_new_protected:Npn \@@_parse_type_meaning_key:nn #1#2
  {
    \tl_if_empty:fTF { \int_to_roman:n { -0 #1 } }
      {
        \tl_if_blank:nTF {#2}
          { \int_set:Nn \l_@@_seq_item_int {#1} }
          { \msg_error:nnnn { scontents } { type-key-value-unknown } {#1} {#2} }
      }
      {
        \tl_if_blank:nTF {#2}
          { \msg_error:nnn { scontents } { type-key-unknown } {#1} }
          { \msg_error:nnnn { scontents } { type-key-value-unknown } {#1} {#2} }
      }
  }
%    \end{macrocode}
% \end{macro}
%
% \subsection{Compatibility layer with Plain}
%
% When loading the package outside of \LaTeX{} we can't usually use \pkg{xparse}.
% However since \pkg{xparse} doesn't actually hold any dependency with \LaTeX{}
% except for package-loading commands, we can emulate those commands (much like
% in \pkg{miniltx}) so that \pkg{xparse} is loadable in any format.
%
% The bunch of macros below is adapted from the \LaTeX{} kernel (greatly simplified).
%  \begin{macrocode}
%</core>
%<*loader&!latex>
\seq_new:N \l_@@_compat_seq
\cs_new_protected:Npn \@@_compat_redefine:Npn #1
  {
    \seq_put_right:Nn \l_@@_compat_seq {#1}
    \cs_set_eq:cN { @@_saved_\cs_to_str:N #1: } #1
    \cs_new_protected:Npn #1
  }
\cs_new_protected:Npn \@@_compat_restore:
  { \seq_map_function:NN \l_@@_compat_seq \@@_compat_restore:N }
\cs_new_protected:Npn \@@_compat_restore:N #1
  {
    \cs_set_eq:Nc #1 { @@_saved_\cs_to_str:N #1: }
    \cs_undefine:c { @@_saved_\cs_to_str:N #1: }
  }
\cs_generate_variant:Nn \@@_compat_redefine:Npn { c }
\cs_new_protected:Npn \@@_optarg:nn #1 #2
  { \peek_charcode_ignore_spaces:NTF [ {#1} {#1[#2]} }
\cs_new_protected:Npn \@@_stararg:nn #1 #2
  { \peek_charcode_remove_ignore_spaces:NTF * {#1} {#2} }
\@@_compat_redefine:Npn \RequirePackage
  { \@@_optarg:nn { \@@_require_auxi:wn } { } }
\cs_new_protected:Npn \@@_require_auxi:wn [#1] #2
  { \@@_optarg:nn { \@@_require_auxii:wnw [{#1}]{#2} } { } }
\cs_new:Npn \@@_zap_space:ww #1~#2
  {
    #1 \if_meaning:w #2 \q_mark
      \exp_after:wN \use_none:n
    \else:
      \exp_after:wN \@@_zap_space:ww
    \fi: #2
  }
\cs_new_protected:Npn \@@_require_auxii:wnw [#1] #2 [#3]
  {
    \tl_set:Nx \l_@@_temp_tl { \@@_zap_space:ww #2 ~ \q_mark }
    \clist_map_function:NN \l_@@_temp_tl \@@_require_auxiii:n
  }
\cs_new_protected:Npn \@@_require_auxiii:n #1
  {
    \str_if_eq:eeF {expl3} {#1}
      { \msg_error:nnn { scontents } { invalid-package } {#1} }
  }
\msg_new:nnn { scontents } { invalid-package }
  { Package~‘#1’~invalid~in~scontents.~This~is~an~error~in~scontents. }
\@@_compat_redefine:cpn { @ifpackagelater } #1
  { \exp_args:Nc \@@_package_later_aux:Nn { ver@#1.sty } }
\cs_new_protected:Npn \@@_package_later_aux:Nn #1 #2
  {
    \int_compare:nNnTF
      { \exp_after:wN \@@_parse_version:w #1 //00 \q_mark } <
      { \exp_after:wN \@@_parse_version:w #2 //00 \q_mark }
  }
\cs_new:Npn \@@_parse_version:w #1 { \@@_parse_version_auxi:w 0#1 }
\cs_new:Npn \@@_parse_version_auxi:w #1/#2/#3#4#5 \q_mark
  { \@@_parse_version_auxii:w #1-#2-#3#4 \q_mark }
\cs_new:Npn \@@_parse_version_auxii:w #1-#2-#3#4#5 \q_mark
  { \tl_if_blank:nF {#2} {#1} #2 #3 #4 }
\@@_compat_redefine:Npn \ProvidesExplPackage #1 #2 #3 #4
  { \@@_provides_aux:nn {#1} { #2 \ifx\relax#3\relax\else v#3\space\fi #4 } }
\cs_new_protected:Npn \@@_provides_aux:nn #1 #2
  {
    \tl_gset:cx { ver@#1.sty } {#2}
    \iow_log:n { Package~#1:~#2 }
    \ExplSyntaxOn
  }
\@@_compat_redefine:Npn \DeclareOption
  { \@@_stararg:nn { \use_none:n } { \use_none:nn } }
\@@_compat_redefine:Npn \ProcessOptions
  { \@@_stararg:nn { } { } }
%    \end{macrocode}
%
% Now that the compatibility layer is defined, we can finally load \pkg{xparse}.
% \pkg{xparse} expects to be loaded with §\ExplSyntaxOff§ (not much harm would
% be done otherwise, but just to be on the safe side).
%
% Within \pkg{xparse} a §\RequirePackage{expl3}§ is done. We can ignore that since
% we have already loaded \pkg{expl3}. Next, a §\@ifpackagelater§ test is done:
% we do that test too to ensure that \pkg{xparse} is compatible with the current
% running version of \pkg{expl3}. The following §\ProvidesExplPackage§ simply
% defines §\ver@xparse.sty§ for any other package that might uset it, and then
% does §\ExplSyntaxOn§. At the end of the package, \pkg{xparse} parses (heh)
% the package options. Since we don't have those in non-\LaTeX{} formats, they are
% ignored. Okay, so load \pkg{xparse}:
%    \begin{macrocode}
\int_set:Nn \l_@@_tmpa_int { \char_value_catcode:n { `\@ } }
\char_set_catcode_letter:N \@
\exp_after:wN
\ExplSyntaxOff
\file_input:n { xparse.sty }
\ExplSyntaxOn
\char_set_catcode:nn { `\@ } { \l_@@_tmpa_int }
\@@_compat_restore:
%</loader&!latex>
%<*core>
%    \end{macrocode}
% (actually we don't need to do §\ExplSyntaxOn§ there because we don't have
%  \LaTeX{}'s full package loading mechanism, so the \pkg{expl3} syntax remains
%  active after \pkg{xparse} is loaded, but it doesn't harm either).
%
% \subsection{Programming of the sequences}
%
%   The storage of the package is done using |seq| variables.  Here we
%   set up the macros that will manage the variables.
% \begin{macro}{\@@_append_contents:nn}
%   \cs{@@_append_contents:nn} creates a seq variable if one didn't
%   exist and appends the contents in the argument to the right of the
%   sequence.
%
%    \begin{macrocode}
\cs_new_protected:Npn \@@_append_contents:nn #1#2
  {
    \tl_if_blank:nT {#1}
      { \msg_error:nn { scontents } { empty-store-cmd } }
    \seq_if_exist:cF { g_@@_name_#1_seq }
      { \seq_new:c { g_@@_name_#1_seq } }
    \seq_gput_right:cn { g_@@_name_#1_seq } {#2}
  }
\cs_generate_variant:Nn \@@_append_contents:nn { Vx }
%    \end{macrocode}
% \end{macro}
%
% \begin{macro}{\@@_getfrom_seq:nn,\@@_getfrom_seq:nnn}
%   \cs{@@_getfrom_seq:nn} retrieves the saved item from the
%   sequence.
%
%    \begin{macrocode}
\cs_new:Npn \@@_getfrom_seq:nn #1#2
  {
    \seq_if_exist:cTF { g_@@_name_#2_seq }
      {
        \exp_args:Nf \@@_getfrom_seq:nnn
          { \seq_count:c { g_@@_name_#2_seq } }
          {#1} {#2}
      }
      { \msg_expandable_error:nnn { scontents } { undefined-storage } {#2} }
  }
\cs_new:Npn \@@_getfrom_seq:nnn #1#2#3
  {
    \bool_lazy_or:nnTF
      { \int_compare_p:nNn {#2} = { 0 } }
      { \int_compare_p:nNn { \int_abs:n {#2} } > {#1} }
      { \msg_expandable_error:nnnnn { scontents } { index-out-of-range } {#2} {#3} {#1} }
      { \seq_item:cn { g_@@_name_#3_seq } {#2} }
  }
%    \end{macrocode}
% \end{macro}
%
% \begin{macro}{\@@_lastfrom_seq:n}
%   \cs{@@_lastfrom_seq:n} retrieves the last saved item from the
%   sequence when \cs{l_@@_print_env_bool} or \cs{l_@@_print_cmd_bool}
%   is true.
%
%    \begin{macrocode}
\cs_new_protected:Npn \@@_lastfrom_seq:n #1
  {
    \tl_gset:Nx \g_@@_temp_tl { \seq_item:cn { g_@@_name_#1_seq } {-1} }
    \group_insert_after:N \@@_rescan_tokens:V
    \group_insert_after:N \g_@@_temp_tl
    \group_insert_after:N \tl_gclear:N
    \group_insert_after:N \g_@@_temp_tl
  }
%    \end{macrocode}
% \end{macro}
%
% \begin{macro}{\@@_store_to_seq:NN}
%   The \cs{@@_store_to_seq:NN} writes the recorded contents
%   in §#1§ to the log and stores it in §#2§.
%
%    \begin{macrocode}
\cs_new_protected:Npn \@@_store_to_seq:NN #1#2
  {
    \tl_log:N #1
    \@@_append_contents:Vx #2 { \exp_not:V #1 }
  }
%    \end{macrocode}
% \end{macro}
%
% \subsection{Construction of environment \env{scontents}}
%
% We define the environment \env{scontents}, next to the system
% \myoarg{key \textnormal{\textcolor{gray}{=}} val}. The environment
% is divided into two parts.
%
% \begin{macro}{
%     scontents,
%     \scontents,
%     \startscontents,
%     \endscontents,
%     \stopscontents,
%     \@@_scontents_env_begin:,
%     \@@_scontents_env_end:,
%   }
%   This is the main environment used in the document.
%
%    \begin{macrocode}
%</core>
%<*loader>
%<!context>\NewDocumentEnvironment { scontents } { }
%<context>\cs_new_protected:Npn \startscontents
  {
%<plain|context>    \group_begin:
    \@@_scontents_env_begin:
  }
%<context>\cs_new_protected:Npn \stopscontents
  {
    \@@_scontents_env_end:
%<plain|context>    \group_end:
  }
%</loader>
%<*core>
\cs_new_protected:Npn \@@_scontents_env_begin:
  {
    \char_set_catcode_active:N \^^M
    \@@_start_environment:w
  }
\cs_new_protected:Npn \@@_scontents_env_end:
  {
    \@@_stop_environment:
    \@@_atend_environment:
  }
%    \end{macrocode}
% \end{macro}
%
% \subsubsection{key val for environment}
%
% Define a \myoarg{key \textnormal{\textcolor{gray}{=}} val} for
% environment \env{scontents}
%
% \begin{macro}{\@@_grab_optional:n,\@@_grab_optional:w}
%   The macro \cs{@@_grab_optional:w} is called from the
%   \env{scontents} environment with the tokens following the
%   §\begin{scontents}§ when the next character is a §[§. This function
%   is defined using \mypkg{xparse} to exploit its delimited argument
%   processor.
%
%   The function is called from a context where §^^M§ is active, so
%   \cs{@@_normalise_line_ends:N} is used to replace active §^^M§
%   characters by spaces.
%    \begin{macrocode}
%</core>
%<*loader>
\NewDocumentCommand \@@_grab_optional:w { r[] }
  { \@@_grab_optional:n {#1} }
%</loader>
%<*core>
\cs_new_protected:Npn \@@_grab_optional:n #1
  {
    \tl_if_novalue:nF {#1}
      {
        \tl_set:Nn \l_@@_temp_tl {#1}
        \@@_normalise_line_ends:N \l_@@_temp_tl
        \keys_set:nV { scontents / scontents } \l_@@_temp_tl
      }
    \@@_start_after_option:w
  }
%    \end{macrocode}
% \end{macro}
%
% \subsubsection{The environment itself}
%
% \begin{macro}{\@@_start_environment:w,\@@_start_after_option:w,
%     \@@_check_line_process:xn,\@@_stop_environment:}
%   Here we make §^^I§, §^^L§ and §^^M§ active characters so that the end of
%   line can be \enquote{seen} to be used as a delimiter, and \hologo{TeX} doesn't try to
%   eliminate space-like characters.
%
%   First we check if the immediate next token after §\begin{scontents}§
%   is a §[§. If it is, then \cs{@@_grab_optional:w} is called
%   to do the heavy lifting. \cs{@@_grab_optional:w} processes the optional
%   argument and calls \cs{@@_start_after_option:w}.
%
%   \cs{@@_start_after_option:w} also checks for trailing tokens after
%   the optional argument and issues an error if any.
%
%   In all cases, \cs{@@_check_line_process:xn} ckecks that
%   everything past §\begin{scontents}§ is empty and then process
%   the environment. \cs{@@_check_line_process:xn} calls the
%   \cs{@@_file_tl_write_start:V} function, which will then read the contents
%   of the environment and optionally store them in a token list or write to
%   an external file.
%
%   When that's done, \cs{@@_file_write_stop:N} does the cleanup. This part of
%   the code is inspired and adapted from the code of the package \pkg{xsimverb}
%   by Clemens Niederberger.
%
%    \begin{macrocode}
\group_begin:
  \char_set_catcode_active:N \^^I
  \char_set_catcode_active:N \^^L
  \char_set_catcode_active:N \^^M
  \cs_new_protected:Npn \@@_normalise_line_ends:N #1
    { \tl_replace_all:Nnn #1 { ^^M } { ~ } }
  \cs_new_protected:Npn \@@_start_environment:w #1 ^^M
    {
      \tl_if_head_is_N_type:nTF {#1}
        {
          \str_if_eq:eeTF { \tl_head:n {#1} } { [ }
            { \@@_grab_optional:w #1 ^^M }
            { \@@_check_line_process:xn { } {#1} }
        }
        { \@@_check_line_process:xn { } {#1} }
    }
  \cs_new_protected:Npn \@@_start_after_option:w #1 ^^M
    { \@@_check_line_process:xn { [...] } {#1} }
  \cs_new_protected:Npn \@@_check_line_process:xn #1 #2
    {
      \tl_if_blank:nF {#2}
        {
          \msg_error:nnxn { scontents } { junk-after-begin }
            { after~\c_backslash_str begin{scontents} #1 } {#2}
        }
      \@@_make_control_chars_active:
      \@@_file_tl_write_start:V \l_@@_fname_out_tl
    }
  \cs_new_protected:Npn \@@_stop_environment:
    {
      \@@_file_write_stop:N \l_@@_macro_tmp_tl
      \bool_lazy_and:nnT
        { \l_@@_storing_bool }
        { \tl_if_empty_p:N \l_@@_macro_tmp_tl }
        { \msg_warning:nnn { scontents } { empty-environment } }
    }
%    \end{macrocode}
% \end{macro}
%
% \begin{macro}{
%     \@@_file_tl_write_start:n,
%     \@@_file_tl_write_start:V,
%     \@@_verb_processor_iterate:w,
%     \@@_file_write_stop:N,
%     \@@_remove_leading_nl:n,
%     \@@_remove_leading_nl:w,
%   }
%   This is the main macro to collect the contents of a verbatim environment.
%   The macro starts a group, opens the \mymeta{output file}, if necessary,
%   sets verbatim catcodes, and then issues §^^M§ (set equal to \cs{@@_ret:w})
%   to read the environment line by line until reaching its end. The output
%   token list will be appended with an active §^^J§ character and the line
%   just read, and this line is written to the output file, if any. At the end
%   of the environment the \mymeta{output file} is closed (if it was open),
%   and the output token list is smuggled out of the verbatim group. A leading
%   §^^J§ is removed from the token list using \cs{@@_remove_leading_nl:n}
%  (which expects an active §^^J§ token at the head of the token list; a low
%  level \hologo{TeX} error is raised otherwise).
%
%    \begin{macrocode}
  \cs_new_protected:Npn \@@_file_tl_write_start:n #1
    {
      \group_begin:
        \bool_if:nT  { \l_@@_writing_bool && \l_@@_overwrite_bool }
          {
             \file_if_exist:nTF {#1}
               { \msg_warning:nnx { scontents } { overwrite-file } {#1} }
               { \msg_warning:nnx { scontents } { writing-file } {#1} }
             \iow_open:Nn \l_@@_file_iow {#1}
          }
        \bool_if:nT  { \l_@@_writing_bool && !\l_@@_overwrite_bool }
          {
             \file_if_exist:nF {#1}
               { \msg_warning:nnx { scontents } { writing-file } {#1} }
              \iow_open:Nn \l_@@_file_iow {#1}
          }
        \tl_clear:N \l_@@_file_tl
        \seq_map_function:NN \l_char_special_seq \char_set_catcode_other:N
        \int_step_function:nnN { 128 } { 255 } \char_set_catcode_letter:n
        \cs_set_protected:Npx \@@_ret:w ##1 ^^M
          {
            \exp_not:N \@@_verb_processor_iterate:w
            ##1 \c_@@_end_env_tl
                \c_@@_end_env_tl
                \exp_not:N \q_@@_stop
          }
        \@@_make_control_chars_active:
        \@@_ret:w
    }
  \use:x
    {
      \cs_new:Npn \exp_not:N \@@_verb_processor_iterate:w
        ##1 \c_@@_end_env_tl
        ##2 \c_@@_end_env_tl
        ##3 \exp_not:N \q_@@_stop
    } {
        \tl_if_blank:nTF {#3}
          {
            \@@_analyse_nesting:n {#1}
            \@@_verb_processor_output:n {#1}
          }
          {
            \@@_if_nested:TF
              {
                \@@_nesting_decr:
                \@@_verb_processor_output:x
                  { \exp_not:n {#1} \c_@@_end_env_tl \exp_not:n {#2} }
              }
              {
                \tl_if_blank:nF {#1}
                  { \@@_verb_processor_output:n {#1} }
                \cs_set_protected:Npx \@@_ret:w
                  {
                    \@@_format_case:nnn
                      { \exp_not:N \end{scontents} } % LaTeX
                      { \endscontents } % Plain/Generic
                      { \stopscontents } % ConTeXt
                    \bool_lazy_or:nnF
                      { \tl_if_blank_p:n {#2} }
                      { \str_if_eq_p:ee {#2} { \c_percent_str } }
                      {
                        \str_if_eq:VnF \c_@@_hidden_space_str {#2}
                          { \msg_warning:nnn { scontents } { rescanning-text } {#2} }
                        \@@_rescan_tokens:n {#2}
                      }
                  }
                \char_set_active_eq:NN ^^M \@@_ret:w
              }
          }
        ^^M
      }
  \cs_new_protected:Npn \@@_file_write_stop:N #1
    {
      \bool_if:NT \l_@@_writing_bool
        { \iow_close:N \l_@@_file_iow }
      \use:x
        {
          \group_end:
          \bool_if:NT \l_@@_storing_bool
            {
              \tl_set:Nn \exp_not:N #1
                { \exp_args:NV \@@_remove_leading_nl:n \l_@@_file_tl }
            }
        }
    }
  \cs_new:Npn \@@_remove_leading_nl:n #1
    {
      \tl_if_head_is_N_type:nTF {#1}
        {
          \exp_args:Nf
            \@@_remove_leading_nl:nn
              { \tl_head:n {#1} } {#1}
        }
        { \exp_not:n {#1} }
    }
  \cs_new:Npn \@@_remove_leading_nl:nn #1 #2
    {
      \token_if_eq_meaning:NNTF ^^J #1
        { \exp_not:o { \@@_remove_leading_nl:w #2 } }
        { \exp_not:n {#2} }
    }
  \cs_new:Npn \@@_remove_leading_nl:w ^^J { }
%    \end{macrocode}
% \end{macro}
%
% \begin{macro}{
%     \@@_verb_processor_output:n,
%     \@@_verb_processor_output:x,
%     \@@_analyse_nesting:n,
%     \@@_analyse_nesting:w,
%     \@@_nesting_decr:,
%     \@@_use_none_delimit_by_q_stop:w,
%   }
% \begin{macro}[TF]{\@@_if_nested:}
%   \cs{@@_verb_processor_output:n} does the output of each line read,
%   to a token list and to a file, depending on the booleans
%   \cs{l_@@_writing_bool} and \cs{l_@@_storing_bool}.
%
%   \cs{@@_analyse_nesting:n} scans nested §\begin{scontents}§
%   and steps a \cs{l_@@_env_nesting_int} counter. The \cs{@@_if_nested:}
%   conditional tests if we're in a nested environment, and
%   \cs{@@_nesting_decr:} reduces the nesting level, if an §\end{scontents}§
%   is found.
%
%   Multiple §\end{scontents}§ in the same line are not supported...
%    \begin{macrocode}
  \cs_new_protected:Npn \@@_verb_processor_output:n #1
    {
      \bool_if:NT \l_@@_writing_bool
        { \iow_now:Nn \l_@@_file_iow {#1} }
      \bool_if:NT \l_@@_storing_bool
        { \tl_put_right:Nn \l_@@_file_tl { ^^J #1 } }
    }
  \cs_generate_variant:Nn \@@_verb_processor_output:n { x }
  \cs_new_protected:Npx \@@_analyse_nesting:n #1
    {
      \int_zero:N \l_@@_tmpa_int
      \exp_not:N \@@_analyse_nesting:w #1
        \c_backslash_str begin
          \c_left_brace_str \exp_not:N \q_@@_mark \c_right_brace_str
      \exp_not:N \q_@@_stop
      \int_compare:nNnT { \l_@@_tmpa_int } > { 1 }
        { \msg_warning:nn { scontents } { multiple-begin } }
    }
  \use:x
    {
      \cs_new_protected:Npn \exp_not:N \@@_analyse_nesting:w ##1
        \c_backslash_str begin \c_left_brace_str ##2 \c_right_brace_str
    }   {
          \if_meaning:w \q_@@_mark #2
            \exp_after:wN \use_i:nn
          \else:
            \exp_after:wN \use_ii:nn
          \fi:
            { \@@_use_none_delimit_by_q_stop:w }
            {
              \str_if_eq:eeT {#2} {scontents}
                {
                  \int_incr:N \l_@@_env_nesting_int
                  \int_incr:N \l_@@_tmpa_int
                  \@@_analyse_nesting:w
                }
              \@@_analyse_nesting:w
            }
        }
  \cs_new_protected:Npn \@@_nesting_decr:
    { \int_decr:N \l_@@_env_nesting_int }
  \prg_new_protected_conditional:Npnn \@@_if_nested: { TF }
    {
      \int_compare:nNnTF { \l_@@_env_nesting_int } > { \c_zero_int }
        { \prg_return_true: }
        { \prg_return_false: }
    }
  \cs_new:Npn \@@_use_none_delimit_by_q_stop:w #1 \q_@@_stop { }
\group_end:
\cs_generate_variant:Nn \@@_file_tl_write_start:n { V }
%    \end{macrocode}
% \end{macro}
% \end{macro}
%
% \subsubsection{Recording of the content in the sequence}
%
% \begin{macro}{\@@_atend_environment:}
%   Finishes the environment by optionally calling \cs{@@_store_to_seq:}
%   and then clearing the temporary token list.
%
%    \begin{macrocode}
\cs_new_protected:Npn \@@_atend_environment:
  {
    \bool_if:NT \l_@@_storing_bool
      {
        \bool_if:NF \l_@@_forced_eol_bool
          {
            \tl_put_right:Nx \l_@@_macro_tmp_tl
              { \c_@@_hidden_space_str }
          }
        \@@_store_to_seq:NN \l_@@_macro_tmp_tl
          \l_@@_name_seq_env_tl
        \bool_if:NT \l_@@_print_env_bool
          { \@@_lastfrom_seq:n \l_@@_name_seq_env_tl }
      }
  }
%</core>
%    \end{macrocode}
% \end{macro}
%
% \begin{macro}[int]{\verbatimsc,\endverbatimsc}
%   In Plain we emulate \LaTeX's \env{verbatim} environment.
%    \begin{macrocode}
%<*plain>
\bool_new:N \l_@@_temp_bool
\cs_new_protected:Npn \verbatimsc
  {
    \group_begin:
      \@@_verbatimsc_aux: \frenchspacing \@@_vobeyspaces:
      \@@_xverb:
  }
\cs_new_protected:Npn \endverbatimsc
  { \group_end: }
\cs_new_protected:Npn \@@_verbatimsc_aux:
  {
    \skip_vertical:N \parskip
    \int_set:Nn \parindent { 0pt }
    \skip_set:Nn \parfillskip { 0pt plus 1fil }
    \int_set:Nn \parskip { 0pt plus0pt minus0pt }
    \tex_par:D
    \bool_set_false:N \l_@@_temp_bool
    \cs_set:Npn \par
      {
        \bool_if:NTF \l_@@_temp_bool
          {
            \mode_leave_vertical:
            \null
            \tex_par:D
            \penalty \interlinepenalty
          }
          {
            \bool_set_true:N \l_@@_temp_bool
            \mode_if_horizontal:T
              { \tex_par:D \penalty \interlinepenalty }
          }
      }
    \cs_set_eq:NN \do \char_set_catcode_other:N
    \dospecials \obeylines
    \tl_use:N \l_@@_verb_font_tl
    \cs_set_eq:NN \do \@@_do_noligs:N
    \@@_nolig_list:
    \tex_everypar:D \exp_after:wN
      { \tex_the:D \tex_everypar:D \tex_unpenalty:D }
  }
\cs_new_protected:Npn \@@_nolig_list:
  { \do\`\do\<\do\>\do\,\do\'\do\- }
\cs_new_protected:Npn \@@_vobeyspaces:
  { \@@_set_active_eq:NN \  \@@_xobeysp: }
\cs_new_protected:Npn \@@_xobeysp:
  { \mode_leave_vertical: \nobreak \ }
%</plain>
%    \end{macrocode}
% \end{macro}
%
% \begin{macro}[int]{\dospecials}
%   \pkg{xparse} also requires \LaTeX's §\dospecials§. In case it doesn't
%   exist (at the time \pkg{scontents} is loaded) we define §\dospecials§
%   to use the \cs{l_char_special_seq}.
%    \begin{macrocode}
%<*!latex>
\cs_if_exist:NF \dospecials
  {
    \cs_new:Npn \dospecials
      { \seq_map_function:NN \l_char_special_seq \do }
  }
%</!latex>
%    \end{macrocode}
% \end{macro}
%
% \subsection{The command \cs{Scontents}}
%
% User command to \mymeta{stored content}, adapted from
% \url{https://tex.stackexchange.com/a/500281/7832}.
%
% \begin{macro}{
%     \Scontents,
%     \@@_norm_arg:n,
%     \@@_Scontents_auxi:N,
%     \@@_Scontents_internal:nn,
%     \@@_verb_arg:w,
%     \@@_verb_arg_internal:n,
%   }
%   The §\Scontents§ macro starts by parsing an optional argument and
%   then delegates to \cs{@@_verb_arg:w} or \cs{@@_norm_arg:n} depending whether
%   a star (§*§) argument is present.
%
%   \cs{@@_norm_arg:n} grabs a normal argument, adds it to the |seq| varaible,
%   and optionally prints it.
%
%   \cs{@@_verb_arg:w} grabs a verbatim argument using \pkg{xparse}'s |+v|
%   argument parser.
%
%    \begin{macrocode}
%<*loader>
\NewDocumentCommand { \Scontents }{ !s !O{} }
  { \@@_Scontents_internal:nn {#1} {#2} }
%</loader>
%<*core>
\cs_new_protected:Npn \@@_Scontents_internal:nn #1 #2
  {
    \group_begin:
      \tl_if_novalue:nF {#2}
        { \keys_set:nn { scontents / Scontents } {#2} }
      \char_set_catcode_active:n { 9 }
      \bool_if:NTF #1
        { \@@_verb_arg:w }
        { \@@_norm_arg:n }
  }
\cs_new_protected:Npn \@@_norm_arg:n #1
  {
      \tl_set:Nx \l_@@_temp_tl { \exp_not:n {#1} }
      \tl_put_right:Nx \l_@@_temp_tl { \c_@@_hidden_space_str }
      \@@_store_to_seq:NN \l_@@_temp_tl \l_@@_name_seq_cmd_tl
      \bool_if:NT \l_@@_print_cmd_bool
        { \@@_lastfrom_seq:n \l_@@_name_seq_cmd_tl }
    \group_end:
  }
%</core>
%<*loader>
\NewDocumentCommand { \@@_verb_arg:w } { +v }
  { \@@_verb_arg_internal:n {#1} }
%</loader>
%<*core>
\cs_new_protected:Npn \@@_verb_arg_internal:n #1
  {
      \tl_set:Nx \l_@@_temp_tl { \exp_not:n {#1} }
      \tl_replace_all:Nxx \l_@@_temp_tl { \iow_char:N \^^M } { \iow_char:N \^^J }
      \bool_if:NF \l_@@_forced_eol_bool
        { \tl_put_right:Nx \l_@@_temp_tl { \c_@@_hidden_space_str } }
      \@@_store_to_seq:NN \l_@@_temp_tl \l_@@_name_seq_cmd_tl
      \bool_if:NT \l_@@_print_cmd_bool
        { \@@_lastfrom_seq:n \l_@@_name_seq_cmd_tl }
    \group_end:
  }
%    \end{macrocode}
% \end{macro}
%
% \subsection{The command \cs{getstored}}
%
% \begin{macro}{\getstored}
%   User command §\getstored§ to extract \mymeta{stored content} in |seq|
%   (robust).
%
%    \begin{macrocode}
%</core>
%<*loader>
\NewDocumentCommand { \getstored } { O{1} m }
  { \@@_getstored_internal:nn {#1} {#2} }
%</loader>
%<*core>
\cs_new_protected:Npn \@@_getstored_internal:nn #1 #2
  {
    \group_begin:
      \int_set:Nn \tex_newlinechar:D { `\^^J }
      \@@_rescan_tokens:x
        {
    \endgroup % This assumes \catcode`\\=0... Things might go off otherwise.
    \@@_getfrom_seq:nn {#1} {#2}
        }
  }
%    \end{macrocode}
% \end{macro}
%
% \subsection{The command \cs{foreachsc}}
%
% \begin{macro}{\foreachsc}
%   User command §\foreachsc§ to loop over \mymeta{stored content} in |seq|.
%
%    \begin{macrocode}
%</core>
%<*loader>
\NewDocumentCommand { \foreachsc } { o m }
  { \@@_foreachsc_internal:nn {#1} {#2} }
%</loader>
%<*core>
\cs_new_protected:Npn \@@_foreachsc_internal:nn #1 #2
  {
    \group_begin:
      \tl_if_novalue:nF {#1} { \keys_set:nn { scontents / foreachsc } {#1} }
      \tl_set:Nn \l_@@_foreach_name_seq_tl {#2}
      \seq_clear:N \l_@@_foreach_print_seq
      \bool_if:NF \l_@@_foreach_stop_bool
        {
          \int_set:Nn \l_@@_foreach_stop_int
            { \seq_count:c { g_@@_name_#2_seq } }
        }
      \int_step_function:nnnN
        { \l_@@_foreach_start_int }
        { \l_@@_foreach_step_int }
        { \l_@@_foreach_stop_int }
        \@@_foreach_add_body:n
      \tl_gset:Nx \g_@@_temp_tl
        {
          \seq_use:Nn \l_@@_foreach_print_seq
            { \tl_use:N \l_@@_foreach_sep_tl }
        }
    \group_end:
    \exp_after:wN \tl_gclear:N
    \exp_after:wN \g_@@_temp_tl
      \g_@@_temp_tl
  }
\cs_new_protected:Npn \@@_foreach_add_body:n #1
  {
    \seq_put_right:Nx \l_@@_foreach_print_seq
      {
        \bool_if:NT \l_@@_foreach_before_bool
          { \exp_not:V \l_@@_foreach_before_tl }
        \bool_if:NTF \l_@@_foreach_wrapper_bool
          { \@@_foreach_wrapper:n }
          { \use:n }
            { \getstored [#1] { \tl_use:N \l_@@_foreach_name_seq_tl } }
        \bool_if:NT \l_@@_foreach_after_bool
          { \exp_not:V \l_@@_foreach_after_tl }
      }
  }
%    \end{macrocode}
% \end{macro}
%
% \subsection{The command \cs{typestored}}
%
% \begin{macro}{\typestored,\@@_verb_print:N,\@@_xverb:w,verbatimsc}
%   The §\typestored§ commands fetches a buffer from memory, prints it
%   to the log file, and then calls \cs{@@_verb_print:N}.
%
%    \begin{macrocode}
%</core>
%<*loader>
\NewDocumentCommand { \typestored } { o m }
  { \@@_typestored_internal:nn {#1} {#2} }
%</loader>
%<*core>
\cs_new_protected:Npn \@@_typestored_internal:nn #1 #2
  {
    \group_begin:
      \int_set:Nn \l_@@_seq_item_int { 1 }
      \tl_if_novalue:nF {#1} { \keys_set:nn { scontents / typemeaning } {#1} }
      \tl_set:Nx \l_@@_temp_tl
        { \exp_args:NV \@@_getfrom_seq:nn \l_@@_seq_item_int {#2} }
      \tl_remove_once:NV \l_@@_temp_tl \c_@@_hidden_space_str
      \tl_log:N \l_@@_temp_tl
      \tl_if_empty:NF \l_@@_temp_tl
        { \@@_verb_print:N \l_@@_temp_tl }
    \group_end:
  }
%    \end{macrocode}
%
%   The \cs{@@_verb_print:N} macro is defined with active carriage return
%   (\textsc{ascii} 13) characters to mimick an actual verbatim environment
%   \enquote{on the loose}. The contents of the environment are placed in a
%   §verbatimsc§ environment and rescanned using \cs{@@_rescan_tokens:x}.
%
%    \begin{macrocode}
\group_begin:
  \char_set_catcode_active:N \^^M
  \cs_new_protected:Npn \@@_verb_print:N #1
    {
      \tl_if_blank:VT #1
        { \msg_error:nnn { scontents } { empty-variable } {#1} }
      \cs_set_eq:NN \@@_verb_print_EOL: ^^M
      \cs_set_eq:NN ^^M \scan_stop:
      \cs_set_eq:cN { do@noligs } \@@_do_noligs:N
      \int_set:Nn \tex_newlinechar:D { `\^^J }
      \@@_rescan_tokens:x
        {
          \@@_format_case:nnn
            { \exp_not:N \begin{verbatimsc} } % LaTeX
            { \verbatimsc } % Plain/Generic
            { \startverbatimsc } % ConTeXt
            ^^M
          \exp_not:V #1 ^^M
          \g_@@_end_verbatimsc_tl
        }
      \cs_set_eq:NN ^^M \@@_verb_print_EOL:
    }
\group_end:
%    \end{macrocode}
%
%   Finally, the §verbatimsc§ environment is defined.
%
%    \begin{macrocode}
\cs_new_protected:Npn \@@_xverb:
  {
    \char_set_catcode_active:n { 9 }
    \char_set_active_eq:nN { 9 } \@@_tabs_to_spaces:
    \@@_xverb:w
  }
%</core>
%<*loader>
%<*!context>
\use:x
  {
    \cs_new_protected:Npn \exp_not:N \@@_xverb:w
      ##1 \g_@@_end_verbatimsc_tl
%<latex>      { ##1 \exp_not:N \end{verbatimsc} }
%<plain>      { ##1 \exp_not:N \endverbatimsc }
%<context>      { ##1 \exp_not:N \stopverbatimsc }
  }
%</!context>
%<*latex>
\NewDocumentEnvironment { verbatimsc } { }
  {
    \cs_set_eq:cN { @xverbatim } \@@_xverb:
    \verbatim
  }
  { }
%</latex>
%<context>\definetyping[verbatimsc]
%</loader>
%<*core>
%    \end{macrocode}
% \end{macro}
%
% \subsection{Some auxiliaries}
%
% \begin{macro}{\@@_tabs_to_spaces:}
%   In a verbatim context the |TAB| character is made active and set
%   equal to \cs{@@_tabs_to_spaces:}, to produce as many spaces as
%   the §width-tab§ key was set to.
%
%    \begin{macrocode}
\cs_new:Npn \@@_tabs_to_spaces:
  { \prg_replicate:nn { \l_@@_tab_width_int } { ~ } }
%    \end{macrocode}
% \end{macro}
%
% \begin{macro}{\@@_do_noligs:N}
%   \cs{@@_do_noligs:N} is an alternative definition for \LaTeXe's
%   §\do@noligs§ which makes sure to not consume following space
%   tokens. The \LaTeXe{} version ends with §\char`#1§, which
%   leaves \TeX{} still looking for an \mymeta{optional~space}.
%   This version uses \cs{char_generate:nn} to ensure that doesn't
%   happen.
%
%    \begin{macrocode}
\cs_new:Npn \@@_do_noligs:N #1
  {
    \char_set_catcode_active:N #1
    \char_set_active_eq:Nc #1 { @@_active_char_ \token_to_str:N #1 : }
    \cs_set:cpx { @@_active_char_ \token_to_str:N #1 : }
      {
        \mode_leave_vertical:
        \tex_kern:D \c_zero_dim
        \char_generate:nn { `#1 } { 12 }
      }
  }
%    \end{macrocode}
% \end{macro}
%
% \begin{macro}{
%     \@@_set_active_eq:NN,
%     \@@_make_control_chars_active:,
%   }
%   Shortcut definitions for common catcode changes.
%   The §^^L§ needs a special treatment in non-\LaTeX{} mode
%   because in Plain\,\TeX{} it is an §\outer§ token.
%    \begin{macrocode}
\cs_new_protected:Npn \@@_set_active_eq:NN #1
  {
    \char_set_catcode_active:N #1
    \char_set_active_eq:NN #1
  }
%</core>
%<*loader>
\group_begin:
%<plain>  \char_set_catcode_active:n { `\* }
  \cs_new_protected:Npn \@@_plain_disable_outer_par:
%<*plain>
    {
      \group_begin:
        \char_set_lccode:nn { `\* } { `\^^L }
        \tex_lowercase:D { \group_end:
        \tex_let:D * \scan_stop:
      }
    }
%</plain>
%<latex|context>    { }
\group_end:
%</loader>
%<*core>
\group_begin:
  \char_set_catcode_active:N \*
  \cs_new_protected:Npn \@@_make_control_chars_active:
    {
      \@@_plain_disable_outer_par:
      \@@_set_active_eq:NN \^^I \@@_tab:
      \@@_set_active_eq:NN \^^L \@@_par:
      \@@_set_active_eq:NN \^^M \@@_ret:w
    }
\group_end:
%    \end{macrocode}
% \end{macro}
%
% \subsection{The command \cs{setupsc}}
%
% User command §\setupsc§ to setup module.
%
% \begin{macro}{\setupsc}
%   A user-level wrapper for \cs{keys_set:nn}§{ scontents }§.
%
%    \begin{macrocode}
%</core>
%<*loader>
\NewDocumentCommand { \setupsc } { +m }
  { \keys_set:nn { scontents } {#1} }
%</loader>
%<*core>
%    \end{macrocode}
% \end{macro}
%
% \subsection{The command \cs{meaningsc}}
%
% \begin{macro}{\meaningsc}
%   User command §\meaningsc§ to see content stored in |seq|.
%
%    \begin{macrocode}
%</core>
%<*loader>
\NewDocumentCommand { \meaningsc } { o m }
  { \@@_meaningsc_internal:nn {#1} {#2} }
%</loader>
%<*core>
\cs_new_protected:Npn \@@_meaningsc_internal:nn #1 #2
  {
    \group_begin:
      \int_set:Nn \l_@@_seq_item_int { 1 }
      \tl_if_novalue:nF {#1} { \keys_set:nn { scontents / typemeaning } {#1} }
      \@@_meaningsc:n {#2}
    \group_end:
  }
\group_begin:
  \char_set_catcode_active:N \^^I
  \cs_new_protected:Npn \@@_meaningsc:n #1
    {
      \tl_set:Nx \l_@@_temp_tl
        { \exp_args:NV \@@_getfrom_seq:nn \l_@@_seq_item_int {#1} }
      \tl_replace_all:Nxn \l_@@_temp_tl { \iow_char:N \^^J } { ~ }
      \tl_remove_once:NV  \l_@@_temp_tl \c_@@_hidden_space_str
      \tl_log:N \l_@@_temp_tl
      \tl_use:N \l_@@_verb_font_tl
      \tl_replace_all:Nnx \l_@@_temp_tl { ^^I } { \@@_tabs_to_spaces: }
      \cs_replacement_spec:N \l_@@_temp_tl
    }
\group_end:
%    \end{macrocode}
% \end{macro}
%
% \subsection{The command \cs{countsc}}
%
%
% \begin{macro}{\countsc}
%   User command §\countsc§ to count number of contents stored in |seq|.
%
%    \begin{macrocode}
%</core>
%<*loader>
\NewExpandableDocumentCommand { \countsc } { m }
  { \seq_count:c { g_@@_name_#1_seq } }
%</loader>
%<*core>
%    \end{macrocode}
% \end{macro}
%
% \subsection{The command \cs{cleanseqsc}}
%
%
% \begin{macro}{\cleanseqsc}
%   A user command §\cleanseqsc§ to clear (remove) a defined |seq|.
%
%    \begin{macrocode}
%</core>
%<*loader>
\NewDocumentCommand { \cleanseqsc } { m }
  { \seq_clear_new:c { g_@@_name_#1_seq } }
%</loader>
%<*core>
%    \end{macrocode}
% \end{macro}
%
% \subsection{Warning and error messages}
%
% Warning and error messages used throughout the package.
%
%    \begin{macrocode}
\msg_new:nnn { scontents } { junk-after-begin }
  {
    Junk~characters~#1~\msg_line_context: :
    \\ \\
    #2
  }
\msg_new:nnn { scontents } { empty-stored-content }
  { Empty~value~for~key~‘getstored’~\msg_line_context:. }
\msg_new:nnn { scontents } { empty-variable }
  { Variable~‘#1’~empty~\msg_line_context:. }
\msg_new:nnn { scontents } { overwrite-file }
  { Overwriting~file~‘#1’. }
\msg_new:nnn { scontents } { writing-file }
  { Writing~file~‘#1’. }
\msg_new:nnn { scontents } { rescanning-text }
  { Rescanning~text~‘#1’~after~\c_backslash_str end{scontents}~\msg_line_context:.}
\msg_new:nnn { scontents } { multiple-begin }
  { Multiple~\c_backslash_str begin{scontents}~\msg_line_context:.}
\msg_new:nnn { scontents } { undefined-storage }
  { Storage~named~‘#1’~is~not~defined. }
\msg_new:nnn { scontents } { index-out-of-range }
  {
    \int_compare:nNnTF {#1} = { 0 }
      { Index~of~sequence~cannot~be~zero. }
      {
        Index~‘#1’~out~of~range~for~‘#2’.~
        \int_compare:nNnTF {#1} > { 0 }
          { Max = } { Min = -} #3.
      }
  }
\msg_new:nnnn { scontents } { env-key-unknown }
  { The~key~‘#1’~is~unknown~by~environment~‘scontents’~and~is~being~ignored.}
  {
    The~environment~‘scontents’~does~not~have~a~key~called~‘#1’.\\
    Check~that~you~have~spelled~the~key~name~correctly.
  }
\msg_new:nnnn { scontents } { env-key-value-unknown }
  { The~key~‘#1=#2’~is~unknown~by~environment~‘scontents’~and~is~being~ignored. }
  {
    The~environment~‘scontents’~does~not~have~a~key~called~‘#1’.\\
    Check~that~you~have~spelled~the~key~name~correctly.
  }
\msg_new:nnnn { scontents } { cmd-key-unknown }
  { The~key~‘#1’~is~unknown~by~‘\c_backslash_str Scontents’~and~is~being~ignored.}
  {
    The~command~‘\c_backslash_str Scontents’~does~not~have~a~key~called~‘#1’.\\
    Check~that~you~have~spelled~the~key~name~correctly.
  }
\msg_new:nnnn { scontents } { cmd-key-value-unknown }
  { The~key~‘#1=#2’~is~unknown~by~‘\c_backslash_str Scontents’~and~is~being~ignored. }
  {
    The~command~‘\c_backslash_str Scontents’~does~not~have~a~key~called~‘#1’.\\
    Check~that~you~have~spelled~the~key~name~correctly.
  }
\msg_new:nnnn { scontents } { for-key-unknown }
  { The~key~‘#1’~is~unknown~by~‘\c_backslash_str foreachsc’~and~is~being~ignored.}
  {
    The~command~‘\c_backslash_str foreachsc’~does~not~have~a~key~called~‘#1’.\\
    Check~that~you~have~spelled~the~key~name~correctly.
  }
\msg_new:nnnn { scontents } { for-key-value-unknown }
  { The~key~‘#1=#2’~is~unknown~by~‘\c_backslash_str foreachsc’~and~is~being~ignored. }
  {
    The~command~‘\c_backslash_str foreachsc’~does~not~have~a~key~called~‘#1’.\\
    Check~that~you~have~spelled~the~key~name~correctly.
  }
\msg_new:nnnn { scontents } { type-key-unknown }
  { The~key~‘#1’~is~unknown~and~is~being~ignored. }
  {
    This~command~does~not~have~a~key~called~‘#1’.\\
    This~command~only~accepts~the~key~‘width-tab’.
  }
\msg_new:nnnn { scontents } { type-key-value-unknown }
  { The~key~‘#1’~to~which~you~passed~‘#2’~is~unknown~and~is~being~ignored. }
  {
    This~command~does~not~have~a~key~called~‘#1’.\\
    This~command~only~accepts~the~key~‘width-tab’.
  }
\msg_new:nnn { scontents } { empty-environment }
  { scontents~environment~empty~\msg_line_context:. }
\msg_new:nnnn { scontents } { verbatim-newline }
  { Verbatim~argument~of~#1~ended~by~end~of~line. }
  {
    The~verbatim~argument~of~the~#1~cannot~contain~more~than~one~line,~
    but~the~end~
    of~the~current~line~has~been~reached.~You~may~have~forgotten~the~
    closing~delimiter.
    \\ \\
    LaTeX~will~ignore~‘#2’.
  }
\msg_new:nnnn { scontents } { verbatim-tokenized }
  { The~verbatim~#1~cannot~be~used~inside~an~argument. }
  {
    The~#1~takes~a~verbatim~argument.~
    It~may~not~appear~within~the~argument~of~another~function.~
    It~received~an~illegal~token \tl_if_empty:nF {#3} { ~‘#3’ } .
    \\ \\
    LaTeX~will~ignore~‘#2’.
  }
%    \end{macrocode}
%
% \subsection{Finish package}
%
% Finish package implementation.
%
%    \begin{macrocode}
%</core>
%<plain|context>\ExplSyntaxOff
%    \end{macrocode}
%
% \Finale
