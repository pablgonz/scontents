% arara: pdflatex: { interaction: batchmode }
% arara: clean: { extensions: [ aux, log] }
\pdfinfoomitdate1
\pdfsuppressptexinfo-1
\pdftrailerid{}
\documentclass{article}
\usepackage{scontents}
% only for test
\usepackage[top=0.5in,bottom=0.5in, left=1in, right=1in]{geometry}
\usepackage{fvextra}
\usepackage{listings}
\setlength{\parindent}{0pt}
\begin{document}

\edef\tmpa{\the\interactionmode}
\scrollmode
\def\testthis#1{\meaningsc[#1]{nowhere}\par\hrulefill}

Nothing:
\begin{scontents}[store-env=nowhere]
hey
blub\end{scontents}

Verbatim (hey blub):
\testthis{1}

Math-mode ($blob$):
\begin{scontents}[store-env=nowhere]
hey
\end{scontents}$blob$

Verbatim (hey):
\testthis{2}

Math-mode ($blob$):
\begin{scontents}[store-env=nowhere]
hey
blub\end{scontents}$blob$

Verbatim (hey blub):
\testthis{3}

Nothing:
\begin{scontents}[store-env=nowhere]
\begin{scontents}[store-env=nowhere]
  hello
\end{scontents}
\end{scontents}

Verbatim (\texttt{\string\begin}\dots):
\testthis{4}

Multi-line verbatim (\texttt{\string\begin}\dots):
\typestored[4]{nowhere}

Nothing (+error):
\typestored[5]{nowhere} % Error

Nothing:
\getstored[4]{nowhere} % Process the inner environment

Multi-line verbatim (hello):
\typestored[5]{nowhere} % Not an error anymore

\hrulefill
\setupsc{store-env=error-storage}

\begin{scontents}[store-env=nowhere]
\1
\end{scontents}
Should be 1: \meaningsc[-1]{nowhere}

Should raise an error about ` bzzz':\par
\begin{scontents}[
  store-env
  =
  nowhere
] bzzz
\2
\end{scontents}
Should be 2: \meaningsc[-1]{nowhere}

Should raise an error about ` [store-env=nowhere]':
\begin{scontents} [store-env=nowhere]
\3
\end{scontents}
Should still be 2: \meaningsc[-1]{nowhere}

\begin{scontents}
[store-env=nowhere]
\4
\end{scontents}
Should be 2: \meaningsc[-1]{nowhere}
But this should be ``[store-env=nowhere] 4'': \meaningsc[-1]{error-storage}

\begin{scontents}%
[store-env=nowhere]
\5
\end{scontents}
Should be 5: \meaningsc[-1]{nowhere}

\begin{scontents}%[store-env=nowhere]
\6
\end{scontents}
Should still be 5: \meaningsc[-1]{nowhere}
And this should be empty: [\meaningsc[-1]{error-storage}]

\interactionmode=\tmpa

\setupsc{store-env=contents}

\hrulefill

% custom :)
\makeatletter
\let\verbatimsc\@undefined
\let\endverbatimsc\@undefined
\makeatother
\ExplSyntaxOn
\use:n
  {
\ExplSyntaxOff
\DefineVerbatimEnvironment{verbatimsc}{Verbatim}
  {
    numbers=left,
    frame=single,
    codes*=
      {
        \char_set_catcode_active:n { 9 }
        \char_set_active_eq:nN { 9 } \__scontents_tabs_to_spaces:
      }
  }
  }

\Scontents[store-cmd=nospace]{Write in one line \LaTeX: $E=mc^2$, no space after dot.}

\Scontents*[store-cmd=nospace]|Other in one line, no space after dot.|

\Scontents*[store-cmd=nospace]|Other	in	one	line,	no	space	after	dot.|

\Scontents*[store-cmd=nospace]|TODO: Tab after comma:,	.|

XX\getstored[1]{nospace}YY\par
XX\getstored[2]{nospace}YY\par
XX\getstored[3]{nospace}YY\par

\typestored[1]{nospace}\par
\typestored[2]{nospace}\par
\typestored[3]{nospace}\par
\typestored[3, width-tab=1]{nospace}\par
\typestored[3, width-tab=2]{nospace}\par
\typestored[3, width-tab=3]{nospace}\par
\typestored[3, width-tab=4]{nospace}\par
\typestored[4, width-tab=1]{nospace}\par

\meaningsc[1]{nospace}\par
\meaningsc[2]{nospace}\par
\meaningsc[3]{nospace}\par
\meaningsc[3, width-tab=1]{nospace}\par
\meaningsc[3, width-tab=2]{nospace}\par
\meaningsc[3, width-tab=3]{nospace}\par
\meaningsc[3, width-tab=4]{nospace}\par
\meaningsc[4, width-tab=1]{nospace}\par

% These should raise errors
\edef\tmpa{\the\interactionmode}
\scrollmode
\getstored[0]{nospace}
\getstored[5]{nospace}
\getstored[-5]{nospace}
\getstored[1]{blub}
\interactionmode=\tmpa

% These outputs should look the same
FIRST\par
This ^^J is normal ^^J text.
\verb|This is from the verb command.|
\verb*|This is from the verb* command.|
This is normal text.
\begin{verbatim}
This is from the verbatim environment:
&%{}~
Soy un ^^J Soy un ^^M text.
\end{verbatim}

SECOND\par
% These outputs should look the same with print-env=true
\begin{scontents}[store-env=newattempt,print-env=true]
This ^^J is normal ^^J text.
\verb|This is from the verb command.|
\verb*|This is from the verb* command.|
This is normal text.
\begin{verbatim}
This is from the verbatim environment:
&%{}~
Soy un ^^J Soy un ^^M text.
\end{verbatim}
\end{scontents}This text is removed from the output because it is on the same line.
THIRD\par
% These outputs should look the same
\getstored[1]{newattempt}\par

MEANINGSC\par
\meaningsc[1]{newattempt}\par
% Tab
\begin{scontents}[write-out=withtab.tsc]
No tab
	One tab
		Two tab and
		This ^^J is normal ^^J text.
\begin{verbatim}
      (E) verbatim
          environment
\end{verbatim}
\end{scontents}This text is removed from the output because it is on the same line.

\typestored[1]{newattempt}

\section{Test \texttt{\textbackslash Scontents[key=val]}}

Test \verb+\Scontents[print-cmd=false]+\par

\Scontents[print-cmd=false]{
Using \texttt{\textbackslash Scontents} command, \par
no environment verbatim support,\par
but \Verb{fvextra inline yes} and \lstinline[basicstyle=\ttfamily]|lstinline yes|,
save in seq \texttt{contents} with index $1$.}\par

Test \verb+\Scontents[print-cmd=true]+\par

\Scontents[print-cmd=true]{
Using \texttt{\textbackslash Scontents} command, no environment verbatim support,
but \Verb{fvextra inline yes} and \lstinline[basicstyle=\ttfamily]|lstinline yes|,
save in seq \texttt{contents} with index $2$.}

Test \verb+\Scontents[store-cmd=testcmd]+\par

\Scontents[store-cmd=testcmd]{
Using \texttt{\textbackslash Scontents} command, no environment verbatim support,
but \Verb{fvextra inline yes} and \lstinline[basicstyle=\ttfamily]|lstinline yes|,
save in seq \texttt{testcmd} with index 1.}

\subsection{Test \texttt{\textbackslash getstored[index]\{seq name\}}}

\noindent\hrulefill

Test \verb+\getstored[1]{contents}+:\par

\getstored[1]{contents}

\noindent\hrulefill

Test \verb+\getstored[2]{contents}+:\par

\getstored[2]{contents}

\noindent\hrulefill

Test \verb+\getstored[1]{testcmd}+:\par

\getstored[1]{testcmd}

\noindent\hrulefill

\subsection{Test \texttt{\textbackslash meaningsc[index]\{seq name\}}}

\noindent\hrulefill

Test \verb+\meaningsc[1]{contents}+:\par

\meaningsc[1]{contents}

\noindent\hrulefill

Test \verb+\meaningsc[2]{contents}+:\par

\meaningsc[2]{contents}

\noindent\hrulefill

Test \verb+\meaningsc[1]{testcmd}+:\par

\meaningsc[1]{testcmd}

\noindent\hrulefill

\section{Test \texttt{\textbackslash Scontents*[key=val]}}

Test \verb+\setupsc{store-cmd=testcmdvrb}+\par
\setupsc{store-cmd=testcmdvrb}

Test \verb+\Scontents*[print-cmd=false]+\par

\Scontents*[print-cmd=false]{Using \texttt{\textbackslash Scontents*} command,

with verbatim support.

\begin{verbatim*}
      verbatim
              environment
\end{verbatim*}

\Verb{fvextra inline yes}, save in seq \texttt{testcmdvrb} with index 1.}

Test \verb+\Scontents*[print-cmd=true]+\par
\Scontents*[print-cmd=true]|Using \verb/\Scontents*[print-cmd=true]+...+/ command, with verbatim support,

\begin{verbatim}
      more verbatim environment
\end{verbatim}

And \Verb{fvextra inline yes}, save in seq \texttt{testcmdvrb} with index $2$.|

Test \verb+\Scontents[store-cmd=othervrb]+\par
\Scontents*[store-cmd=othervrb]{
Using \texttt{\textbackslash Scontents*} command, with verbatim support,

\begin{verbatim}
      other verbatim  environment
\end{verbatim}

And \Verb{fvextra inline yes}, save in seq \texttt{othervrb} with index $1$.}

\subsection{Test \texttt{\textbackslash getstored[index]\{seq name\}}}

\noindent\hrulefill

Test \verb+\getstored[1]{testcmdvrb}+:\par

\getstored[1]{testcmdvrb}

\noindent\hrulefill

Test \verb+\getstored[2]{testcmdvrb}+:\par

\getstored[2]{testcmdvrb}

\noindent\hrulefill

Test \verb+\getstored[1]{othervrb}+:\par

\getstored[1]{othervrb}

\noindent\hrulefill

\subsection{Test \texttt{\textbackslash meaningsc[index]\{seq name\}}}

\noindent\hrulefill

Test \verb+\meaningsc[1]{testcmdvrb}+:\par

\meaningsc[1]{testcmdvrb}

\noindent\hrulefill

Test \verb+\meaningsc[2]{testcmdvrb}+:\par

\meaningsc[2]{testcmdvrb}

\noindent\hrulefill

Test \verb+\meaningsc[1]{othervrb}+:\par

\meaningsc[1]{othervrb}

\noindent\hrulefill

\subsection{Test \texttt{\textbackslash typestored[index]\{seq name\}}}

\noindent\hrulefill

Test \verb+\typestored[1]{testcmdvrb}+:\par

\typestored[1]{testcmdvrb}

\noindent\hrulefill

Test \verb+\typestored[2]{testcmdvrb}+:\par

\typestored[2]{testcmdvrb}

\noindent\hrulefill

Test \verb+\typestored[1]{othervrb}+:\par

\typestored[1]{othervrb}

\noindent\hrulefill

\section{Test \texttt{\textbackslash Scontents*} and \texttt{\textbackslash Scontents} back to default}

\Scontents[store-cmd=contents]{Using \texttt{\textbackslash Scontents} command with braces,
save in seq \texttt{contents} with index $3$.}
\getstored[3]{contents}

\Scontents*|Using \texttt{\textbackslash Scontents*} command (verbatimized), save in seq \texttt{testcmdvrb} with index $3$.|
\getstored[3]{testcmdvrb}

\section{Test on mobile arguments}

\noindent\hrulefill

Test \verb+\subsection{\getstored[3]{testcmdvrb}}+:\par

\subsection{\getstored[3]{testcmdvrb}}

\noindent\hrulefill

Test \verb+\subsection{\getstored[3]{contents}}+:\par

\subsection{\getstored[3]{contents}}

\noindent\hrulefill

\section{Test \texttt{\{scontents\}} environment}
\setupsc{store-env=testenv, print-env=true}
\subsection{no \texttt{[key=val]}}
Test \verb+scontents+ no \verb+[key=val]+\par

\begin{scontents}[force-eol]
Using \verb+scontents+ env, \verb+[force-eol]+, save in seq \verb+testenv+ index 1.\par
We have coded this in \LaTeX: $E=mc^2$.

Prove new \Verb*{ verbatim from fvextra with braces } and environment \verb+Verbatim*+.
\begin{Verbatim*}
      (A) verbatim
          environment
\end{Verbatim*}
\end{scontents}

\enlargethispage{1.5em}

\subsubsection{\Verb{\getstored[1]{testenv}}}
%
XX\getstored[1]{testenv}YY

\subsubsection{\Verb{\meaningsc[1]{testenv}}}
\meaningsc[1]{testenv}

\subsubsection{\Verb{\typestored[1]{testenv}}}
\typestored[1]{testenv}

\subsection{with \texttt{[key=val]}}

\subsubsection{\Verb{[write-env=\jobname.tsc]}}

\begin{scontents}[write-env=\jobname.tsc]
Using \verb+scontents+ env with \verb+[write-env=\jobname.tsc]+, save in
seq \verb+testenv+ with index 2 and write env body in file \verb+\jobname.tsc+.

We have coded this in \LaTeX: $E=mc^2$.\par
\begin{verbatim*}
      (B) verbatim
          environment
\end{verbatim*}
\end{scontents}

\subsubsection{\Verb{\getstored[2]{testenv}}}
\getstored[2]{testenv}

\subsubsection{\Verb{\meaningsc[2]{testenv}}}
\meaningsc[2]{testenv}

\subsubsection{\Verb{\typestored[2]{testenv}}}
\typestored[2]{testenv}

\subsubsection{\Verb{\VerbatimInput{\jobname.tsc}}}

\VerbatimInput[frame=single,numbers=left,numbersep=3pt]{\jobname.tsc}

\subsubsection{\Verb{[print-env=true]}}

\begin{scontents}[print-env=true]
Using \verb+scontents+ env with \verb+[print-env=true]+, save in
seq \verb+testenv+ with index 3.

We have coded this in \LaTeX: $E=mc^2$.\par
\begin{verbatim*}
      (C) verbatim
          environment
\end{verbatim*}
\end{scontents}

\Verb{\getstored[3]{testenv}}\par
\getstored[3]{testenv}\par

\Verb{\meaningsc[3]{testenv}}\par
\meaningsc[3]{testenv}\par

\Verb{\typestored[3]{testenv}}\par
\typestored[3]{testenv}

\subsubsection{\Verb{[print-env=true,write-env=\jobname-1.tsc]}}

\begin{scontents}[print-env=true,write-env=\jobname-1.tsc]
Using \verb+scontents+ env with \verb+[print-env=true,write-env=\jobname-1.tsc]+, save in
seq \verb+testenv+ with index 4.

We have coded this in \LaTeX: $E=mc^2$.\par
\begin{verbatim*}
      (D) verbatim
          environment
\end{verbatim*}
\end{scontents}

\Verb{\getstored[4]{testenv}}\par
\getstored[4]{testenv}\par

\Verb{\meaningsc[4]{testenv}}\par
\meaningsc[4]{testenv}\par

\Verb{\typestored[4]{testenv}}\par
\typestored[4]{testenv}

\Verb{\VerbatimInput{\jobname-1.tsc}}\par

\VerbatimInput[frame=single,numbers=left,numbersep=3pt]{\jobname-1.tsc}

\subsection{\Verb{[write-out=\jobname-2.tsc]}}

\begin{scontents}[write-out=\jobname-2.tsc]
Using \verb+scontents+ env with \verb+[write-out=\jobname-2.tsc]+,
not added to seq, only write a external file
We have coded this in \LaTeX: $E=mc^2$.\par
\begin{verbatim}
      (E) verbatim
          environment
\end{verbatim}
\end{scontents}

\Verb{\VerbatimInput{\jobname-2.tsc}}\par

\VerbatimInput[frame=single,numbers=left,numbersep=3pt]{\jobname-2.tsc}

\subsubsection{\Verb{[print-env=true,store-env=other]}}

\begin{scontents}[print-env=true,store-env=other]
Using \verb+scontents+ env with \verb+[print-env=true,store-env=other]+, save in
seq \verb+other+ with index 1.

We have coded this in \LaTeX: $E=mc^2$.\par
\begin{verbatim*}
      (F) verbatim
          environment
\end{verbatim*}
\end{scontents}

\Verb{\getstored[1]{other}}\par
\getstored[1]{other}\par

\Verb{\meaningsc[1]{other}}\par
\meaningsc[1]{other}\par

\Verb{\typestored[1]{other}}\par
\typestored[1]{other}

\section{A simple application}
\setupsc{ store-env=answers, store-cmd= answers }

\newcounter{exeNr}
\newenvironment{exercise}
  {\refstepcounter{exeNr}\par\noindent This is exercise~\theexeNr}
  {\par}
\subsection{Exercises}
\begin{exercise}
\end{exercise}

\begin{scontents}[print-env=true, force-eol=true]
This is the answer to exercise 1, the shebang line for a Perl script

\begin{Verbatim}
#!/usr/bin/env perl
\end{Verbatim}
\end{scontents}YYYYYYYYYYxxx

\typestored[1]{answers}

YYY\getstored[1]{answers}XXX


\begin{exercise}
\end{exercise}
\Scontents*{This is the answer to exercise 2}

\begin{exercise}
\end{exercise}
\Scontents{This is the answer to exercise 3}

\subsection{Answers}
\newcounter{ansNr}
\newenvironment{answer}
  {\refstepcounter{ansNr}\par\noindent Answer~\theansNr:}
  {\par}

\foreachsc[before={\begin{answer}},after={\end{answer}}]{answers}


\subsection{typestored}
\RecustomVerbatimEnvironment{verbatimsc}{Verbatim}{frame=none,numbers=left}
\typestored[1]{answers}
\typestored[2]{answers}
% nested
\subsection{Nested}

\begin{scontents}[print-env=false, store-env=outer]
This text is in the outermost environment (before nested).
\begin{scontents}[print-env=false, store-env=inner]
This text is found in the most inner environment.
\end{scontents}
This text is in the outermost environment (after nested).
\end{scontents}

\subsubsection{getstored}
% \getstored[1]{outer}

\getstored[1]{outer}\par
\getstored[1]{inner}\par

\subsubsection{count}
% count
Number of items stored in  \verb+outer+ \countsc{outer}\par
Number of items stored in  \verb+inner+ \countsc{inner}

% \newenvsc{scnest}[store-env=nest]
\subsection{newenvsc}
\newenvsc{scnest}[store-env=nest,print-env=false]

\begin{scnest}
This text is in the outermost environment (before nested).
\begin{scnest}[store-env=innernest]
This text is found in the most inner environment.
\end{scnest}
This text is in the outermost environment (after nested).
\end{scnest}

\getstored{nest}\par
\getstored{innernest}

% write-cmd and write-out
\subsection{write-cmd}
% write-out
Text
\Scontents*[write-out=one.ltx]{Something
A \TeX \textbf{B} and tab 		??? Z.}
% write-cmd
\Scontents*[store-cmd=writekey,overwrite,write-cmd=writekey1.tsc]{Something		  \verb|B|.}
\Scontents*[store-cmd=writekey,write-cmd=writekey2.tsc]+Something  		\verb|C|.+
Text
% count
Number of items stored in  \verb+writekey+ \countsc{writekey}.\par
% foreachsc
\foreachsc[sep={\\[10pt]}]{writekey}
% getstored
\getstored[1]{writekey}\par
\getstored{writekey}

\subsection{envname=maybePrint}
\newenvsc{maybePrint}[store-env=maybePrint,print-env=false]
START
\begin{maybePrint}
  $\langle$environment contents$\rangle$
\end{maybePrint}
TEXT

% \setupsc[envname=maybePrint]{print-env=true}
\setupsc{print-env=true}

START
\begin{maybePrint}
  $\langle$environment contents$\rangle$
\end{maybePrint}
TEXT

START
\begin{maybePrint}
  $\langle$environment contents$\rangle$
\end{maybePrint}
TEXT
\subsection{mergesc}
\setupsc{print-env=false}
% Fix part of a MCE that should go before babel's loading
\begin{scontents}[store-env=mce]
\documentclass[french]{article}
\usepackage[T1]{fontenc}
\usepackage[utf8]{inputenc}
\usepackage{lmodern}
\usepackage[a4paper]{geometry}
\end{scontents}
% Fix part of a MCE that should go after (>=) babel's loading
\begin{scontents}[store-env=mce]
\usepackage{babel}
\begin{document}
\end{scontents}
% Fix part of a MCE that should go after its body
\begin{scontents}[store-env=mce]
\end{document}
\end{scontents}
%
\section{First annswer}
% Variable part of a MCE that should added to the fixed preamble, before babel's
% loading
\begin{scontents}[store-env=mce-1]
\usepackage{amsmath}
\end{scontents}
% Variable part of a MCE being the code snippet
\begin{scontents}[store-env=mce-1]
\begin{align}
  0 & \neq 1 \\
  1 & \neq 0
\end{align}
\end{scontents}

\begin{description}
\item[Preamble's addition]\leavevmode
  \typestored[1]{mce-1}
\item[Code snippet]\leavevmode
  \typestored[2]{mce-1}
\item[MCE]\leavevmode
  \mergesc[typestored, print-cmd=true]
    {
      {mce}[1], {mce-1}[1], {mce}[2], {mce-1}[2], {mce}[3]
    }
\end{description}

\section{Second annswer}
% Variable part of a MCE that should added to the fixed preamble, before babel's
% loading
\begin{scontents}[store-env=mce-2]
\usepackage{amsmath}
\end{scontents}
% Variable part of a MCE being the code snippet
\begin{scontents}[store-env=mce-2]
\begin{flalign}
  0 & \neq 1 \\
  1 & \neq 0
\end{flalign}
\end{scontents}

\begin{description}
\item[Preamble's addition]\leavevmode
  \typestored[1]{mce-2}
\item[Code snippet]\leavevmode
  \typestored[2]{mce-2}
\item[MCE]\leavevmode
  \mergesc[typestored, print-cmd=true, write-out=mce.txt, overwrite=true]
    {
      {mce}[1], {mce-2}[1], {mce}[2], {mce-2}[2], {mce}[3]
    }
\end{description}
\end{document}
